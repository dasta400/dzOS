% ==========================================================================
\section{Setting up the system}
% ==========================================================================
\label{sec:setting_system}

% ==========================================================================
\subsection{dastaZ80 Original}
% ==========================================================================

You will only need:

\begin{itemize}
    \item The dastaZ80 computer.
    \item A 5 Volts (4 Amp) power supply with a female 2.1mm barrel-style DC
    connector (positive polarity).
    \item A Micro SD card.
    \item A monitor with VGA input.
    \item A LIR2032 Li-Ion Rechargeable 3.6V Battery Button Cell.
    \item Optionally:
    \begin{itemize}
        \item A monitor with NTSC Composite input.
        \item A 3.5mm jack to 3 RCA Audio/Video cable\footnote{This is the same
            cable used on the Raspberry Pi for Composite output}.
    \end{itemize}
\end{itemize}

% ==========================================================================
\subsection{dastaZ80DB}
% ==========================================================================

You will need:

\begin{itemize}
    \item The dastaZ80DB computer.
    \item Another computer, with serial port (either USB or RS-232).
    \begin{itemize}
        \item For USB, you will need a TTL-to-USB cable.
        \item For RS-232, you will need a TTL-to-RS-232 converter and a RS-232
            null modem cable.
    \end{itemize}
    \item A 5 Volts (4 Amp) power supply with a female 2.1mm barrel-style DC
    connector (positive polarity).
    \item A Micro SD card.
    \item A LIR2032 Li-Ion Rechargeable 3.6V Battery Button Cell
    \item Optionally:
    \begin{itemize}
        \item A monitor with VGA input.
        \item A monitor with NTSC Composite input.
        \item A 3.5mm jack to 3 RCA Audio/Video cable\footnote{This is the same
            cable used on the Raspberry Pi for Composite output}.
    \end{itemize}
\end{itemize}

% ==========================================================================
\subsection{Lets put it together}
% ==========================================================================

\begin{enumerate}
    \item Insert the battery LIR2032 in the battery holder. Positive goes up.
    \item On a modern PC, format the SD card with FAT32.
    \item Create a file of 33 MB on the SD card.
    \begin{itemize}
        \item For example using Linux terminal: \texttt{fallocate -l \$((33*1024*1024)) dastaZ80.img}
    \end{itemize}
    \item Create a file named \textit{\_disks.cfg} in the root of the SD card,
    and add two lines to it:
    \begin{itemize}
        \item dastaZ80.img
        \item \#
    \end{itemize}
    \item Introduce the SD card in the SD card slot at the back of the
        computer case. This procedure MUST be performed with the computer
        switched off. For the dastaZ80DB, the SD card slot is at the front of
        the case.
    \item Connect the jack of the Audio/Video cable to the A/V connector at
        the back of the computer case. This procedure SHOULD be performed with
        the computer unplugged.
    \item Connect the female power supply connector to the male connector at
        the back of the case.
    \item For the dastaZ80 Original, connect the VGA cable from the monitor to
        the VGA connector at the back of the computer case. This procedure
        SHOULD be performed with the computer unplugged.
    \item For the dastaZ80DB, there are two options:
    \begin{itemize}
        \item Connect the TTL-to-USB (or TTL-to-RS-232) cable to the 4 pins
            header labelled \textit{TTL I/O}, to have the input (keyboard) and
            output (screen) from/to another computer.
        \item Or, connect the TTL-to-USB (or TTL-to-RS-232) cable to
            \textit{TTL I/O}, but also connect the VGA cable to the VGA
            connector. You will then need to configure the 
            \hyperref[subsubsec:controlpanel]{Control Panel} to use VGA output
            instead of TTL output. This way, the signal will be send to the VGA
            monitor and not to the other computer.
    \end{itemize}
\end{enumerate}

That’s really it. Switch the Power Switch to the ON position and you should see
text on your VGA monitor. The computer and DZOS are ready to use.

If you are using the dastaZ80DB with the TTL-to-USB (or TTL-to-RS-232) cable for
both input and output (i.e. no VGA), you need to open a terminal program in your
other computer and set it up to connect at 115,200bps 8N1 no hardware flow
control. After flipping the switch, you should see text on the terminal program.

Nevertheless, it's worth noting that by following this procedure the disk
image in the SD card will be empty, and therefore no software will be
available. See how to add software to your disk image in the following
subsection.

    % ==========================================================================
    \subsection{Transferring software to and from a disk image}
    % ==========================================================================

        % ==========================================================================
        \subsubsection{Copy to a disk image}
        % ==========================================================================

        Software for the dastaZ80 can be created on a PC, or downloaded for example
        from the \href{https://github.com/dasta400/dzSoftware}{DZOS Software repository}
        on Github\footnote{DZOS Software repository: 
        \url{https://github.com/dasta400/dzSoftware}}. But how do we transfer that
        software into a disk image on our SD card?

        Another DZOS related repository is the \href{https://github.com/dasta400/asmdc}
        {Arduino Serial Multi-Device Controller for dastaZ80's dzOS}, or ASMDC for
        short.

        In this repository there is a tool, called \textit{imgmngr} (Image Manager)
        which can be used to manipulate files inside disk images. It allows to add
        files, extract files, rename files, delete files, change the file attributes,
        and display the disk image contents.

        Lets imagine we have a binary, called \textit{helloworld}, in our PC, and we
        want to copy it to the disk image, called \textit{dastaZ80.img}, that we
        created following the previous steps explained in the
        \hyperref[sec:setting_system]{Setting up the system} section. Simply execute
        \textit{imgmngr} like: \texttt{imgmngr dastaZ80.img -add helloworld}

        Now, if we introduce the SD card into the SD Card slot of the dastaZ80,
        switch it on and execute the DZOS command \hyperref[cmd:cat]{cat}, we will
        see a file \textit{helloworld}.

        % ==========================================================================
        \subsubsection{Copy from a disk image}
        % ==========================================================================

        What if we created a file in DZOS and we want to make a backup or simply
        share it with somebody else?

        As we have seen, the same tool \textit{imgmngr} can also be used to extract
        files from a disk image: \texttt{imgmngr dastaZ80.img -get helloworld}

        Once finished, we will have a file \textit{helloworld} in our PC, with the
        same contents as the file \textit{helloworld} in the disk image.

    % ==========================================================================
    \subsection{Dual Video Output}
    % ==========================================================================

    The dastaZ80 has two simultaneous video outputs: VGA and Composite.

    The VGA output, called \textit{High Resolution}, is the default output for
    the Operating System. As it provides 80 columns, it is ideal for
    applications.

    The Composite output, called \textit{Low Resolution}, can only provide 40
    columns in Text Mode and 32 in Graphics Modes\footnote{This is a limitation
    imposed by the Video Display Controller used, a Texas Instruments TMS9918A.},
    making it less ideal for applications. But in contrast to the VGA output, it
    offers graphics and hardware sprites, so it makes it more suitable for video
    games and graphics output from applications.

    % ==========================================================================
    \subsection{Dual Joystick Port}
    % ==========================================================================
    The \textbf{Dual Digital Joystick Port} consist of two DE-9 Male connectors
    for connection of one or two Atari joystick ports.

    Compatible joysticks are those used on Atari 800, Atari VCS, Atari ST,
    VIC-20, C64, Amiga and ZX Spectrum. Other joysticks, like the ones for the
    Amstrad CPC and MSX, changed the +5V and GND pins, so it MUST not be used.