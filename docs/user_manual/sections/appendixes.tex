% ==========================================================================
\section{Appendixes}
% ==========================================================================
\label{sec:appendixes}

    % ==========================================================================
    \subsection{Floppy Disk Drive Error Codes}
    % ==========================================================================

    Extracted from: \url{https://github.com/dhansel/ArduinoFDC#troubleshooting}

    \begin{itemize}
        \item \textbf{0}: No error, the operation succeeded.
        \item \textbf{1}: Internal \textbf{ASMDC} error.
        \item \textbf{2}: No disk in drive or drive does not have power.
        \item \textbf{3}: Disk not formatted or not formatted for the correct
        density.
        \item \textbf{4}: Bad disk or unknown format or misaligned disk drive.
        \item \textbf{5}: Bad disk or unknown format.
        \item \textbf{6}: Bad disk or unknown format.
        \item \textbf{7}: Drive does not have power.
        \item \textbf{8}: Disk is write protected or bad disk.
        \item \textbf{9}: Disk is write protected.
    \end{itemize}

    % ==========================================================================
    \subsection{MS BASIC  Error Codes}
    % ==========================================================================

    Error codes are displayed with the message: \texttt{?$<$error\_code$>$ Error}.
    For example, \texttt{?SN Error} for a syntax error.

    \begin{itemize}
        \item \textbf{NF}: NEXT without FOR
        \item \textbf{SN}: Syntax error
        \item \textbf{RG}: RETURN without GOSUB
        \item \textbf{OD}: Out of DATA
        \item \textbf{FC}: Function call error
        \item \textbf{OV}: Overflow
        \item \textbf{OM}: Out of memory
        \item \textbf{UL}: Undefined line number
        \item \textbf{BS}: Bad subscript
        \item \textbf{DD}: Re-DIMensioned array
        \item \textbf{DZ}: Division by zero (/0)
        \item \textbf{ID}: Illegal direct
        \item \textbf{TM}: Type miss-match
        \item \textbf{OS}: Out of string space
        \item \textbf{LS}: String too long
        \item \textbf{ST}: String formula too complex
        \item \textbf{CN}: Can't CONTinue
        \item \textbf{UF}: UnDEFined FN function
        \item \textbf{MO}: Missing operand
        \item \textbf{HX}: HEX error
        \item \textbf{BN}: BIN error
        \item \textbf{LE}: File not found while LOADing
    \end{itemize}

    % ==========================================================================
    \subsection{Low Resolution Screen Modes}
    % ==========================================================================
    \label{sec:vdpscrmodes}

    \begin{itemize}
        \item Mode 0: \textbf{Text Mode}
        \begin{itemize}
            \item 40 columns by 24 lines.
            \item 6x8 bytes characters.
            \item 2 colours (Text and Background).
            \item No Sprites.
        \end{itemize}
        \item Mode 1: \textbf{Graphics I Mode}
        \begin{itemize}
            \item 256 by 192 pixels.
            \item 32 columns by 24 lines for text.
            \item 8x8 bytes characters for text.
            \item 15 colours.
            \item Sprites.
        \end{itemize}
        \item Mode 2: \textbf{Graphics II Mode}
        \begin{itemize}
            \item 256 by 192 pixels.
            \item 32 columns by 24 lines for text.
            \item 8x8 bytes characters for text.
            \item 15 colours.
            \item Sprites.
        \end{itemize}
        \item Mode 3: \textbf{Multicolour Mode}
        \begin{itemize}
            \item 64 by 48 pixels.
            \item No characters for text.
            \item 15 colours.
            \item Sprites.
        \end{itemize}
        \item Mode 4: \textbf{Graphics II Mode Bitmapped}: same as mode 2, but
        screen is bitmapped for addressing every pixel individually.
    \end{itemize}

    % ==========================================================================
    \subsection{Useful pokes}
    % ==========================================================================

    Some of the OS behaviour can be modified by simply changing values stored in
    \textbf{RAM}.
    
    This can be done with the command \hyperref[cmd:poke]{poke}.

    \begin{itemize}
        \item \texttt{poke 40AC,01} - Show deleted files with the command
            \hyperref[cmd:cat]{cat}.
        \item \texttt{poke 40AC,00} - Hide deleted files with the command
            \hyperref[cmd:cat]{cat}. This is default value when the computer is
            turned ON or after a reset.
        \item \texttt{poke 4176,nn} - \textbf{DISK} number (\textit{nn})
            where the operations read and write will be performed.
        \item \texttt{poke 4195,nn} to \texttt{poke 41A3,nn} - Change the 
            colour in which messages will be displayed in the \textbf{High
            Resolution Display}. See \textit{dastaZ80 Programmer's Reference
            Guide}\cite{dastaz80progref} for a complete list of message types.
    \end{itemize}

    % ==========================================================================
    \subsection{How to copy files}
    % ==========================================================================

    At the moment, DZOS doesn't have a command for copying files. But this can
    be easily achieved using commands \hyperref[cmd:load]{load} and
    \hyperref[cmd:save]{save}.

    The \textit{load} command reads bytes from a file in a \textbf{DISK} and
    copies them into \textbf{RAM}. On the other hand, the \textit{save}
    command does exactly the opposite. Hence, we can  use these two commands to
    read the bytes of the file we want to copy, store these bytes in
    \textbf{RAM} and then store these bytes from \textbf{RAM} to \textbf{DISK}
    with a new filename.

    Let's look at an example: we have a file called \textit{testfile}, which is
    956 bytes long, in the disk 1 and we want to copy it to the disk 2.

    First, we need to be at the \textbf{DISK} unit were the original file is.
    For our example, it's disk 1. So, if we are not already in disk 1, we use
    the command \texttt{dsk 1}.

    Next, we load the file into \textbf{RAM} with the command
    \texttt{load testfile}. Once the file is loaded by the OS, it will tell us
    in which \textbf{MEMORY} address was stored. Lets imagine, for this example,
    that the address was \texttt{0x4420}.

    Next, we need to position to the destination disk (disk 2 in our example)
    with the command \texttt{dsk 2}.

    Finally, we use the command \texttt{save 4420 956} to instruct the OS to
    create a new file with 956 bytes from \textbf{RAM}, starting at address 
    \textbf{0x4420}. The OS will ask us for a filename, and proceed to perform 
    the saving. For our example we will keep the same name \textit{testfile}.

    Now, we have two exact copies of the file \textit{testfile}. One in
    \textbf{DISK} 1 and another in \textbf{DISK} 2.

    The only downside, a part from the fact that we need to write 3 to 4
    commands instead of just one, is that the \textit{File Attributes} will not
    be the same as the original file, as the OS has just created a new file. We
    can set the attributes with the command \hyperref[cmd:chgattr]{chgattr}.