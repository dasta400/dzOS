% ==========================================================================
\section{OS Commands}
% ==========================================================================
There are a number of commands included in the operating system. These
commands are stored in \textbf{MEMORY} at boot time, and therefore can be
called at any time from the command prompt.

Some commands may have mandatory and/or optional parameters. These
parameters MUST be entered in the order listed. Interchanging the order of
parameters will result on undesired behaviour.

Parameters can be separated either by a comma or a space. For clarity, in
this document all parameters are separated by a comma.

Programs stored in \textbf{DISK} can be executed directly by simply entering
the filename as a command. But only those in the current \textbf{DISK} (see
command \hyperref[cmd:dsk]{dsk}) and with attribute \textit{EXE}.

    % ==========================================================================
    \subsection{General Commands}\label{gencmds}
    % ==========================================================================
        
        % ======================================================================
        \subsubsection{{help}}
        % ======================================================================
        Shows a list of the most important commands available for the user.
        For a complete list of commands, refer always to this manual.

        \hspace{1.9cm}\textbf{$>$ help}

        \textbf{Parameters}: None

        % ======================================================================
        \subsubsection{{peek}}
        % ======================================================================
        \label{cmd:peek}

        Prints the value of the byte stored at a specified \textbf{MEMORY}
        address.

        \hspace{1.9cm}\textbf{$>$ peek \textit{$<$address$>$}}

        \textbf{Parameters}:

        \hspace{1cm}\textbf{\textit{address}}: address where the user wants
        to get the value from.

        \textbf{Example}: \texttt{$>$ peek 41A0}

        Will print (in hexadecimal) whatever byte is at location 0x41A0.

        % ======================================================================
        \subsubsection{{poke}}
        % ======================================================================
        \label{cmd:poke}

        Changes the value of the byte stored at a specified \textbf{MEMORY}
        address.

        \hspace{1.9cm}\textbf{$>$ poke \textit{$<$address$>$,$<$value$>$}}

        \textbf{Parameters}:

        \hspace{1cm}\textbf{\textit{address}}: address where the user wants
        to change a value.
        
        \hspace{1cm}\textbf{\textit{value}}: new value (in Hexadecimal notation)
        to be stored at \textit{address}.

        \textbf{Example}: \texttt{$>$ poke 41A0,2D}

        Will overwrite the contents of the address 0x41A0 with the value
        0x2D.

        % ======================================================================
        \subsubsection{{autopoke}}
        % ======================================================================
        Allows the user to enter a series of values to be stored at the
        starting address and its consecutive addresses. Think of it like a
        way to do poke but without having to enter the \textbf{MEMORY}
        address each time.

        After entering the command, a different command prompt, denoted by
        the symbol \$, will be displayed.

        Values are entered one by one after the symbol \$. Pressing 
        \textit{Return} with no value will end the command.

        \hspace{1.9cm}\textbf{$>$ autopoke \textit{$<$address$>$}}

        \textbf{Parameters}:

        \hspace{1cm}\textbf{\textit{address}}: address where the user wants
        to start changing values.

        \textbf{Example}: \texttt{$>$ autopoke 41A0}

        Will overwrite the contents of the address 0x41A0 with the first
        value entered by the user at the \$ prompt. Next value entered will
        overwrite the contents of the address 0x41A1, next 0x41A2, and so on
        until the end of the command.

        % ======================================================================
        \subsubsection{{halt}}
        % ======================================================================
        \label{cmd:halt}
        
        Tells the \textbf{DISK} controller to close all files, disables
        interrupts and puts the CPU in halted state, effectively making the
        computer unusable until next power cycle (\textit{Have you tried turning
        it off and on again?}).

        SHOULD be used before switching the computer off, to ensure all
        \textbf{DISK} data has been correctly saved. MUST not be used while the
        busy light of the \textbf{DISK} is on.

        IMPORTANT: \textbf{to use the computer again, you MUST turn it off and
        on again. Do NOT just press the reset button}. Otherwise corruption of
        data will occur, because the \textbf{DISK} controller is only reset at
        power on, and not when the \textbf{RESET} button is pressed.

        \hspace{1.9cm}\textbf{$>$ halt}

        \textbf{Parameters}: None

        % ======================================================================
        \subsubsection{{run}}
        % ======================================================================
        \label{cmd:run}

        Transfers the Program Counter (PC) of the Z80 to the specified address.
        In other words, this command is used to directly run code that has been
        already loaded in \textbf{RAM}, for example with the command 
        \hyperref[cmd:load]{load}.

        \hspace{1.9cm}\textbf{$>$ run \textit{$<$address$>$}}

        \textbf{Parameters}:

        \hspace{1cm}\textbf{\textit{address}}: address from where to start
        running.
        
        \textbf{Example}: \texttt{$>$ run 4420}

        The \textbf{CPU} will start running whatever instructions finds from
        \texttt{0x4420} and onwards. Programs run this way MUST end with a
        jump instruction (JP) to CLI prompt address, as described in the
        \textit{dastaZ80 Programmer’s Reference Guide}\cite{dastaz80progref}.
        Otherwise the user will have to reset the computer to get back to CLI.
        Not harmful but cumbersome.

        % ======================================================================
        \subsubsection{{crc16}}
        % ======================================================================
        Generates a CRC-16/BUYPASS1\footnote{A 16-bit cyclic redundancy
        check (CRC) based on the IBM Binary Synchronous Communications
        protocol\cite{ibmbsc} (BSC or Bisync). It uses the polynomial
        $X^{16} + X^{15} +X^2 + 1$}

        There are two formats of this command: 
        
        \textbf{crc16 \textit{$<$start\_address$>$ $<$end\_address$>$}}
        and \textbf{crc16 \textit{$<$filename$>$}} 
        
        Here the former is described. See the section \textit{\ref{dskcmds} 
        DISK Commands} on page \pageref{dskcmds} for the other format.

        \hspace{1.9cm}\textbf{$>$ crc16 \textit{$<$start\_address$>$
        $<$end\_address$>$}}
        
        \textbf{Parameters}:

        \hspace{1cm}\textbf{\textit{start\_address}}: first address from
        where the bytes to calculate the CRC will be read.

        \hspace{1cm}\textbf{\textit{end\_address}}: last address from where
        the bytes to calculate the CRC will be read.

        \textbf{Example}: \texttt{$>$ crc16 0000,0100}

        Will calculate the CRC of all bytes in MEMORY between the two
        specified address and show it on the screen:
        
        \hspace{1cm}\texttt{CRC16:\ 0x2F25}

        % ======================================================================
        \subsubsection{{clrram}}
        % ======================================================================
        Fills with zeros the entire Free RAM area (i.e. from 0x4420 to
        0xFFFF).

        \hspace{1.9cm}\textbf{$>$ clrram}

        \textbf{Parameters}: None

    % ======================================================================
    \subsection{Real-Time Clock (RTC) Commands}
    % ======================================================================
        % ======================================================================
        \subsubsection{{date}}
        % ======================================================================
        Shows the current date and day of the week from the Real-Time Clock
        (\textbf{RTC}).

        \hspace{1.9cm}\textbf{$>$ date}

        \textbf{Parameters}: None

        Will show (will differe depending on the date on the \textbf{RTC}):

        \hspace{1cm}\texttt{Today:\ 22/11/2022 Tue}

        % ======================================================================
        \subsubsection{{time}}
        % ======================================================================
        Shows the current time from the Real-Time Clock (\textbf{RTC}).

        \hspace{1.9cm}\textbf{$>$ time}

        \textbf{Parameters}: None

        Will show (will differe depending on the time on the \textbf{RTC}):

        \hspace{1cm}\texttt{Now:\ 16:24:36}

        % ======================================================================
        \subsubsection{{setdate}}
        % ======================================================================
        Changes the current date stored in the Real-Time Clock (\textbf{RTC}).

        \hspace{1.9cm}\textbf{$>$ setdate \textit{$<$yyyy$>$$<$mm$>$$<$dd$>$$<$dow$>$}}

        \textbf{Parameters}:

        \hspace{1cm}\textbf{\textit{yyyy}}: year (2 digits).

        \hspace{1cm}\textbf{\textit{mm}}: month.

        \hspace{1cm}\textbf{\textit{dd}}: day.

        \hspace{1cm}\textbf{\textit{yyyy}}: dow (Day of the Week. 1=Sunday).

        \textbf{Example}: \texttt{$>$ setdate 2211032}

        % ======================================================================
        \subsubsection{{settime}}
        % ======================================================================
        Changes the current time stored in the Real-Time Clock (\textbf{RTC}).

        \hspace{1.9cm}\textbf{$>$ settime \textit{$<$hh$>$$<$mm$>$$<$ss$>$}}

        \textbf{Parameters}:

        \hspace{1cm}\textbf{\textit{hh}}: hour.

        \hspace{1cm}\textbf{\textit{mm}}: minutes.

        \hspace{1cm}\textbf{\textit{ss}}: seconds.

        \textbf{Example}: \texttt{$>$ settime 185700}

    % ==========================================================================
    \subsection{Disk Commands}\label{dskcmds}
    % ==========================================================================
        % ======================================================================
        \subsubsection{{cat}}
        % ======================================================================
        \label{cmd:cat}

        Shows a catalogue of the files stored in the \textbf{DISK}.

        \hspace{1.9cm}\textbf{$>$ cat}

        \textbf{Parameters}: None

        \textbf{Example}: \texttt{$>$ cat}

        Will show (will differ depending on the contents of your \textbf{DISK}):

        \texttt{
        \resizebox{12cm}{!}{
            \begin{tabular}{l l l l l l}
                \multicolumn{6}{l}{Disk Catalogue}\\
                \multicolumn{6}{l}{----------------------------------------------------------------------}\\
                File & Type & Last Modified & Load Address & Attributes & Size\\
                \multicolumn{6}{l}{----------------------------------------------------------------------}\\
                HelloWorld & EXE & 12-03-2022 13:21:44 & 4420 & R SE & 38\\
                file2 & TXT & 11-05-2022 17:12:45 & 0000 & SE & 241
            \end{tabular}
        }}

        \hfill\break

        By default, deleted files are not shown in the catalogue. To show also
        deleted files do a \textit{poke 40ac, 01}. And a \textit{poke 40ac, 00}
        to hide them again.

        Deleted files are identified by a \textasciitilde symbol in the first
        character of the filename.

        \texttt{
        \resizebox{12cm}{!}{
            \begin{tabular}{l l l l l l}
                \multicolumn{6}{l}{Disk Catalogue}\\
                \multicolumn{6}{l}{----------------------------------------------------------------------}\\
                File & Type & Last Modified & Load Address & Attributes & Size\\
                \multicolumn{6}{l}{----------------------------------------------------------------------}\\
                \textasciitilde elloWorld & EXE & 12-03-2022 13:21:44 & 4420 & R SE & 38\\
                file2 & TXT & 11-05-2022 17:12:45 & 0000 & SE & 241
            \end{tabular}
        }}

        \hfill\break

        The File Types are:

        \begin{itemize}
            \item \textbf{USR}: User defined.
            \item \textbf{EXE}: Executable binary.
            \item \textbf{BIN}: Binary (non-executable) data.
            \item \textbf{BAS}: BASIC code.
            \item \textbf{TXT}: Plain ASCII Text file.
            \item \textbf{FN6}: Font (6×8) for Text Mode.
            \item \textbf{FN8}: Font (8×8) for Graphics Modes.
            \item \textbf{SC1}: Screen 1 (Graphics I Mode) Picture.
            \item \textbf{SC2}: Screen 2 (Graphics II Mode) Picture.
            \item \textbf{SC3}: Screen 3 (Multicolour Mode) Picture.
        \end{itemize}

        % ======================================================================
        \subsubsection{{erasedsk}}
        % ======================================================================
        Overwrittes all bytes of all sectors in a \textbf{DISK} in the
        \textbf{FDD}, with \texttt{0xF6} \\

        \underline{This is a destructive action} and it makes the \textbf{DISK}
        unusable to any (included dzOS) computer, as there is no file system in
        the disk after the command is completed.
        
        Before it can be used by dzOS, the command \textit{formatdsk} MUST be
        executed.

        It is recommended to only use this command in the case of wanting to
        destroy all data in a \textbf{DISK}, because \textit{formatdsk} doesn't
        actually delete any data, or to check if a Floppy Disk is faulty.
        Otherwise, the command \textit{formatdsk} SHOULD be the right command
        for normal usage of the computer.

        \hspace{1.9cm}\textbf{$>$ erasedsk}

        \textbf{Parameters}: None

        \textbf{Example}: \texttt{$>$ erasedsk}

        % ======================================================================
        \subsubsection{{formatdsk}}
        % ======================================================================
        Formats a \textbf{DISK} with DZFS format. \underline{This is a
        destructive action} and makes the \textbf{DISK} unsuable by any
        computers not using DZFS as their file system. It overwrites the DZFS
        \textit{Superblock} and \textit{BAT}.

        \hspace{1.9cm}\textbf{$>$ formatdsk \textit{$<$label$>$}}

        \textbf{Parameters}:

        \hspace{1cm}\textbf{\textit{label}}: a name given to the \textbf{DISK}.
        Useful for identifying different disks. It can contain any characters,
        with a maximum of 16.

        \textbf{Example}: \texttt{$>$ formatdsk mainDisk}

        Will format the SD card inserted in the SD card slot at the back of the
        computer case, having \textit{mainDisk} as disk label.

        % ======================================================================
        \subsubsection{{load}}
        % ======================================================================
        \label{cmd:load}

        Loads a file from \textbf{DISK} to \textbf{RAM}.
        
        The file will be loaded in \textbf{RAM} at the address from which it was
        originally saved. This address is stored in the DZFS BAT and cannot be
        changed. 

        \hspace{1.9cm}\textbf{$>$ load \textit{$<$filename$>$}}

        \textbf{Parameters}:

        \hspace{1cm}\textbf{\textit{filename}}: the name of the file that is to
        be loaded.

        \textbf{Example}: \texttt{$>$ load HelloWorld}

        Will load the contents (bytes) of the file \textit{HelloWorld} and copy
        them into the \textbf{RAM} address from which it was originally saved.

        % ======================================================================
        \subsubsection{{rename}}
        % ======================================================================
        Changes the name of a file.

        \hspace{1.9cm}\textbf{$>$ rename \textit{$<$current\_filename$>$,
        $<$new\_filename$>$}}

        \textbf{Parameters}:

        \hspace{1cm}\textbf{\textit{current\_filename}}: the name of the file as
        existing in the \textbf{DISK} at the moment of executing this command.
        
        \hspace{1cm}\textbf{\textit{new\_filename}}: the name that the file will
        have after the command is executed.

        \textbf{Example}: \texttt{$>$ rename HelloWorld,Hello}

        Will change the name of the file \textit{HelloWorld} to \textit{Hello}.

        % ======================================================================
        \subsubsection{{delete}}
        % ======================================================================
        Deletes a file from the \textbf{DISK}.

        Technically is not deleting anything but just changing the first
        character of the filename to a \textasciitilde symbol, which makes it to
        not show up with the command \textit{cat}. Hence, it can be undeleted by
        simply renaming the file. But be aware, when saving new files DZFS looks
        for a free space \footnote{By free space on the \textbf{DISK} we
        understand a Block in the DZFS BAT that was never used before by a
        file.} on the \textbf{DISK}, but if it does not find any it starts
        re-using space from files marked as deleted and hence overwriting data
        on the \textbf{DISK}.

        \hspace{1.9cm}\textbf{$>$ delete \textit{$<$filename$>$}}

        \textbf{Parameters}:

        \hspace{1cm}\textbf{\textit{filename}}: the name of the file that is to
        be deleted.
        
        \textbf{Example}: \texttt{$>$ delete HelloWorld}

        Will delete the file \textit{HelloWorld}.

        % ======================================================================
        \subsubsection{{chgattr}}
        % ======================================================================
        \label{cmd:chgattr}

        Files in DZFS can have any of the following attributes:
        
        \begin{itemize}
            \item \textbf{Read Only} (R): it cannot be overwritten, renamed or
            deleted.
            \item \textbf{Hidden} (H): it does not show up in the results
            produced by the command \textit{cat}.
            \item \textbf{System} (S): this is a file used by DZOS and it MUST
            not be altered.
            \item \textbf{Executable} (E): this is an executable file and can be
            run directly with the command \textit{run $<$filename$>$}.
        \end{itemize}

        \hspace{1.9cm}\textbf{$>$ chgattr \textit{$<$filename$>$,$<$RHSE$>$}}

        \textbf{Parameters}:

        \hspace{1cm}\textbf{\textit{filename}}: the name of the file to change
        the attributes.

        \hspace{1cm}\textbf{\textit{RHSE}}: the new attributes (see list above)
        that are to be set to the specified file. Attributes are actually not
        changed but re-assigned. For example, if you have a file with attribute
        \textit{R} and specified only \textit{E}, it will change from Read Only
        to Executable. In order to keep both, you MUST specify both values,
        \textit{RE}.
        
        \textbf{Example}: \texttt{$>$ chgattr HelloWorld,RE}

        Will set the attributes of the the file \textit{HelloWorld} to Read Only
        and Executable.

        % ======================================================================
        \subsubsection{{save}}
        % ======================================================================
        \label{cmd:save}

        Saves the bytes of specified \textbf{MEMORY} addresses to a new file in
        the \textbf{DISK}.

        \hspace{1.9cm}\textbf{$>$ save \textit{$<$start\_address$>$,
        $<$number\_of\_bytes$>$}}

        \textbf{Parameters}:

        \hspace{1cm}\textbf{\textit{$<$start\_address$>$}}: first address where
        the bytes that the user wants to save are located in \textbf{MEMORY}.

        \hspace{1cm}\textbf{\textit{$<$number\_of\_bytes$>$}}: total number of
        bytes, starting at \textit{start\_address} that will be saved to
        \textbf{DISK}.
        
        \textbf{Example}: \texttt{$>$ save 4420,512}

        Will create a new file, with the name entered by the user when prompted,
        with 512 bytes of the contents of \textbf{MEMORY} from 0x4420 to 0x461F.

        % ======================================================================
        \subsubsection{{dsk}}
        % ======================================================================
        \label{cmd:dsk}

        Changes current disk for all \textbf{DISK} operations.

        \hspace{1.9cm}\textbf{$>$ dsk \textit{$<$n$>$}}

        \textbf{Parameters}:

        \hspace{1cm}\textbf{\textit{$<$n$>$}}: \textbf{DISK} number.

        \textbf{Example}: \texttt{$>$ dsk 0}

        Will change to \textbf{FDD}, and all the \textbf{DISK} operations will
        be performed in the \textbf{FDD} until the next boot or a new \textit{dsk}
        command.

        The CLI prompt changes to indicate which disk is in use.

        % ======================================================================
        \subsubsection{{diskinfo}}
        % ======================================================================
        Shows some information about the \textbf{DISK}.

        \hspace{1.9cm}\textbf{$>$ diskinfo}

        \textbf{Parameters}: None

        \textbf{Example}: \texttt{$>$ diskinfo}

        Will show (will differ depending on the contents of your \textbf{DISK}):

        \texttt{
        \resizebox{12cm}{!}{
            \begin{tabular}{l r l l}
                \multicolumn{4}{l}{Disk Information}\\
                & Volume .\ .\ : & dastaZ80 Main & (S/N: 352A15F2)\\
                & File System: & DZFSV1 &\\
                & Created on : & 03/10/2022 14:22:32 &\\
                & Partitions : & 01 &\\
                & Bytes per Sector: & 512 &\\
                & Sectors per Block: & 64 &
            \end{tabular}
        }}

        % ======================================================================
        \subsubsection{{disklist}}
        % ======================================================================
        Shows a list of all available \textbf{DISK} (\textbf{FDD} and Disk Image
        Files on the \textbf{SD}).

        \hspace{1.9cm}\textbf{$>$ disklist}

        \textbf{Parameters}: None

        \textbf{Example}: \texttt{$>$ disklist}

        Will show (will differ depending on the Disk Image Files on your
        \textbf{DISK}):

        \texttt{
        \resizebox{6cm}{!}{
            \begin{tabular}{l l l r}\\
                & DISK0 & FDD & \\
                & DISK1 & dastaZ80.img & 128 MB \\
                & DISK2 & msbasic.img   & 32 MB \\
                & DISK3 & empty.img     & 16 MB
            \end{tabular}
        }}

        \hfill\break

        \textbf{IMPORTANT}: When the list (210 bytes in total, for a maximum of
        15 Disk Image Files) is retrieved from the \textbf{ASMDC}, dzOS stores
        it at the very bottom of the RAM (\texttt{0xFF2D}). In case that you may
        have a program loaded that uses those low bytes, after executing the
        \textit{disklist} command the program will be corrupted.

        % % ======================================================================
        % \subsubsection{{crc16}}
        % % ======================================================================
        % Generates a CRC-16/BUYPASS1\footnote{A 16-bit cyclic redundancy
        % check (CRC) based on the IBM Binary Synchronous Communications
        % protocol (BSC or Bisync). It uses the polynomial
        % $X^{16} + X^{15} +X^2 + 1$}

        % There are two formats of this command: 
        
        % \textbf{crc16 \textit{$<$start\_address$>$ $<$end\_address$>$}}
        % and \textbf{crc16 \textit{$<$filename$>$}} 
        
        % Here the latter is described. See the section \textit{\ref{gencmds} 
        % General Commands} on page \pageref{gencmds} for the other format.

        % \hspace{1.9cm}\textbf{$>$ crc16 \textit{$<$filename$>$}}
        
        % \textbf{Parameters}:

        % \hspace{1cm}\textbf{\textit{filename}}: the name of the file for which
        % the CRC will be calculated.

        % \textbf{Example}: \texttt{$>$ crc16 HelloWorld}

        % Will calculate the CRC of all bytes on the file and show it on the
        % screen:
        
        % \hspace{1cm}\texttt{CRC16:\ 0x3ABC}

    % ==========================================================================
    \subsection{VDP (\textit{Low Resolution Screen}) Commands}\label{vdpcmds}
    % ==========================================================================
        % ======================================================================
        \subsubsection{{vpoke}}
        % ======================================================================
        \label{cmd:vpoke}

        Changes the value of the byte stored at a specified \textbf{VRAM}
        address.

        \hspace{1.9cm}\textbf{$>$ vpoke \textit{$<$address$>$,$<$value$>$}}

        \textbf{Parameters}:

        \hspace{1cm}\textbf{\textit{address}}: address where the user wants
        to change a value.
        
        \hspace{1cm}\textbf{\textit{value}}: new value (in Hexadecimal notation)
        to be stored at \textit{address}.

        \textbf{Example}: \texttt{$>$ vpoke 3800,00}

        Will overwrite the contents of the address 0x3800 of the \textbf{VRAM}
        with the value 0x00.

        % ======================================================================
        \subsubsection{{screen}}
        % ======================================================================
        Changes the \textbf{Low Resolution Screen} display mode.

        \hspace{1.9cm}\textbf{$>$ screen \textit{$<$mode$>$}}

        \textbf{Parameters}:

        \hspace{1cm}\textbf{\textit{mode}}: one of the valid
        \hyperref[sec:vdpscrmodes]{\textbf{Low Resolution} Screen Modes}:

        \begin{itemize}
            \item \textbf{0}: \textbf{Text Mode}.
            \item \textbf{1}: \textbf{Graphics I Mode}.
            \item \textbf{2}: \textbf{Graphics II Mode}.
            \item \textbf{3}: \textbf{Multicolour Mode}.
            \item \textbf{4}: \textbf{Graphics II Mode Bitmapped}.
        \end{itemize}
        
        \textbf{Example}: \texttt{$>$ screen 0}

        Will put the \textbf{Low Resolution Screen} in Text Mode.

        % ======================================================================
        \subsubsection{{clsvdp}}
        % ======================================================================
        Clears the contents of the \textbf{Low Resolution Screen}.

        \hspace{1.9cm}\textbf{$>$ clsvdp}

        \textbf{Parameters}: None