% ==========================================================================
\section{MS BASIC 4.7b}
% ==========================================================================
\label{sub:msbasic}

The Nascom 2 computer\footnote{The Nascom 2 was a single-board computer kit
issued in the United Kingdom in December 1979.} came with MS BASIC 4.7
installed in ROM, and the disassembled code was published in the
\textit{80-BUS NEWS} magazine: \cite{80busnews23}, \cite{80busnews24},
\cite{80busnews25}, \cite{80busnews26}, \cite{80busnews31},
\cite{80busnews32}, \cite{80busnews33}.

Grant Searle published a modification (version 4.7b)in his Grant's \textit{7-chip Z80
computer} webpage\cite{searle2}.

Grant's version was then modified (version 4.7b(dz\textit{maj}.\textit{min}.\textit{patch}))
to run on dastaZ80 under dzOS, and adding more commands.

    % ==========================================================================
    \subsection{Differences NASCOM MS BASIC v4.7 and Grant Searle's v4.7b}
    % ==========================================================================

    \begin{itemize}
        \item Added \hyperref[msbasic:lang:hex]{HEX\$(nn)}
        \item Added \hyperref[msbasic:lang:hex]{BIN\$(nn)}
        \item Added \&Hnn for interpreting numbers as Hexadecimal
        \item Added \&Bnn for interpreting numbers as Binary
        \item Removed SCREEN
        \item Removed CLOAD
        \item Removed CSAVE
    \end{itemize}

    % ==========================================================================
    \subsection{Differences Grant Searle's 4.7b and dastaZ80
    v4.7b(dz\textit{maj}.\textit{min}.\textit{patch})}
    % ==========================================================================

    \begin{itemize}
        \item Added \hyperref[msbasic:lang:cat]{CAT}
        \item Added \hyperref[msbasic:lang:colour]{COLOUR}
        \item Added \hyperref[msbasic:lang:load]{LOAD}
        \item Added \hyperref[msbasic:lang:save]{SAVE}
        \item Added \hyperref[msbasic:lang:screen]{SCREEN}
        \item Added \hyperref[msbasic:lang:spoke]{SPOKE}
        \item Added \hyperref[msbasic:lang:vpeek]{VPEEK}
        \item Added \hyperref[msbasic:lang:vpoke]{VPOKE}
    \end{itemize}

    % ==========================================================================
    \subsection{MS BASIC characteristics}
    % ==========================================================================

    \begin{itemize}
        \item Commands
        \begin{itemize}
            \item There is no support for \textit{ELSE} in \textit{IF}...
            \textit{THEN}. Instead, it MUST be done with another \textit{IF}...
            \textit{THEN}.
        \end{itemize}
        \item Variables
        \begin{itemize}
            \item The first character of a variable name MUST be a letter.
            \item No reserved words may appear as as variable names.
            \item Can be of any length, but any alphanumeric characters after
            the first two are ignored. Therefore \textit{COURSE},
            \textit{COLOUR} and\textit{COMIC} are the same variable because all
            start with textit{CO}.
            \item Integer numbers are signed (i.e. from -32,768 to +32,767). To
            refer to a location \textit{n} above 32,767, you must provide the
            2's complement number (i.e \textit{n}-65536).
            \item Flotaing point is in the range 1.70141E38 to 2.9387E-38
        \end{itemize}
    \end{itemize}

    % ==========================================================================
    \subsection{Speeding up programs}
    % ==========================================================================

    \begin{itemize}
        \item Delete \textit{REM} statements.
        \item Delete spaces. For example, \textit{10FORA=0TO10} is faster than
        \textit{10 FOR A=0 TO 10}
        \item Use \textit{NEXT} without the index variable.
        \item Use variables instead of constants, especially in \textit{FOR}
        loop and other code that is executed repeatedly.
        \item Reuse variable names and keep the list of variables as short as
        possible. Variables are set up in a table in the order of their first
        appearance in the program. Later in the program, BASIC searches the
        table for the variable at each reference. Variables at the head of the
        table take less time to search.
        \item MS BASIC uses a \textit{garbage collector} to clear out unwanted 
        space. The frequency of grabage collection is inversely proportional to
        the amount of string space. The time garbage collection takes is
        proportional to the square of the number of string variables. To
        minimise the time, make string space as large as possible and use as few
        string variables as possible.
    \end{itemize}

    % ==========================================================================
    \subsection{Operators}
    % ==========================================================================

    \begin{itemize}
        \item \textbf{+} Addition
        \item \textbf{-} Subtraction
        \item \textbf{*} Multiplication
        \item \textbf{/} Division
        \item \textbf{\textasciicircum} Power of
    \end{itemize}

    % ==========================================================================
    \subsection{Relational Operators}
    % ==========================================================================

    \begin{itemize}
        \item \textbf{$>$}, \textbf{$<$}, \textbf{$<>$}, \textbf{=},
        \textbf{$<=$}, \textbf{$>=$}
    \end{itemize}

    % ==========================================================================
    \subsection{Logical Operators}
    % ==========================================================================

    \begin{itemize}
        \item \textbf{AND}, \textbf{NOT}, \textbf{OR}
    \end{itemize}

    % ==========================================================================
    \subsection{Operators precedence}
    % ==========================================================================
    Operators are shown in decressing order of precerdence.

    \begin{itemize}
        \item Expressions enclosed in parenthesis
        \item Exponent \textasciicircum
        \item Negation -
        \item Multiplication and Division
        \item Addition and subtraction
        \item Equal
        \item Not equal <>, ><
        \item Less than <
        \item Greater than >
        \item less than or equal to <=, =<
        \item greater than or equal to >=, =<
        \item NOT
        \item AND
        \item OR
    \end{itemize}

    % ==========================================================================
    \subsection{How to call an ASM subroutine}
    % ==========================================================================

    This BASIC provides a way of executing external subroutines, via the
    intrinsic function \textit{USR}.

    The programmer needs to store the address of the subroutine to be called in
    the the work space location reserved for \textit{USR}. In the case of the
    version for dastaZ80, this is at \texttt{0x6148} for the LSB and
    \texttt{0x6149} for the MSB.

    This can be done from BASIC with the instruction \texttt{DPOKE 24904,<address>}

    \textbf{Be aware} that at location \texttt{0x6147} there is stored a
    \textit{jp} instruction, which is what is executed when the function
    \textit{USR} is called from BASIC, and therefore it will jump to the
    subroutine and never come back unless explicitily specified.

    If instead your subroutine contains a \textit{ret}urn instruction, or if you
    are calling dzOS functions, you \textbf{MUST} change the \textit{jp}
    instruction to a \textit{call} instruction.

    This can be done from BASIC with the instruction \texttt{POKE 24903,205}

    Finally, to call the external subroutine, as the \textit{USR} is a function,
    it returns a parameter and therefore it must be received. Either by assigning
    the value to a variable (e.g. \texttt{A=USR(0)}) or by printing it or by
    checking it with an \textit{IF}.

    \begin{itemize}
        \item Valid methods how to use \textit{USR}:
        \begin{itemize}
            \item \texttt{A=USR(0)}
            \item \texttt{IF USR(0)<>0 THEN ...}
            \item \texttt{PRINT USR(0)}
        \end{itemize}
        \item Invalid methods:
        \begin{itemize}
            \item \texttt{USR(0)}, will return \textit{?SN Error} (i.e. Syntax Error)
        \end{itemize}
    \end{itemize}

    % ==========================================================================
    \subsection{Language Reference}
    % ==========================================================================
    Following are listed in alphabetical order the intrinsic functions,
    statements and commands available in the modified version for dastaZ80.

    % ======================================================================
    \subsubsection{{ABS (Absolute)}}
    \label{msbasic:lang:abs}
    % ======================================================================
    Returns absolute value (i.e. no sign) of an expression.

    \hspace{1.9cm}\textbf{ABS(\textit{x})}

    \textbf{Parameters}:

    \hspace{1cm}\textbf{\textit{x}}: expression.

    \textbf{Examples}:
    \begin{itemize}
        \item \texttt{PRINT ABS(4.75)} prints 4.75
        \item \texttt{A=ABS(-45)} A equals 45
    \end{itemize}

    % ======================================================================
    \subsubsection{{ASC (ASCII code)}}
    \label{msbasic:lang:asc}
    % ======================================================================
    Returns the ASCII code of the first character of a string.

    \hspace{1.9cm}\textbf{ASC(\textit{x\$})}

    \textbf{Parameters}:

    \hspace{1cm}\textbf{\textit{x\$}}: string.

    \textbf{Examples}:
    \begin{itemize}
        \item \texttt{PRINT ASC("D")} prints 68
        \item \texttt{A=ASC("DAVE")} A equals 68
    \end{itemize}

    % ======================================================================
    \subsubsection{{ATN (Arctangent)}}
    \label{msbasic:lang:atn}
    % ======================================================================
    Returns arctangent in radians (range -\textpi/2 to \textpi/2) of a number.

    \hspace{1.9cm}\textbf{ATN(\textit{x})}

    \textbf{Parameters}:

    \hspace{1cm}\textbf{\textit{x}}: number.

    \textbf{Examples}:
    \begin{itemize}
        \item \texttt{PRINT ATN(4)} prints 1.32582
        \item \texttt{A=ATN(4)} A equals 1.32582
    \end{itemize}

    % ======================================================================
    \subsubsection{{BIN\$ (Convert to Binary)}}
    \label{msbasic:lang:bin}
    % ======================================================================
    Converts an expression into a string representing the binary value.

    \hspace{1.9cm}\textbf{BIN\$(\textit{x})}

    \textbf{Parameters}:

    \hspace{1cm}\textbf{\textit{x}}: expression to convert to Binary.

    \textbf{Examples}:
    \begin{itemize}
        \item \texttt{PRINT BIN\$(10)} prints 1010
        \item \texttt{PRINT BIN\$(1000)} prints 1111101000
    \end{itemize}

    % ======================================================================
    \subsubsection{{CAT (Disk Catalog)}}
    \label{msbasic:lang:cat}
    % ======================================================================
    Shows a list of BASIC programs in the current disk. Only files of type
    BASIC are listed, any other files are ignored.

    \hspace{1.9cm}\textbf{CAT}

    \textbf{Parameters}: None

    % ======================================================================
    \subsubsection{{CHR\$ (ASCII to character)}}
    \label{msbasic:lang:chr}
    % ======================================================================
    Returns a one character string containing the character corresponding to
    the specified ASCII code.

    \hspace{1.9cm}\textbf{CHR\$(\textit{x})}

    \textbf{Parameters}:

    \hspace{1cm}\textbf{\textit{x}}: number from ASCII table.

    \textbf{Examples}:
    \begin{itemize}
        \item \texttt{PRINT CHR\$(65)} prints A
        \item \texttt{A=CHR\$(65)} A equals A
    \end{itemize}

    % ======================================================================
    \subsubsection{{CLEAR (Byte from RAM)}}
    \label{msbasic:lang:clear}
    % ======================================================================
    Sets all program variables to zero and all strings to empty string.

    \hspace{1.9cm}\textbf{CLEAR [\textit{s},\textit{t}]}

    \textbf{Parameters}:

    \hspace{1cm}\textbf{\textit{s}}: optionally, if \textit{s} is present and is
    a numeric expression, it sets the string area to \textit{s}. Default is 100.
    Must be a value between -32768 and 32767.

    \hspace{1cm}\textbf{\textit{t}}: optionally, if \textit{t} is present, it
    sets the top of the \textit{RAM} available for MS BASIC to \textit{t}.
    Must be a value between -32768 and 32767.

    \textbf{Examples}:
    \begin{itemize}
        \item \texttt{CLEAR} clears all program variables and strings
        \item \texttt{CLEAR 150} sets the string area to 150 characters
        \item \texttt{CLEAR 150,-30000} sets the string area to 150 characters,
        and assigns 30 KB of \textit{RAM} available for MS BASIC
    \end{itemize}

    % ======================================================================
    \subsubsection{{CLS (Clear Screen)}}
    \label{msbasic:lang:cls}
    % ======================================================================
    Clears the screen and sets the cursor to the top line.

    \hspace{1.9cm}\textbf{CLS}

    \textbf{Parameters}: None

    % ======================================================================
    \subsubsection{{COLOUR (Foreground/Background colours)}}
    \label{msbasic:lang:colour}
    % ======================================================================
    Changes the foreground and background colours of the \textbf{Low
    Resolution Display} screen.

    \hspace{1.9cm}\textbf{COLOUR \textit{f},\textit{b}}

    \textbf{Parameters}:

    \hspace{1cm}\textbf{\textit{f}}: number representing one
    of the available \textbf{VDP} colours.

    \hspace{1cm}\textbf{\textit{b}}: number representing one
    of the available \textbf{VDP} colours.

    \textbf{Examples}:
    \begin{itemize}
        \item \texttt{COLOUR 16,4}
    \end{itemize}

    \textbf{Available VDP colours:}

    \begin{itemize}
        \item 0 = Black
        \item 1 = Red
        \item 2 = Green
        \item 3 = Yellow
        \item 4 = Blue
        \item 5 = Magenta
        \item 6 = Cyan
        \item 7 = White
        \item 8 = Bright Black (Grey)
        \item 9 = Bright Red
        \item 10 = Bright Green
        \item 11 = Bright Yellow
        \item 12 = Bright Blue
        \item 13 = Bright Magenta
        \item 14 = Bright Cyan
        \item 15 = Bright White
    \end{itemize}

    % ======================================================================
    \subsubsection{{CONT (Continue STOPped execution)}}
    \label{msbasic:lang:cont}
    % ======================================================================
    Continues program execution after \textit{Escape} key was pressed or a 
    \hyperref[msbasic:lang:stop]{STOP} or \hyperref[msbasic:lang:end]{END}
    statement has been executed. Execution resumes at the statement after
    the break occurred unless input from the terminal was interrupted, in which
    case execution resumes with the reprinting of the prompt (?)

    \hspace{1.9cm}\textbf{CONT}

    \textbf{Parameters}: None

    % ======================================================================
    \subsubsection{{COS (Cosine)}}
    \label{msbasic:lang:cos}
    % ======================================================================
    Returns cosine in radians (range -\textpi/2 to \textpi/2) of a number.

    \hspace{1.9cm}\textbf{COS(\textit{x})}

    \textbf{Parameters}:

    \hspace{1cm}\textbf{\textit{x}}: number.

    \textbf{Examples}:
    \begin{itemize}
        \item \texttt{PRINT COS(4)} prints -.653644
        \item \texttt{A=COS(20)} A equals .408082
    \end{itemize}

    % ======================================================================
    \subsubsection{{DATA (List of elements)}}
    \label{msbasic:lang:data}
    % ======================================================================
    Specifies data to be read by \hyperref[msbasic:lang:read]{READ} statement.
    List elements can be numbers or strings, and are separated by commas.

    \hspace{1.9cm}\textbf{DATA(\textit{element1, element2,...,elementn})}

    \textbf{Parameters}:

    \hspace{1cm}\textbf{\textit{element}}: a number or a string.

    \textbf{Examples}:
    \begin{itemize}
        \item \texttt{DATA 1,2,3}
        \item \texttt{DATA "A","B","C"}
        \item \texttt{DATA "dastaZ80","dzOS"}
    \end{itemize}

    % ======================================================================
    \subsubsection{{DEEK (Word from RAM)}}
    \label{msbasic:lang:deek}
    % ======================================================================
    Reads a word (2 bytes) from a pair of continguous memory locations.

    \hspace{1.9cm}\textbf{DEEK \textit{mem}}

    \textbf{Parameters}:

    \hspace{1cm}\textbf{\textit{mem}}: memory address where first byte will be
    read from. Second byte will be read from \textit{mem} + 1

    If \textit{x} or \textit{mem} are negative, they are interpreted as 65536 +
    \textit{X} or \textit{mem}.

    \textbf{Examples}:
    \begin{itemize}
        \item \texttt{A=DEEK(\&H4420)} the value found at \textit{RAM} address
        \textit{0x4420} is assigned to the variable A
    \end{itemize}

    % ======================================================================
    \subsubsection{{DEF FN (Define user-defined function)}}
    \label{msbasic:lang:deffn}
    % ======================================================================
    Defines a user-defined function. Definitions are restricted to one line 

    \hspace{1.9cm}\textbf{DEF FN\textit{name}(\textit{arg})=\textit{expression}}

    \textbf{Parameters}:

    \hspace{1cm}\textbf{\textit{name}}: function name.

    \hspace{1cm}\textbf{\textit{arg}}: argument to be passed to the function.

    \hspace{1cm}\textbf{\textit{expression}}: the function.

    To call the function \textit{name}, we just write its full name (i.e. FN
    followed by \textit{name}).

    \textbf{Examples}:
    \begin{itemize}
        \item \texttt{10 DEF FNKEL2CEL(KEL)=KEL-273.15}
        \item \texttt{20 PRINT FNKEL2CEL(28)}
    \end{itemize}

    % ======================================================================
    \subsubsection{{DIM (Allocate array)}}
    \label{msbasic:lang:dim}
    % ======================================================================
    Allocates space for array variables. More than one variable may be
    dimensioned by one DIM statement up to the limit of the line. The value of
    each expression gives the maximum subscript possible. The smallest subscript
    is 0. Without a DIM statement, an array is assumed to have maximum subscript
    of 10 for each dimension referenced.

    Array can be resized during program execution, by using a variable as its
    dimension.

    Elements in a strings array can be up to 255 characters.

    \hspace{1.9cm}\textbf{DIM \textit{name},\textit{dim1},\textit{dim2},...,
    \textit{dimn}}

    \textbf{Parameters}:

    \hspace{1cm}\textbf{\textit{name}}: name of the array.

    \hspace{1cm}\textbf{\textit{dim}}: dimension, count started from zero. Hence,
    DIM A(4) will contain five elements, from 0 to 4.

    \textbf{Examples}:
    \begin{itemize}
        \item \texttt{DIM A(4)}, 1 dimension array with five elements
        \item \texttt{DIM B(4,2)}, 2 dimensions array with 15 elements
        \item \texttt{DIM C(I)}, 1 dimension array with \textit{I} elements
        \item \texttt{DIM D\$(10)}, 1 dimension string array with 10 elements
        \item \texttt{DIM E(10),F(5),G(10)}, allocates three arrays (E, F and G)
    \end{itemize}

    % ======================================================================
    \subsubsection{{DOKE (Word to RAM)}}
    \label{msbasic:lang:doke}
    % ======================================================================
    Stores a word (2 bytes) into a pair of continguous memory locations.

    \hspace{1.9cm}\textbf{DOKE \textit{mem},\textit{w}}

    \textbf{Parameters}:

    \hspace{1cm}\textbf{\textit{w}}: word to be stored.

    \hspace{1cm}\textbf{\textit{mem}}: memory address where first byte of
    \textit{w} will be stored. Second byte will be stored at \textit{mem} + 1

    If \textit{x} or \textit{mem} are negative, they are interpreted as 65536 +
    \textit{X} or \textit{mem}.

    \textbf{Examples}:
    \begin{itemize}
        \item \texttt{DOKE \&H4420,\&H10AB}
        \item \texttt{DOKE 17440,4267}
    \end{itemize}

    % ======================================================================
    \subsubsection{{END (Terminate program)}}
    \label{msbasic:lang:end}
    % ======================================================================
    Terminates execution of a program.

    \textbf{Parameters}: None

    \hspace{1.9cm}\textbf{END}

    % ======================================================================
    \subsubsection{{EXP (Exponent)}}
    \label{msbasic:lang:exp}
    % ======================================================================
    Returns the mathematical constant \textit{e} (Euler’s number) to the power
    of a specified number.

    \hspace{1.9cm}\textbf{EXP(\textit{x})}

    \textbf{Parameters}:

    \hspace{1cm}\textbf{\textit{x}}: number.

    \textbf{Examples}:
    \begin{itemize}
        \item \texttt{PRINT EXP(4)} prints 54.5982
        \item \texttt{A=EXP(4)} A equals 54.5982
    \end{itemize}

    % ======================================================================
    \subsubsection{{FOR...NEXT...STEP (Loop execution)}}
    \label{msbasic:lang:fornext}
    % ======================================================================
    Allows repeated execution (loop) of the same statements.

    \hspace{1.9cm}\textbf{FOR \textit{index}=\textit{ini} TO \textit{end} 
    [STEP \textit{s}]}

    \textbf{Parameters}:

    \hspace{1cm}\textbf{\textit{index}}: variable used as index. It will be
    incremented by one each time the program reaches the \textit{NEXT}
    instruction.

    \hspace{1cm}\textbf{\textit{ini}}: initial value of \textit{index} when
    entering the \textit{FOR} loop.

    \hspace{1cm}\textbf{\textit{end}}: Once \textit{index} equals the same value
    as \textit{end}, the \textit{FOR} loop will no longer repeat instructions an
    execution will continue with the statement after \textit{NEXT}.

    \hspace{1cm}\textbf{\textit{s}}: Optionally, the default increment by 1 of
    \textit{index} can be modified. \textit{s} may be a positive or a negative
    number.

    The statement \textit{NEXT} marks the last instruction of each loop. In
    other words, all statements placed between \textit{FOR} and \textit{NEXT}
    will be repeated at each loop.

    The \textit{index} after \textit{NEXT} can be omitted, but for clarity it's
    usually included.

    \textbf{Examples}:
    \begin{itemize}
        \item \texttt{FOR I=0 TO 10:NEXT I}
        \item \texttt{FOR I=10 TO 0 STEP -1:NEXT}
    \end{itemize}

    % ======================================================================
    \subsubsection{{FRE (Available RAM)}}
    \label{msbasic:lang:fre}
    % ======================================================================
    Returns number of bytes in memory not being used by BASIC.

    \hspace{1.9cm}\textbf{FRE(\textit{0})} or \textbf{FRE(\textit{""})}

    \textbf{Parameters}:

    \hspace{1cm}\textbf{\textit{0}}: free space for program and variables.

    \hspace{1cm}\textbf{\textit{""}}: free space in string area.

    \textbf{Examples}:
    \begin{itemize}
        \item \texttt{PRINT FRE(0)} prints available free space for program and
        variables.
        \item \texttt{PRINT FREE("")} prints available free space in string area.
    \end{itemize}

    String area can be cleared with \texttt{CLEAR}.

    % ======================================================================
    \subsubsection{{GOSUB (Branch to subroutine)}}
    \label{msbasic:lang:gosub}
    % ======================================================================
    Unconditional branch to subroutine specified at line number.

    The difference with \hyperref[msbasic:lang:goto]{GOTO} is that
    \textit{GOSUB} will contine execution from next statement found after
    \textit{GOSUB}, once the branched subroutine ends with
    \hyperref[msbasic:lang:return]{RETURN}.

    \hspace{1.9cm}\textbf{GOSUB \textit{x}}

    \textbf{Parameters}:

    \hspace{1cm}\textbf{\textit{x}}: line number.

    \textbf{Examples}:
    \begin{itemize}
        \item \texttt{GOSUB 100}, will execute statements from line 100 until a
        \hyperref[msbasic:lang:return]{RETURN} statement is found, and then will
        continue execution from next statement found after \textit{GOSUB}
    \end{itemize}

    % ======================================================================
    \subsubsection{{GOTO (Branch to line)}}
    \label{msbasic:lang:goto}
    % ======================================================================
    Unconditional branch to specified line number.

    \hspace{1.9cm}\textbf{GOTO \textit{x}}

    \textbf{Parameters}:

    \hspace{1cm}\textbf{\textit{x}}: line number.

    \textbf{Examples}:
    \begin{itemize}
        \item \texttt{GOTO 100}, will continue execution from line 100 and on
    \end{itemize}

    % ======================================================================
    \subsubsection{{HEX\$ (Convert to Hexadecimal)}}
    \label{msbasic:lang:hex}
    % ======================================================================
    Converts an expression into a string representing the hexadecimal value.

    \hspace{1.9cm}\textbf{HEX\$(\textit{x})}

    \textbf{Parameters}:

    \hspace{1cm}\textbf{\textit{x}}: expression to convert to Hexadecimal.

    \textbf{Examples}:
    \begin{itemize}
        \item \texttt{PRINT HEX\$(10)} prints A
        \item \texttt{PRINT HEX\$(1000)} prints 3E8
    \end{itemize}


    % ======================================================================
    \subsubsection{{IF...GOTO (Branch if true)}}
    \label{msbasic:lang:ifgoto}
    % ======================================================================
    Same as \hyperref[msbasic:lang:ifthen]{IF...THEN} except \textit{GOTO} can
    only be followed by a line number.

    % ======================================================================
    \subsubsection{{IF...THEN (Branch/Execute if true)}}
    \label{msbasic:lang:ifthen}
    % ======================================================================
    Executes all statements after \textit{THEN} and until the end of the line,
    if a condition is evaluated as true. Otherwise, execution proceeds at the
    line after \textit{IF...THEN}

    \hspace{1.9cm}\textbf{IF \textit{condition} THEN \textit{statement}}

    \textbf{Parameters}:

    \hspace{1cm}\textbf{\textit{condition}}: condition to test.

    \hspace{1cm}\textbf{\textit{statement}}: statement to execute if
    \textit{condition} evaluated as true. Instead of an statement after, a line
    number can be used effectively doing the same as
    \hyperref[msbasic:lang:ifgoto]{IF...GOTO}

    \textbf{Examples}:
    \begin{itemize}
        \item \texttt{IF A=10 THEN B=A}, only if A was equal to 10, A is
        assigned to B
        \item \texttt{IF A=10 THEN 100}, only if A was equal to 10, execution
        continues at line 100
        \item \texttt{IF A=10 THEN B=A:C=A:D=A}, only if A was equal to 10, A is
        assigned to B, C and D
        \item \texttt{IF A=10 THEN PRINT"A is 10":GOTO 100}, only if A was equal
        to 10, \textit{A is 10} is printed and then execution continues at line
        100
    \end{itemize}

    % ======================================================================
    \subsubsection{{INP (Read from Port)}}
    \label{msbasic:lang:inp}
    % ======================================================================
    Reads a byte from a specified port.

    \hspace{1.9cm}\textbf{INP(\textit{p})}

    \textbf{Parameters}:

    \hspace{1cm}\textbf{\textit{p}}: port number. Must be a value between 0 and
    255.

    \textbf{Examples}:
    \begin{itemize}
        \item \texttt{PRINT INP(80)} prints byte from port 80.
        \item \texttt{A=INP(80)} A equals byte read from port 80.
    \end{itemize}

    % ======================================================================
    \subsubsection{{INPUT (Read from terminal)}}
    \label{msbasic:lang:input}
    % ======================================================================
    Requests input from user and assignes typed values to a variable or variable
    list. It can also be used to print a text before the value is typed.

    A question mark symbol (?) is always printed to indicated that a value must
    be typed. If more than one value is requested, a double question mark symbol
    (??) will be printed.

    \hspace{1.9cm}\textbf{INPUT(\textit{v1},[\textit{v2}],...,\textit{vn})}

    \textbf{Parameters}:

    \hspace{1cm}\textbf{\textit{v}}: variable or variables to assign type values
    followed by \textit{RETURN} key.

    \textbf{Examples}:
    \begin{itemize}
        \item \texttt{INPUT A} prints \textit{?} and waits for user to type a
        numeric value followed by the \textit{RETURN} key. The numeric value
        will be assigned to the variable A.
        \item \texttt{INPUT A,B,C\$} prints \textit{?} and waits for user to
        type a numeric value followed by the \textit{RETURN} key. The numeric
        value will be assigned to the variable A. Then prints \textit{??} and
        waits for user to type a numeric value followed by the \textit{RETURN}
        key. The numeric value will be assigned to the variable B. Then prints
        \textit{??} and waits for user to type a string value followed by the
        \textit{RETURN} key. The string  value will be assigned to the variable
        C\$
        \item \texttt{INPUT "Name";na\$} prints the text \textit{Name?} and
        waits for user to type a string value followed by the \textit{RETURN}
        key
    \end{itemize}

    % ======================================================================
    \subsubsection{{INT (Integer)}}
    \label{msbasic:lang:int}
    % ======================================================================
    Returns the largest integer of a specified floating point number.

    For positive numbers, the result is obtained by discarding the decimals. For
    negative numbers, the result is obtained by rounding to the nearest integer
    greater than the floating point.

    \hspace{1.9cm}\textbf{INT(\textit{f})}

    \textbf{Parameters}:

    \hspace{1cm}\textbf{\textit{f}}: floating number.

    \textbf{Examples}:
    \begin{itemize}
        \item \texttt{PRINT INT(2.54)} prints 2
        \item \texttt{A=INT(-2.54)} A equals -3
    \end{itemize}

    % ======================================================================
    \subsubsection{{LEFT\$ (Leftmost characters of string)}}
    \label{msbasic:lang:left}
    % ======================================================================
    Returns \textit{n} leftmost characters of a string.

    \hspace{1.9cm}\textbf{LEFT\$(\textit{x\$},\textit{n})}

    \textbf{Parameters}:

    \hspace{1cm}\textbf{\textit{x\$}}: string.

    \hspace{1cm}\textbf{\textit{n}}: number of leftmost characters to return.

    \textbf{Examples}:
    \begin{itemize}
        \item \texttt{PRINT LEFT\$("DAVE",2)} prints DA
        \item \texttt{A\$=LEFT\$("DAVE",2)} A\$ equals DA
    \end{itemize}

    % ======================================================================
    \subsubsection{{LEN (Length of string)}}
    \label{msbasic:lang:len}
    % ======================================================================
    Returns the length of a string.

    \hspace{1.9cm}\textbf{LEN(\textit{x\$})}

    \textbf{Parameters}:

    \hspace{1cm}\textbf{\textit{x\$}}: string.

    \textbf{Examples}:
    \begin{itemize}
        \item \texttt{PRINT LEN("DAVE")} prints 4
        \item \texttt{A=LEN("DAVE")} A\$ equals 4
    \end{itemize}

    % ======================================================================
    \subsubsection{{LET (Assign value to variable)}}
    \label{msbasic:lang:let}
    % ======================================================================
    Assigns a value to a variable. It's optional and rarely used.

    \hspace{1.9cm}\textbf{LET \textit{var}=\textit{x}}

    \textbf{Parameters}:

    \hspace{1cm}\textbf{\textit{var}}: variable that will be assigned.

    \hspace{1cm}\textbf{\textit{x}}: value to assign to \textit{var}.

    \textbf{Examples}:
    \begin{itemize}
        \item \texttt{LET A=10} assign 10 to A
        \item \texttt{A=10} does the same, without the need of \textit{LET}
    \end{itemize}

    % ======================================================================
    \subsubsection{{LINES (Lines printed by LIST)}}
    \label{msbasic:lang:lines}
    % ======================================================================
    Defines the number of lines printed by a \hyperref[msbasic:lang:list]{LIST}
    command before pausing. Default is 20 lines.

    \hspace{1.9cm}\textbf{LINES \textit{n}}

    \textbf{Parameters}:

    \hspace{1cm}\textbf{\textit{n}}: number of lines to print by a
    \hyperref[msbasic:lang:list]{LIST} before pausing and waiting for a key to
    continue listing.

    \textbf{Examples}:
    \begin{itemize}
        \item \texttt{LINES 10} \hyperref[msbasic:lang:list]{LIST} will print 10
        lines and then pause
        \item \texttt{LINES 0} \hyperref[msbasic:lang:list]{LIST} won't print
        anything, and it will wait forever. To exit, press CTRL+C
    \end{itemize}

    % ======================================================================
    \subsubsection{{LIST (List BASIC program)}}
    \label{msbasic:lang:list}
    % ======================================================================
    List the contents of the BASIC program in memory, starting from the lowest
    line number.

    The list is terminated by either the end of the program or by pressing
    CTRL+C.

    The list pauses every number of \hyperref[msbasic:lang:lines]{LINES}, until a
    key is pressed.

    \hspace{1.9cm}\textbf{LIST [\textit{n}]}

    \textbf{Parameters}:

    \hspace{1cm}\textbf{\textit{n}}: optionally, line number from which the list
    will start. If line number \textit{n} doesn't exit, it will start from next
    closest line number.

    \textbf{Examples}:
    \begin{itemize}
        \item \texttt{LIST} list program, starting from lowest line
        \item \texttt{LIST 100} list program, starting from line 100
        \item \texttt{LIST 101} line 101 doesn't exist, list starts at next line
        (e.g. 110)
    \end{itemize}

    % ======================================================================
    \subsubsection{{LOAD (Load BASIC program from DISK)}}
    \label{msbasic:lang:load}
    % ======================================================================
    Loads a BASIC program from \textbf{DISK} into \textbf{MEMORY}.

    \hspace{1.9cm}\textbf{LOAD "\textit{f}}

    \textbf{Parameters}:

    \hspace{1cm}\textbf{\textit{f}}: the name of the file to be loaded.

    \textbf{Examples}:
    \begin{itemize}
        \item \texttt{LOAD "mandelbrot}
    \end{itemize}

    % ======================================================================
    \subsubsection{{LOG (Natural logarithm)}}
    \label{msbasic:lang:log}
    % ======================================================================
    Returns the natural logarithm of a number that's greater than zero.

    \hspace{1.9cm}\textbf{LOG(\textit{x})}

    \textbf{Parameters}:

    \hspace{1cm}\textbf{\textit{x}}: number (greater than 0).

    \textbf{Examples}:
    \begin{itemize}
        \item \texttt{PRINT LOG(4)} prints 1.38629
        \item \texttt{PRINT LOG(0)} prints \textit{?FC Error}
        \item \texttt{A=LOG(4)} A\$ equals 1.38629
    \end{itemize}

    % ======================================================================
    \subsubsection{{MID\$ (n characters of a string)}}
    \label{msbasic:lang:mid}
    % ======================================================================
    Returns \textit{n} characters of a string.

    \hspace{1.9cm}\textbf{MID\$(\textit{x\$},\textit{s}[,\textit{n}])}

    \textbf{Parameters}:

    \hspace{1cm}\textbf{\textit{x\$}}: string.

    \hspace{1cm}\textbf{\textit{s}}: character where to start.

    \hspace{1cm}\textbf{\textit{n}}: Optional. Number of characters to return.

    If number of characters (\textit{n}) is omitted, returns all characters from
    start (\textit{s}).

    \textbf{Examples}:
    \begin{itemize}
        \item \texttt{PRINT MID\$("dastaZ80",3)} prints staZ80
        \item \texttt{PRINT MID\$("dastaZ80",3,3)} prints sta
        \item \texttt{PRINT MID\$("dastaZ80",1,3)} prints das
    \end{itemize}

    % ======================================================================
    \subsubsection{{MONITOR (Exit MS BASIC)}}
    \label{msbasic:lang:monitor}
    % ======================================================================
    Quits MS BASIC and transfers the command to dzOS Command-Line Interface (CLI).

    \hspace{1.9cm}\textbf{MONITOR}

    \textbf{Parameters}: None

    % ======================================================================
    \subsubsection{{NEW (Delete current BASIC program)}}
    \label{msbasic:lang:new}
    % ======================================================================
    Deletes the current program and clears all variables.

    \hspace{1.9cm}\textbf{NEW}

    \textbf{Parameters}: None

    % ======================================================================
    \subsubsection{{ON...GOSUB (Branch to subroutine from list)}}
    \label{msbasic:lang:ongosub}
    % ======================================================================
    Branches to subroutine specified at line whose numer is the \textit{n}th in
    the list. And then, once a \hyperref[msbasic:lang:return]{RETURN} statement
    is found in the subroutine, continue execution from next statement.

    \hspace{1.9cm}\textbf{ON \textit{x} GOSUB \textit{n1},\textit{n2},...,
    \textit{nn}}

    \textbf{Parameters}:

    \hspace{1cm}\textbf{\textit{x}}: value that contains the value to be
    considered as \textit{n}th in the list. If \textit{x}=0 or greater than the
    number of elements in the list, execution continues at next statement. If
    \textit{x} is negative or greater than 255, an error occurs.

    \hspace{1cm}\textbf{\textit{n}}: Line number.

    \textbf{Examples}:
    \begin{itemize}
        \item \texttt{ON A GOSUB 100,200,300} if A=1, it will branch to line
        100. If A=2, it will branch to line 200. If A=3, it will branch to line
        300.
    \end{itemize}

    % ======================================================================
    \subsubsection{{ON...GOTO (Branch to line from list)}}
    \label{msbasic:lang:ongoto}
    % ======================================================================
    Branches to line whose number is the \textit{n}th in the list.

    \hspace{1.9cm}\textbf{ON \textit{x} GOTO \textit{n1},\textit{n2},...,
    \textit{nn}}

    \textbf{Parameters}:

    \hspace{1cm}\textbf{\textit{x}}: value that contains the value to be
    considered as \textit{n}th in the list. If \textit{x}=0 or greater than the
    number of elements in the list, execution continues at next statement. If
    \textit{x} is negative or greater than 255, an error occurs.

    \hspace{1cm}\textbf{\textit{n}}: Line number.

    \textbf{Examples}:
    \begin{itemize}
        \item \texttt{ON A GOTO 100,200,300} if A=1, it will branch to line 100.
        If A=2, it will branch to line 200. If A=3, it will branch to line 300
    \end{itemize}

    % ======================================================================
    \subsubsection{{OUT (Send to Port)}}
    \label{msbasic:lang:out}
    % ======================================================================
    Sends a byte to a specified port.

    \hspace{1.9cm}\textbf{OUT \textit{p},\textit{x}}

    \textbf{Parameters}:

    \hspace{1cm}\textbf{\textit{p}}: port number. Must be a value between 0 and
    255.

    \hspace{1cm}\textbf{\textit{x}}: value to send. Must be a value between 0
    and 255.

    \textbf{Examples}:
    \begin{itemize}
        \item \texttt{PORT 80,1} sends the value 1 to port 80
    \end{itemize}

    % ======================================================================
    \subsubsection{{PEEK (Byte from RAM)}}
    \label{msbasic:lang:peek}
    % ======================================================================
    Reads a byte from a \textit{RAM} address.

    \hspace{1.9cm}\textbf{PEEK \textit{mem}}

    \textbf{Parameters}:

    \hspace{1cm}\textbf{\textit{mem}}: memory address where the byte will be
    read from.

    \textbf{Examples}:
    \begin{itemize}
        \item \texttt{A=PEEK(\&H4420)} the byte found at \textit{RAM} address
        \textit{0x4420} is assigned to the variable A
    \end{itemize}

    % ======================================================================
    \subsubsection{{POKE (Byte to RAM)}}
    \label{msbasic:lang:poke}
    % ======================================================================
    Stores a value in a specific \textbf{RAM} address.

    \hspace{1.9cm}\textbf{POKE \textit{mem},\textit{x}}

    \textbf{Parameters}:

    \hspace{1cm}\textbf{\textit{mem}}: \textbf{RAM} address where \textit{x}
    will be stored.

    \hspace{1cm}\textbf{\textit{x}}: value to store (0..255).

    \textbf{Examples}:
    \begin{itemize}
        \item \texttt{POKE \&h4420,62}, stores 62 in \textbf{RAM} address
        \texttt{0x4420}
        \item \texttt{POKE 17440,62}, stores 62 in \textbf{RAM} address 17440
    \end{itemize}

    % ======================================================================
    \subsubsection{{PRINT (Print to Terminal)}}
    \label{msbasic:lang:print}
    % ======================================================================
    Prints expression to Terminal. Expression can be a variable or text. A
    Carriage Return (CR) is automatically added to the end of the expression.
    The CR can be omitted if after the expression a semicolon (;) is added.

    \hspace{1.9cm}\textbf{PRINT \textit{x}}

    \textbf{Parameters}:

    \hspace{1cm}\textbf{\textit{x}}: expression.

    Multiple expressions can be printed one after the another, when separated
    with semicolon (;)

    \textbf{Examples}:
    \begin{itemize}
        \item \texttt{PRINT A} prints the value stored in the variable A
        \item \texttt{PRINT "dastaZ80"} prints the text \textit{dastaZ80}
        \item \texttt{PRINT "dasta";"Z80"} prints the text \textit{dastaZ80}
        \item \texttt{PRINT "dasta";A} prints the text \textit{dasta} followed
        by the value stored in the variable A
    \end{itemize}

    % ======================================================================
    \subsubsection{{READ (Assign value from DATA)}}
    \label{msbasic:lang:read}
    % ======================================================================
    Assigns values in \hyperref[msbasic:lang:data]{DATA} statements to variable.
    If multiple READs are issued, the values are assigned in sequence from the
    \hyperref[msbasic:lang:data]{DATA} list.

    \hspace{1.9cm}\textbf{READ \textit{v}}

    \textbf{Parameters}:

    \hspace{1cm}\textbf{\textit{v}}: a variable or list of variables to which
    assign the values from \hyperref[msbasic:lang:data]{DATA}.

    \textbf{Examples}:
    \begin{itemize}
        \item \texttt{10 DATA 74,"dasta"}
        \item \texttt{20 READ A,B\$} will assign 74 to A and "dasta" to B\$
    \end{itemize}

    % ======================================================================
    \subsubsection{{REM (Remark)}}
    \label{msbasic:lang:rem}
    % ======================================================================
    Allows the insertion of remarks, not executed but may be branched into.

    \hspace{1.9cm}\textbf{REM \textit{r}}

    \textbf{Parameters}:

    \hspace{1cm}\textbf{\textit{r}}: text that will be used as remark.

    \textbf{Examples}:
    \begin{itemize}
        \item \texttt{10 REM "dastaZ80"}
    \end{itemize}

    % ======================================================================
    \subsubsection{{RESET (Re-start dastaZ80)}}
    \label{msbasic:lang:reset}
    % ======================================================================
    Reset the computer. It has the same effect as pressing the reset button at
    the side of the computer.

    \hspace{1.9cm}\textbf{RESET}

    \textbf{Parameters}: None

    % ======================================================================
    \subsubsection{{RESTORE (DATA from start)}}
    \label{msbasic:lang:restore}
    % ======================================================================
    Allows \hyperref[msbasic:lang:data]{DATA} values to be re-read, by
    positioning the pointer to the first element of the first
    \hyperref[msbasic:lang:data]{DATA} statement.

    \hspace{1.9cm}\textbf{RESTORE}

    \textbf{Parameters}: None

    \textbf{Examples}:
    \begin{itemize}
        \item \texttt{10 DATA 74,"dasta"}
        \item \texttt{20 READ A} will assign 74 to A, and point to the second
        element
        \item \texttt{30 RESTORE} will point to the first element again
        \item \texttt{20 READ B} will assign 74 to B
    \end{itemize}

    % ======================================================================
    \subsubsection{{RETURN (End of subroutine)}}
    \label{msbasic:lang:return}
    % ======================================================================
    Terminates a subroutine and branches execution to most recent
    \hyperref[msbasic:lang:gosub]{GOSUB}.

    \hspace{1.9cm}\textbf{RESTORE}

    \textbf{Parameters}: None

    % ======================================================================
    \subsubsection{{RIGHT\$ (Rightmost characters of a string)}}
    \label{msbasic:lang:right}
    % ======================================================================
    Returns \textit{n} rightmost characters of a string.

    \hspace{1.9cm}\textbf{RIGHT\$(\textit{x\$},\textit{n})}

    \textbf{Parameters}:

    \hspace{1cm}\textbf{\textit{x\$}}: string.

    \hspace{1cm}\textbf{\textit{n}}: number of rightmost characters to return.

    \textbf{Examples}:
    \begin{itemize}
        \item \texttt{PRINT RIGHT\$("DAVE",2)} prints VE
        \item \texttt{A\$=RIGHT\$("DAVE",2)} A\$ equals VE
    \end{itemize}

    % ======================================================================
    \subsubsection{{RND (Random number)}}
    \label{msbasic:lang:rnd}
    % ======================================================================
    Returns a random number between 0 and 1.

    \hspace{1.9cm}\textbf{RND(\textit{x})}

    \textbf{Parameters}:

    If \textit{x} is negative, the random number generator will be re-seed
    before returning the number. If \textit{x} is zero, the last returned number
    will be returned again. If \textit{x} is postive, the next randdom in the
    swquence will be returned.

    Random number are generated in a sequence, same negative value used in
    \textit{x} will generate the same sequence.

    \textbf{Examples}:
    \begin{itemize}
        \item \texttt{PRINT RND(-1)} re-seed and prints random
        \item \texttt{PRINT RND(0)} prints previous generated random
        \item \texttt{A=RND(1)} A\$ equals next random in the sequence
    \end{itemize}

    % ======================================================================
    \subsubsection{{RUN (Execute BASIC program)}}
    \label{msbasic:lang:run}
    % ======================================================================
    Starts execution of the current program.

    \hspace{1.9cm}\textbf{RUN [\textit{l}]}

    \textbf{Parameters}:

    \hspace{1cm}\textbf{\textit{l}}: optionally, line number to start execution
    from.

    \textbf{Examples}:
    \begin{itemize}
        \item \texttt{RUN} starts execution of program from lowest line number
        \item \texttt{RUN 100} starts execution of program from line 100
    \end{itemize}

    % ======================================================================
    \subsubsection{{SAVE (Save BASIC program to DISK)}}
    \label{msbasic:lang:save}
    % ======================================================================
    Saves current BASIC program from \textbf{MEMORY} into current
    \textbf{DISK}.

    \hspace{1.9cm}\textbf{SAVE "\textit{f}}

    \textbf{Parameters}:

    \hspace{1cm}\textbf{\textit{f}}: the name of the file to be
    saved.

    \textbf{Examples}:
    \begin{itemize}
        \item \texttt{SAVE "mandelbrot}
    \end{itemize}

    \textbf{Hint}: Although there isn't an MS BASIC command to change the
    current \textbf{DISK}, it can easily be changed manually. DZOS
    \textbf{DISK} operations are performed to whatever \textbf{DISK} number
    is stored in an area of \textbf{MEMORY} called \textit{SYSVARS}
    (See \textit{dastaZ80 Manual - Programmer’s Reference Guide} for more
    details). The \textbf{DISK} number is stored at address \texttt{0x4176},
    therefore by changing the value stored at this address we are
    effectively changing the current \textbf{DISK}. To change it simply use
    \texttt{POKE \&h4176,disknum} where \textit{disknum} is a valid
    \textbf{DISK} number.

    % ======================================================================
    \subsubsection{{SCREEN (Change VDP screen mode)}}
    \label{msbasic:lang:screen}
    % ======================================================================
    Changes the \textbf{Low Resolution Display} screen mode.

    \hspace{1.9cm}\textbf{SCREEN \textit{m}}

    \textbf{Parameters}:

    \hspace{1cm}\textbf{\textit{m}}: one of the valid
    \hyperref[sec:vdpscrmodes]{\textbf{Low Resolution} Screen Modes}:

    \begin{itemize}
        \item 0 = Text Mode
        \item 1 = Graphics I Mode
        \item 2 = Graphics II Mode
        \item 3 = Multicolour Mode
        \item 4 = Graphics II Mode Bitmapped
    \end{itemize}

    \textbf{Examples}:
    \begin{itemize}
        \item \texttt{SCREEN 2}
    \end{itemize}

    % ======================================================================
    \subsubsection{{SGN (Sign)}}
    \label{msbasic:lang:sgn}
    % ======================================================================
    Returns the sign of a specified number.

    If number is 0, return is 0. If number is positive, return is 1. If number
    is negative, return is -1.

    \hspace{1.9cm}\textbf{SGN(\textit{x})}

    \textbf{Parameters}:

    \hspace{1cm}\textbf{\textit{x}}: number.

    \textbf{Examples}:
    \begin{itemize}
        \item \texttt{PRINT SGN(9)} print 1
        \item \texttt{PRINT SGN(0)} prints 0
        \item \texttt{A=SGN(-3)} A\$ equals -1
    \end{itemize}

    % ======================================================================
    \subsubsection{{SIN (Sine)}}
    \label{msbasic:lang:sin}
    % ======================================================================
    Returns sine in radians (range -\textpi/2 to \textpi/2) of a number.

    \hspace{1.9cm}\textbf{SIN(\textit{x})}

    \textbf{Parameters}:

    \hspace{1cm}\textbf{\textit{x}}: number.

    \textbf{Examples}:
    \begin{itemize}
        \item \texttt{PRINT SIN(4)} prints -.756802
        \item \texttt{A=SIN(20)} A equals -.756802
    \end{itemize}

    % ======================================================================
    \subsubsection{{SPC (Print n spaces)}}
    \label{msbasic:lang:spc}
    % ======================================================================
    Prints \textit{n} number of blanks. It can only be used in conjuntion with
    \hyperref[msbasic:lang:print]{PRINT}.

    \hspace{1.9cm}\textbf{SPC(\textit{n})}

    \textbf{Parameters}:

    \hspace{1cm}\textbf{\textit{n}}: number of blanks. MUST be a positive number
    less than 256.

    \textbf{Examples}:
    \begin{itemize}
        \item \texttt{PRINT SPC(4)};"dastaZ80" prints 4 blank spaces and then
        prints dastaZ80
    \end{itemize}

    % ======================================================================
    \subsubsection{{SPOKE (Byte to PSG)}}
    \label{msbasic:lang:spoke}
    % ======================================================================
    Writes a value in a specific \textbf{PSG} register.

    \hspace{1.9cm}\textbf{SPOKE \textit{r},\textit{x}}

    \textbf{Parameters}:

    \hspace{1cm}\textbf{\textit{r}}: \textbf{PSG} register number (0-13).

    \hspace{1cm}\textbf{\textit{x}}: value to set (0..255).

    \textbf{Examples}:
    \begin{itemize}
        \item \texttt{SPOKE 7,62}
    \end{itemize}


    % ======================================================================
    \subsubsection{{SQR (Square root)}}
    \label{msbasic:lang:sqr}
    % ======================================================================
    Returns square root of a number.

    \hspace{1.9cm}\textbf{SQR(\textit{x})}

    \textbf{Parameters}:

    \hspace{1cm}\textbf{\textit{x}}: number, greater than zero.

    \textbf{Examples}:
    \begin{itemize}
        \item \texttt{PRINT SQR(4)} prints 2
        \item \texttt{A=SQR(4)} A equals 2
    \end{itemize}

    % ======================================================================
    \subsubsection{{STOP (Stop program execution)}}
    \label{msbasic:lang:stop}
    % ======================================================================
    Stops program execution. BASIC enters command level and prints
    \textit{Break in nnnn}, where \textit{nnnn} is the line where the STOP was
    found.

    Execution can be continued with \hyperref[msbasic:lang:cont]{CONT}.

    \hspace{1.9cm}\textbf{STOP)}

    \textbf{Parameters}: None

    \textbf{Examples}:
    \begin{itemize}
        \item \texttt{10 PRINT "Hello"} prints Hello
        \item \texttt{20 STOP} programs stops, BASIC prints \textit{Break in 20}
        \item \texttt{30 PRINT "Continued"} this is not printed
        \item \texttt{CONT} execution continues from 30, hence prints
        \textit{Continued}
    \end{itemize}

    % ======================================================================
    \subsubsection{{STR\$ (Number to String)}}
    \label{msbasic:lang:str}
    % ======================================================================
    Returns string representation of value of a number.

    \hspace{1.9cm}\textbf{STR\$(\textit{x})}

    \textbf{Parameters}:

    \hspace{1cm}\textbf{\textit{x}}: number.

    \textbf{Examples}:
    \begin{itemize}
        \item \texttt{A\$=STR(12)} A\$ equal string 12
        \item \texttt{A\$=STR(-2)} A\$ equals string -2
    \end{itemize}

    % ======================================================================
    \subsubsection{{TAB (Move cursor n columns)}}
    \label{msbasic:lang:tab}
    % ======================================================================
    Moves the cursor to the \textit{n}th column on the terminal. It can only be
    used in conjuntion with \hyperref[msbasic:lang:print]{PRINT}.

    \hspace{1.9cm}\textbf{TAB\$(\textit{n})}

    \textbf{Parameters}:

    \hspace{1cm}\textbf{\textit{n}}: number. UST be a positive number less than
    256.

    \textbf{Examples}:
    \begin{itemize}
        \item \texttt{PRINT TAB(4)};"dastaZ80" moves the cursor to column 4 and
        then prints dastaZ80
    \end{itemize}

    % ======================================================================
    \subsubsection{{TAN (Tangent)}}
    \label{msbasic:lang:tan}
    % ======================================================================
    Returns tangent in radians (range -\textpi/2 to \textpi/2) of a number.

    \hspace{1.9cm}\textbf{TAN(\textit{x})}

    \textbf{Parameters}:

    \hspace{1cm}\textbf{\textit{x}}: number.

    \textbf{Examples}:
    \begin{itemize}
        \item \texttt{PRINT TAN(4)} prints 1.15782
        \item \texttt{A=TAN(4)} A equals 1.15782
    \end{itemize}

    % ======================================================================
    \subsubsection{{USR (Call user-defined subroutine)}}
    \label{msbasic:lang:usr}
    % ======================================================================
    Calls the user's machine language subroutine with an argument.

    After boot, USR points to an \textit{Illegal function call Error} and
    therefore it must first be configured in order to be used. To configure it,
    we need to set the address to which USR will jump. The two bytes of the
    address are stored at \texttt{0x473F} and \texttt{0x4740}

    \hspace{1.9cm}\textbf{USR(\textit{x})}

    \textbf{Parameters}:

    \hspace{1cm}\textbf{\textit{x}}: argument. Mandatory even if not used by the
    called subroutine.

    \textbf{Examples}:
    \begin{itemize}
        \item \texttt{USR(0)} jumps to whatever address is stored at
        \texttt{0x473F}
    \end{itemize}

    % ======================================================================
    \subsubsection{{VAL (Numeric value of a string)}}
    \label{msbasic:lang:val}
    % ======================================================================
    Returns numerical value of a string.

    \hspace{1.9cm}\textbf{VAL(\textit{x})}

    \textbf{Parameters}:

    \hspace{1cm}\textbf{\textit{x}}: number.

    If first character if \textit{x} is not a number, or +, -, \$, \%, returns zero.

    If first character is \$, the number is interpreted as hexadecimal.

    If first character is \%, the number is interpreted as binary.

    \textbf{Examples}:
    \begin{itemize}
        \item \texttt{A=VAL("45")} A equals 45
        \item \texttt{A=VAL("+45")} A equals 45
        \item \texttt{A=VAL("-45")} A equals -45
        \item \texttt{A=VAL("\$FF")} A equals 255
        \item \texttt{A=VAL("\%1010")} A equals 10
        \item 
    \end{itemize}

    % ======================================================================
    \subsubsection{{VPEEK (Byte from VRAM)}}
    \label{msbasic:lang:vpeek}
    % ======================================================================
    Gets the value at a specific \textbf{VRAM} address.

    \hspace{1.9cm}\textbf{VPEEK \textit{x}}

    \textbf{Parameters}:

    \hspace{1cm}\textbf{\textit{x}}: \textbf{VRAM} address.

    \textbf{Examples}:
    \begin{itemize}
        \item \texttt{PRINT VPEEK(6144)}
        \item \texttt{A=VPEEK(6144)}
    \end{itemize}

    % ======================================================================
    \subsubsection{{VPOKE (Byte to VRAM)}}
    \label{msbasic:lang:vpoke}
    % ======================================================================
    Writes a value at a specific \textbf{VRAM} address.

    \hspace{1.9cm}\textbf{VPOKE \textit{x},\textit{v}}

    \textbf{Parameters}:

    \hspace{1cm}\textbf{\textit{x}}: \textbf{VRAM} address.

    \hspace{1cm}\textbf{\textit{v}}: value to set (0..255).

    \textbf{Examples}:
    \begin{itemize}
        \item \texttt{VPOKE 6144,171}
    \end{itemize}

    % ======================================================================
    \subsubsection{{WAIT (Waits for value in Port)}}
    \label{msbasic:lang:wait}
    % ======================================================================
    Execution waits until a value read for a Port and XOR'd with a specified
    value is non-zero. Optionally, the value read from the port can be also
    AND'ed with another value.

    \hspace{1.9cm}\textbf{WAIT \textit{p},\textit{o}[,\textit{a}]}

    \textbf{Parameters}:

    \hspace{1cm}\textbf{\textit{p}}: port number where to read value from.

    \hspace{1cm}\textbf{\textit{o}}: this value (0..255) will be XOR'd with the
    value read from \textit{p}.

    \hspace{1cm}\textbf{\textit{a}}: optionally, this value (0..255) will be
    AND'ed with the result of \textit{o} XOR'd with the value read from
    \textit{p}. If not specified, it defaults to zero.

    \textbf{Examples}:
    \begin{itemize}
        \item \texttt{WAIT 20,6} execution stops until either bit 1 or bit 2 of
        port 20 are equal to 1
        \item \texttt{WAIT 10,255,7} execution stops until any of the most
        significant 5 bits of port 10 are one or any of the least significant 3
        bits are zero
    \end{itemize}
