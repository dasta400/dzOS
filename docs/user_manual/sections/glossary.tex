% ==========================================================================
\section{Glossary}
% ==========================================================================

\begin{itemize}
    \item \textbf{Block Allocation Table (BAT)}: is a data structure used to
        track disk blocks on a \textbf{DISK}. The Table is used as index of
        files, and each entry in the Table constains details about each file
        (filename, attributes, size, etc.) and details on how the OS can access
        the data (e.g. in which Sector the file is stored). Technically, the BAT
        is not allocating Blocks but Sectors, so it should have been named
        Sector Allocation Table instead.
    \item \textbf{BIOS}: The Basic Input/Output System is the firmware used to
        provide hardware related subroutines to the operating system. It is
        stored in an EEPROM that cannot be modified unless extracting the chip,
        but during the boot sequence the entire OS (BIOS, Kernel and CLI) are
        copied into RAM.
    \item \textbf{CLI}: The Command-Line Interface is the part of the operating
        systems that is in charge of reading and interpreting input entered by
        the user and outputting information back to the user.
    \item \textbf{Composite video}: Composite video (also knows by CVBS) is an
        analog video format that combines, on one wire, the video information
        required to recreate a colour picture, line and frame synchronisation
        pulses. A yellow RCA connector is typically used for this type of signal.
    \item \textbf{CPU}: A Central Processing Unit is the chip that executes
        instructions and I/O operations. In the case of the dastaZ80, the CPU is
        a Zilog Z80 running at 7.3728 Mhz.
    \item \textbf{Disk Image file}: A disk image file contains a snapshot of a
        bit-by-bit copy of a storage device's structure. In the case of dastaZ80,
        this structure is defined by the characteristics of the DZFS (dastaZ80
        File System). Each image file can be understood as a separate hard disk
        connected to the computer, but in the form of a file instead of being a
        physical hard disk drive.
    \item \textbf{DZFS}: dastaZ80 File System is a file system of my own design,
        for storage devices, aimed at simplicity. It allows Disk Image files of
        a maximum of 33 MB for a maximum of 1024 files in each image. Each file
        can be a maximum of 32 KB in size.
    \item \textbf{EEPROM}: Electrically Erasable Programmable Read-Only Memory
        is a type of non-volatile memory that can be erased and re-programmed.
    \item \textbf{File System}: A file system manages access to the data and the
        metadata of the files, and the available space of the device, dividing
        the storage area into units of storage and keeping a map of every
        storage unit of the device.
    \item \textbf{General-Purpose Input/Output (GPIO)}: is a set of pins which
        carries \textbf{CPU} (among others) signals, and allows external devices
        to be plugged into the GPIO connector an be managed by the computer.
        Typical devices connected to a GPIO are; modems, RAM expansions,
        Real-Time Clocks (RTC), hard drives.
    \item \textbf{Kernel}: The Kernel is a layer of the operating system that
        it is in between the software and the BIOS. Thus the Kernel is agnostic
        to the hardware, and is only responsible for calling the needed
        subroutines.
    \item \textbf{Logical Block Addressing (LBA)}: is a scheme for specifying
        the location of blocks of data stored on computer storage devices. In
        DZOS, it is based in blocks of sectors, where each block is 64 sectors
        a each sector is 512 bytes.
    \item \textbf{LED}: A Light-Emitting Diode is a semiconductor device that
        emits light when current flows through it. It is commonly used in
        computers, as light indicator, due to their high durability and low
        power consumption.
        \item \textbf{NTSC}: National Television System Committee, is the
        standard for analog television used mainly in USA, Japan and some parts
        of South America.
    \item \textbf{PAL}: Phase Alternating Line (PAL) is the standard for analog
        television used mainly in European countries, some African countries,
        some Asian countries, and some Soth American countries.
    \item \textbf{RAM}: Read-Access Memory is a type of volatile memory, that is
        generally used for storing temporary data, like for example programs
        loaded from disk or input entered by the user.
    \item \textbf{RCA connector}: Is a type of electrical connector commonly
        used to carry audio and video signals. It is also called
        \textit{phono connector}. The standard is to have yellow connector for
        video, and white and red for audio.
    \item \textbf{ROM}: Read-Only Memory is a type of non-volatile memory that 
        cannot be electronically modified after the manufacture process.
    \item \textbf{Sector}: A virtual group of 512 bytes stored in a
        \textbf{DISK}. 
    \item \textbf{Superblock}: The first 512 bytes on a \textbf{DISK} contain
        fundamental information about the \textbf{DISK} geometry, and is used by
        the OS to know how to access every other information on the
        \textbf{DISK}. On IBM PC-compatibles, this is known as the
        \textit{Master Boot Record} (MBR). I decided to call it
        \textit{Superblock}, as it is an orphan Sector that doesn't belong to
        any Block and it's physically store above any other Block.
    \item \textbf{VGA}: The Video Graphics Array is a video display controller
        that became the standard for video output for computers around 1990s,
        and remained until the introduction of other standards around 2000s.
\end{itemize}