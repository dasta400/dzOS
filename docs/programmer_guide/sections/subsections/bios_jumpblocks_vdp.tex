    % ==========================================================================
    \subsection{VDP Routines}
    % ==========================================================================

        % ==========================================================================
        \subsubsection{F\_BIOS\_VDP\_SET\_ADDR\_WR}
        \label{func:fbiosvdpsetaddrwr}
        % ==========================================================================
        \begin{tabular}{l p{15cm}}
            \hline\textbf{Action}
            & Set a \textbf{VRAM} address for writting. \\
            \hline\textbf{Entry} & \texttt{HL} = address to be set\\
            \hline\textbf{Exit} & None\\
            \hline\textbf{Destroys} & \texttt{C}, \texttt{H} \\
            \hline\textbf{Calls} & None\\
            \hline
        \end{tabular}

        % ==========================================================================
        \subsubsection{F\_BIOS\_VDP\_SET\_ADDR\_RD}
        \label{func:fbiosvdpsetaddrrd}
        % ==========================================================================
        \begin{tabular}{l p{15cm}}
            \hline\textbf{Action}
            & Set a \textbf{VRAM} address for reading. \\
            \hline\textbf{Entry} & \texttt{HL} = address to be read\\
            \hline\textbf{Exit} & None\\
            \hline\textbf{Destroys} & \texttt{A}, \texttt{C} \\
            \hline\textbf{Calls} & None\\
            \hline
        \end{tabular}

        % ==========================================================================
        \subsubsection{F\_BIOS\_VDP\_SET\_REGISTER}
        \label{func:fbiosvdpsetregister}
        % ==========================================================================
        \begin{tabular}{l p{15cm}}
            \hline\textbf{Action}
            & Set a value to a \textbf{VDP} register. \\
            \hline\multirow[t]{2}{4em}{\textbf{Entry}}
            & \texttt{A} = register number\\
            & \texttt{B} = value to set\\
            \hline\textbf{Exit} & None\\
            \hline\textbf{Destroys} & \texttt{C} \\
            \hline\textbf{Calls} & None\\
            \hline
        \end{tabular}

        % ==========================================================================
        \subsubsection{F\_BIOS\_VDP\_EI}
        \label{func:fbiosvdpei}
        % ==========================================================================
        \begin{tabular}{l p{15cm}}
            \hline\multirow[t]{2}{4em}{\textbf{Action}}& Enable \textbf{VDP}
            Interrupts.\\
            & This is independent of the value (bit 5) in the \textbf{VDP}
            \textit{Register 1}. What this does is that the NMI subroutine reads
            the \textbf{VDP} \textit{Status Register} again in each run, and
            therefore it does allow more interrupts to happen.\\
            \hline\textbf{Entry} & None\\
            \hline\textbf{Exit} & None\\
            \hline\textbf{Destroys} & \texttt{A}\\
            \hline\textbf{Calls} & \hyperref[func:fbiosvdpreadstatreg]
            {F\_BIOS\_VDP\_READ\_STATREG}\\
            \hline
        \end{tabular}

        % ==========================================================================
        \subsubsection{F\_BIOS\_VDP\_DI}
        \label{func:fbiosvdpdi}
        % ==========================================================================
        \begin{tabular}{l p{15cm}}
            \hline\multirow[t]{3}{4em}{\textbf{Action}}& Disable \textbf{VDP}
            Interrupts.\\
            & This is independent of the value (bit 5) in the \textbf{VDP}
            \textit{Register 1}. What this does is that the NMI subroutine does
            not read the \textbf{VDP} \textit{Status Register} anymore, and
            therefore does not allow more interrupts to happen.\\
            & \textbf{IMPORTANT}: Disabling \textbf{VDP} Interrupts will stop
            the \hyperref[subsec:jiffy_counter]{Jiffy Counter}.\\
            \hline\textbf{Entry} & None\\
            \hline\textbf{Exit} & None\\
            \hline\textbf{Destroys} & \texttt{A}\\
            \hline\textbf{Calls} & None\\
            \hline
        \end{tabular}

        % ==========================================================================
        \subsubsection{F\_BIOS\_VDP\_READ\_STATREG}
        \label{func:fbiosvdpreadstatreg}
        % ==========================================================================
        \begin{tabular}{l p{15cm}}
            \hline\multirow[t]{2}{4em}{\textbf{Action}}& Read the read-only
            \textbf{VDP} \textit{Status Register}.\\
            & \textbf{IMPORTANT}: Reading the \textbf{VDP} \textit{Status
            Register} clears (acknowledges) the \textbf{VDP} Interrupt. This is
            already done by the BIOS' NMI subroutine, so this function MUST not
            be used, unless NMI subroutines have been disabled with
            \hyperref[func:fbiosvdpdi]{F\_BIOS\_VDP\_DI}\\
            \hline\textbf{Entry} & None\\
            \hline\textbf{Exit} & \texttt{A} = Status Register byte.\\
            \hline\textbf{Destroys} & \texttt{A}, \texttt{C}\\
            \hline\textbf{Calls} & None\\
            \hline
        \end{tabular}

        % ==========================================================================
        \subsubsection{F\_BIOS\_VDP\_VRAM\_CLEAR}
        \label{func:fbiosvdpvramclear}
        % ==========================================================================
        \begin{tabular}{l p{15cm}}
            \hline\textbf{Action}
            & Set all cells of the \textbf{VRAM} (\texttt{0x0000}-
            \texttt{0x3FFF}) to zero. \\
            \hline\textbf{Entry} & None\\
            \hline\textbf{Exit} & None\\
            \hline\textbf{Destroys} & \texttt{A}, \texttt{BC}, \texttt{D},
            \texttt{HL} \\
            \hline\textbf{Calls} & \hyperref[func:fbiosvdpsetaddrwr]
            {F\_BIOS\_VDP\_SET\_ADDR\_WR}\\
            \hline
        \end{tabular}

        % ==========================================================================
        \subsubsection{F\_BIOS\_VDP\_VRAM\_TEST}
        \label{func:fbiosvdpvramtest}
        % ==========================================================================
        \begin{tabular}{l p{15cm}}
            \hline\textbf{Action}
            & Set a value to each \textbf{VRAM} cell and then reads it back. If
            the value is not the same, something went wrong. \\
            \hline\textbf{Entry} & None\\
            \hline\textbf{Exit} & \texttt{C Flag} set if an error ocurred.\\
            \hline\textbf{Destroys} & \texttt{A}, \texttt{BC}, \texttt{D},
            \texttt{HL} \\
            \hline\multirow[t]{2}{4em}{\textbf{Calls}}
            & \hyperref[func:fbiosvdpsetaddrwr]{F\_BIOS\_VDP\_SET\_ADDR\_WR}\\
            & \hyperref[func:fbiosvdpsetaddrrd]{F\_BIOS\_VDP\_SET\_ADDR\_RD}\\
            \hline
        \end{tabular}

        % ==========================================================================
        \subsubsection{F\_BIOS\_VDP\_SET\_MODE\_TXT}
        \label{func:fbiosvdpsetmodetxt}
        % ==========================================================================
        \begin{tabular}{l p{15cm}}
            \hline\textbf{Action}
            & Set \textbf{VDP} to \textit{Text Mode} display.\\
            \hline\multirow[t]{2}{4em}{\textbf{Entry}}
            & \texttt{B} = Sprite size (0=8×8, 1=16×16)\\
            & \texttt{C} = Sprite magnification (0=no magnification,
                1=magnification)\\
            \hline\textbf{Exit} & None\\
            \hline\textbf{Destroys} & \texttt{A}, \texttt{BC}, \texttt{D},
            \texttt{HL} \\
            \hline\multirow[t]{2}{4em}{\textbf{Calls}}
            & \hyperref[func:fbiosvdpsetaddrwr]{F\_BIOS\_VDP\_SET\_ADDR\_WR}\\
            & \hyperref[func:fbiosvdpsetregister]{F\_BIOS\_VDP\_SET\_REGISTER}\\
            \hline
        \end{tabular}

        % ==========================================================================
        \subsubsection{F\_BIOS\_VDP\_SET\_MODE\_G1}
        \label{func:fbiosvdpsetmodeg1}
        % ==========================================================================
        \begin{tabular}{l p{15cm}}
            \hline\textbf{Action}
            & Set \textbf{VDP} to \textit{Graphics I Mode} display.\\
            \hline\multirow[t]{2}{4em}{\textbf{Entry}}
            & \texttt{B} = Sprite size (0=8×8, 1=16×16)\\
            & \texttt{C} = Sprite magnification (0=no magnification,
                1=magnification)\\
            \hline\textbf{Exit} & None\\
            \hline\textbf{Destroys} & \texttt{A}, \texttt{BC}, \texttt{D},
            \texttt{HL} \\
            \hline\multirow[t]{2}{4em}{\textbf{Calls}}
            & \hyperref[func:fbiosvdpsetaddrwr]{F\_BIOS\_VDP\_SET\_ADDR\_WR}\\
            & \hyperref[func:fbiosvdpsetregister]{F\_BIOS\_VDP\_SET\_REGISTER}\\
            \hline
        \end{tabular}

        % ==========================================================================
        \subsubsection{F\_BIOS\_VDP\_SET\_MODE\_G2}
        \label{func:fbiosvdpsetmodeg2}
        % ==========================================================================
        \begin{tabular}{l p{15cm}}
            \hline\textbf{Action}
            & Set \textbf{VDP} to \textit{Graphics II Mode} display.\\
            \hline\multirow[t]{2}{4em}{\textbf{Entry}}
            & \texttt{B} = Sprite size (0=8×8, 1=16×16)\\
            & \texttt{C} = Sprite magnification (0=no magnification,
                1=magnification)\\
            \hline\textbf{Exit} & None\\
            \hline\textbf{Destroys} & \texttt{A}, \texttt{BC}, \texttt{D},
            \texttt{HL} \\
            \hline\multirow[t]{2}{4em}{\textbf{Calls}}
            & \hyperref[func:fbiosvdpsetaddrwr]{F\_BIOS\_VDP\_SET\_ADDR\_WR}\\
            & \hyperref[func:fbiosvdpsetregister]{F\_BIOS\_VDP\_SET\_REGISTER}\\
            \hline
        \end{tabular}

        % ==========================================================================
        \subsubsection{F\_BIOS\_VDP\_SET\_MODE\_G2BM}
        \label{func:fbiosvdpsetmodeg2bm}
        % ==========================================================================
        \begin{tabular}{l p{15cm}}
            \hline\textbf{Action}
            & Set \textbf{VDP} to \textit{Graphics II Bit-mapped Mode} display.\\
            \hline\multirow[t]{2}{4em}{\textbf{Entry}}
            & \texttt{B} = Sprite size (0=8×8, 1=16×16)\\
            & \texttt{C} = Sprite magnification (0=no magnification,
                1=magnification)\\
            \hline\textbf{Exit} & None\\
            \hline\textbf{Destroys} & \texttt{A}, \texttt{BC}, \texttt{D},
            \texttt{HL} \\
            \hline\multirow[t]{2}{4em}{\textbf{Calls}}
            & \hyperref[func:fbiosvdpsetaddrwr]{F\_BIOS\_VDP\_SET\_ADDR\_WR}\\
            & \hyperref[func:fbiosvdpsetregister]{F\_BIOS\_VDP\_SET\_REGISTER}\\
            \hline
        \end{tabular}

        % ==========================================================================
        \subsubsection{F\_BIOS\_VDP\_SET\_MODE\_MULTICLR}
        \label{func:fbiosvdpsetmodemulticlr}
        % ==========================================================================
        \begin{tabular}{l p{15cm}}
            \hline\textbf{Action}
            & Set \textbf{VDP} to \textit{Multicolour Mode} display.\\
            \hline\multirow[t]{2}{4em}{\textbf{Entry}}
            & \texttt{B} = Sprite size (0=8×8, 1=16×16)\\
            & \texttt{C} = Sprite magnification (0=no magnification,
                1=magnification)\\
            \hline\textbf{Exit} & None\\
            \hline\textbf{Destroys} & \texttt{A}, \texttt{BC}, \texttt{D},
            \texttt{HL} \\
            \hline\multirow[t]{2}{4em}{\textbf{Calls}}
            & \hyperref[func:fbiosvdpsetaddrwr]{F\_BIOS\_VDP\_SET\_ADDR\_WR}\\
            & \hyperref[func:fbiosvdpsetregister]{F\_BIOS\_VDP\_SET\_REGISTER}\\
            \hline
        \end{tabular}

        % ==========================================================================
        \subsubsection{F\_BIOS\_VDP\_BYTE\_TO\_VRAM}
        \label{func:fbiosvdpbytetovram}
        % ==========================================================================
        \begin{tabular}{l p{15cm}}
            \hline\textbf{Action}
            & Writes a byte to currently pointed \textbf{VRAM} cell. The 
            \textbf{VDP} autoincrements the \textbf{VRAM} address whenever a
            Read or a Write to \textbf{VRAM} is performed.\\
            \hline\textbf{Entry} & \texttt{A} = byte to be written\\
            \hline\textbf{Exit} & \textbf{VRAM} address autoincremented\\
            \hline\textbf{Destroys} & \texttt{C} \\
            \hline\textbf{Calls} & None\\
            \hline
        \end{tabular}

        % ==========================================================================
        \subsubsection{F\_BIOS\_VDP\_VRAM\_TO\_BYTE}
        \label{func:fbiosvdpvramtobyte}
        % ==========================================================================
        \begin{tabular}{l p{15cm}}
            \hline\textbf{Action}
            & Read a byte from \textbf{VRAM}. The \textbf{VDP} autoincrements
            the \textbf{VRAM} address whenever a Read or a Write to
            \textbf{VRAM} is performed.\\
            \hline\textbf{Entry} & None\\
            \hline\textbf{Exit} & \texttt{A} = read byte, \textbf{VRAM} address
            autoincremented.\\
            \hline\textbf{Destroys} & \texttt{A}, \texttt{C} \\
            \hline\textbf{Calls} & None\\
            \hline
        \end{tabular}

        % ==========================================================================
        \subsubsection{F\_BIOS\_VDP\_JIFFY\_COUNTER}
        \label{func:fbiosvdpjiffycounter}
        % ==========================================================================
        \begin{tabular}{l p{15cm}}
            \hline\textbf{Action}
            & Increments the \hyperref[subsec:jiffy_counter]{Jiffy Counter}.\\
            \hline\textbf{Entry} & None\\
            \hline\textbf{Exit} & None\\
            \hline\textbf{Destroys} & \texttt{A}, \texttt{IX},
            \texttt{VDP\_jiffy\_byte1}, \texttt{VDP\_jiffy\_byte2},
            \texttt{VDP\_jiffy\_byte3}\\
            \hline\textbf{Calls} & None\\
            \hline
        \end{tabular}

        % ==========================================================================
        \subsubsection{F\_BIOS\_VDP\_VBLANK\_WAIT}
        \label{func:fbiosvdpvblankwait}
        % ==========================================================================
        \begin{tabular}{l p{15cm}}
            \hline\multirow[t]{2}{4em}{\textbf{Action}}
            & Test if the \hyperref[sec:ram_memmap]{SYSVARS}
            \textit{VDP\_jiffy\_byte1} has changed. If not, continues waiting
            for the change in a loop. Otherwise, exits.\\
            & See the \hyperref[subsec:vdp_limitations]{VDP Limitations} section
            on the \textit{Appendixes} of this Guide for a possible bug on the
            \textbf{VDP}'s Status Register.\\
            \hline\textbf{Entry} & None\\
            \hline\textbf{Exit} & None\\
            \hline\textbf{Destroys} & \texttt{A}, \texttt{B}\\
            \hline\textbf{Calls} & None\\
            \hline
        \end{tabular}

        % ==========================================================================
        \subsubsection{F\_BIOS\_VDP\_LDIR\_VRAM}
        \label{func:fbiosvdpldirvram}
        % ==========================================================================
        \begin{tabular}{l p{15cm}}
            \hline\textbf{Action}
            & Block transfer from \textbf{RAM} to \textbf{VRAM}.\\
            \hline\multirow[t]{2}{4em}{\textbf{Entry}} & \texttt{BC} = Block 
            length (total number of bytes to copy)\\
            & \texttt{HL} = Start address of \textbf{VRAM}\\
            & \texttt{DE} = Start address of \textbf{RAM}\\
            \hline\textbf{Exit} & None\\
            \hline\textbf{Destroys} & \texttt{A}, \texttt{BC}, \texttt{DE},
            \texttt{HL}, \texttt{tmp\_byte}\\
            \hline\multirow[t]{3}{4em}{\textbf{Call}}
            & \hyperref[func:fkrndiv1616]{F\_KRN\_DIV1616}\\
            & \hyperref[func:fbiosvdpsetaddrwr]{F\_BIOS\_VDP\_SET\_ADDR\_WR}\\
            & \hyperref[func:fbiosvdpbytetovram]{F\_BIOS\_VDP\_BYTE\_TO\_VRAM}\\
            \hline
        \end{tabular}

        % ==========================================================================
        \subsubsection{F\_BIOS\_VDP\_CHAROUT\_ATXY}
        \label{func:fbiosvdpcharoutatxy}
        % ==========================================================================
        \begin{tabular}{l p{15cm}}
            \hline\multirow[t]{2}{4em}{\textbf{Action}}
            & Print a character in the \textbf{Low Resolution display}, at the
            current \textit{VDP\_cursor\_x}, \textit{VDP\_cursor\_y} postition.\\
            \hline\textbf{Entry} & \texttt{A} = Character to be printed, in
            Hexadecimal ASCII.\\
            \hline\textbf{Exit} & None\\
            \hline\textbf{Destroys} & \texttt{BC}, \texttt{DE},
            \texttt{HL}, \texttt{IX}, \texttt{VDP\_cursor\_x},
            \texttt{VDP\_cursor\_y}\\
            \hline\multirow[t]{2}{4em}{\textbf{Call}}
            & \hyperref[func:fbiosvdpsetaddrwr]{F\_BIOS\_VDP\_SET\_ADDR\_WR}\\
            & \hyperref[func:fbiosvdpbytetovram]{F\_BIOS\_VDP\_BYTE\_TO\_VRAM}\\
            \hline
        \end{tabular}