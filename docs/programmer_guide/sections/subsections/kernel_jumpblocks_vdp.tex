% ==========================================================================
\subsection{VDP Routines}
% ==========================================================================

    % ==========================================================================
    \subsubsection{F\_KRN\_VDP\_WRSTR}
    \label{func:fkrnvdpwrstr}
    % ==========================================================================
    \begin{tabular}{l p{15cm}}
        \hline\textbf{Action}
        & Displays a text in the VDP screen, starting at a specified XY
        position. The text MUST be a zero terminated string. \\
        \hline\multirow[t]{3}{4em}{\textbf{Entry}}
        & \texttt{B} = Cursor X (horizontal) start position.\\
        & \texttt{C} = Cursor Y (vertical) start position.\\
        & \texttt{HL} = \textbf{RAM} address of a zero terminated string.\\
        \hline\textbf{Exit} & None\\
        \hline\textbf{Destroys} & \texttt{A}, \texttt{VDP\_cursor\_x},
            \texttt{VDP\_cursor\_y}, \texttt{HL} \\
        \hline\textbf{Calls} & 
        \hyperref[func:fkrnvdpcharoutatxy]{F\_KRN\_VDP\_CHAROUT\_ATXY}\\
        \hline
    \end{tabular}

    % ==========================================================================
    \subsubsection{F\_KRN\_VDP\_GET\_CURSOR\_ADDR}
    \label{func:fkrnvdpgetcursoraddr}
    % ==========================================================================
    \begin{tabular}{l p{15cm}}
        \hline\textbf{Action}
        & Returns the \textbf{VRAM} address of a specific XY position on the
        screen. \\
        \hline\multirow[t]{2}{4em}{\textbf{Entry}}
        & \texttt{B} = Cursor X (horizontal) position.\\
        & \texttt{C} = Cursor Y (vertical) position.\\
        \hline\textbf{Exit} & \texttt{HL} = \textbf{VRAM} address.\\
        \hline\textbf{Destroys} & \texttt{A}, \texttt{B}, \texttt{DE},
            \texttt{HL},\texttt{IX} \\
        \hline\textbf{Calls} & None \\
        \hline
    \end{tabular}

    % ==========================================================================
    \subsubsection{F\_KRN\_VDP\_CLEARSCREEN}
    \label{func:fkrnvdpclearscreen}
    % ==========================================================================
    \begin{tabular}{l p{15cm}}
        \hline\textbf{Action}
        & Clears the \textbf{VDP} screen. \\
        \hline\textbf{Entry} & None\\
        \hline\textbf{Exit} & None\\
        \hline\textbf{Destroys} & \texttt{A}, \texttt{B}, \texttt{DE},
            \texttt{HL} \\
        \hline\multirow[t]{3}{4em}{\textbf{Calls}}
        & \hyperref[func:fkrnserialwrstrclr]{F\_KRN\_SERIAL\_WRSTRCLR}\\
        & \hyperref[func:fbiosvdpsetaddrwr]{F\_BIOS\_VDP\_SET\_ADDR\_WR}\\
        & \hyperref[func:fbiosvdpbytetovram]{F\_BIOS\_VDP\_BYTE\_TO\_VRAM}\\
        \hline
    \end{tabular}

    % ==========================================================================
    \subsubsection{F\_KRN\_VDP\_CHG\_COLOUR\_FGBG}
    \label{func:fkrnvdpchgcolourfgbg}
    % ==========================================================================
    \begin{tabular}{l p{15cm}}
        \hline\textbf{Action}
        & Changes the Foreground and Background colours of the \textbf{VDP}
        screen. For \textit{Text Mode} also sets the border colour to the
        same as the Background colour.\\
        \hline\multirow[t]{2}{4em}{\textbf{Entry}}
        & \texttt{A} = Foreground colour.\\
        & \texttt{B} = Background colour.\\
        \hline\textbf{Exit} & None\\
        \hline\textbf{Destroys} & \texttt{A}, \texttt{B}\\
        \hline\textbf{Calls}
        & \hyperref[func:fbiosvdpsetregister]{F\_BIOS\_VDP\_SET\_REGISTER}\\
        \hline
    \end{tabular}

    % ==========================================================================
    \subsubsection{F\_KRN\_VDP\_CHG\_COLOUR\_BORDER}
    \label{func:fkrnvdpchgcolourborder}
    % ==========================================================================
    \begin{tabular}{l p{15cm}}
        \hline\textbf{Action}
        & Changes the Border colour of the \textbf{VDP} screen, for screen
        modes other than \textit{Text Mode}. In \textit{Text Mode} the
        Border (backdrop) colour is the same as the Background colour.\\
        \hline\textbf{Entry} & \texttt{B} = Border colour.\\
        \hline\textbf{Exit} & None\\
        \hline\textbf{Destroys} & \texttt{A}\\
        \hline\textbf{Calls}
        & \hyperref[func:fbiosvdpsetregister]{F\_BIOS\_VDP\_SET\_REGISTER}\\
        \hline
    \end{tabular}

    % ==========================================================================
    \subsubsection{F\_KRN\_VDP\_SET\_MODE}
    \label{func:fkrnvdpsetmode}
    % ==========================================================================
    \begin{tabular}{l p{15cm}}
        \hline\textbf{Action}
        & Changes the \textbf{VDP} screen mode.\\
        \hline\textbf{Entry} & \texttt{A} = VDP Mode (0-4).\\
        \hline\textbf{Exit} & None\\
        \hline\textbf{Destroys} & \texttt{A}\\
        \hline\multirow[t]{5}{4em}{\textbf{Calls}}
        & \hyperref[func:fbiosvdpsetmodetxt]{F\_BIOS\_VDP\_SET\_MODE\_TXT}\\
        & \hyperref[func:fbiosvdpsetmodeg1]{F\_BIOS\_VDP\_SET\_MODE\_G1}\\
        & \hyperref[func:fbiosvdpsetmodeg2]{F\_BIOS\_VDP\_SET\_MODE\_G2}\\
        & \hyperref[func:fbiosvdpsetmodeg2bm]{F\_BIOS\_VDP\_SET\_MODE\_G2BM}\\
        & \hyperref[func:fbiosvdpsetmodemulticlr]{F\_BIOS\_VDP\_SET\_MODE\_MULTICLR}\\
        \hline
    \end{tabular}

    % ==========================================================================
    \subsubsection{F\_KRN\_VDP\_CHAROUT\_ATXY}
    \label{func:fkrnvdpcharoutatxy}
    % ==========================================================================
    \begin{tabular}{l p{15cm}}
        \hline\textbf{Action}
        & Print a character in the \textbf{Low Resolution display}, at the
            current \textit{VDP\_cursor\_x}, \textit{VDP\_cursor\_y} postition.\\
        & \textit{VDP\_cursor\_x} is incremented by 1, and if it has reached
            the maximum width (Mode 0 = 40, others = 32), resets it to zero and
            increases \textit{VDP\_cursor\_y} by 1.\\
        \hline\textbf{Entry} & \texttt{A} = Character to be printed, in
            Hexadecimal ASCII.\\
        \hline\textbf{Exit} & None\\
        \hline\textbf{Destroys} & \texttt{A}, \texttt{HL},
        \texttt{VDP\_cursor\_x}, \texttt{VDP\_cursor\_y}\\
        \hline\textbf{Calls}
        & \hyperref[func:fbiosvdpcharoutatxy]{F\_BIOS\_VDP\_CHAROUT\_ATXY}\\
        \hline
    \end{tabular}