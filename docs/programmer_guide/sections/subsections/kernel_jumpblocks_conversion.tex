% ==========================================================================
\subsection{Conversion Routines}
% ==========================================================================

    % ==========================================================================
    \subsubsection{F\_KRN\_ASCIIADR\_TO\_HEX}
    \label{func:fkrnasciiadrtohex}
    % ==========================================================================
    \begin{tabular}{l p{9cm}}
        \hline\textbf{Action}
        & Convert an address (or any 2 bytes) from hex ASCII to its 
        hexadecimal value (e.g. 32 35 37 30 are converted into 2570).\\
        \hline\textbf{Entry} & \texttt{IX} = \textbf{MEMORY} address where
        the first byte is located.\\
        \hline\textbf{Exit} & \texttt{HL} = hexadecimal converted value.\\
        \hline\textbf{Destroys} & \texttt{HL}\\
        \hline\textbf{Calls}
        & \hyperref[func:fkrnasciitohex]{F\_KRN\_ASCII\_TO\_HEX}\\
        \hline
    \end{tabular}

    % ==========================================================================
    \subsubsection{F\_KRN\_ASCII\_TO\_HEX}
    \label{func:fkrnasciitohex}
    % ==========================================================================
    \begin{tabular}{l p{9cm}}
        \hline\textbf{Action}
        & Converts two ASCII characters (representing two hexadecimal digits)
        ; to one byte in hexadecimal (e.g. \texttt{0x33} and \texttt{0x45}
        are converted into \texttt{3E}).\\
        \hline\multirow[t]{2}{4em}{\textbf{Entry}}
        & \texttt{H} = Most significant ASCII digit.\\
        & \texttt{L} = Less significant ASCII digit.\\
        \hline\textbf{Exit} & \texttt{A} = Converted value.\\
        \hline\textbf{Destroys} & \texttt{A}, \texttt{BC}\\
        \hline\textbf{Calls} & None\\
        \hline
    \end{tabular}

    % ==========================================================================
    \subsubsection{F\_KRN\_HEX\_TO\_ASCII}
    \label{func:fkrnhextoascii}
    % ==========================================================================
    \begin{tabular}{l p{9cm}}
        \hline\textbf{Action}
        & Converts one byte in hexadecimal to two ASCII printable characters
        (e.g. \texttt{0x3E} is converted into 33 and 45, which are the ASCII
        values of \texttt{3} and \texttt{E}).\\
        \hline\textbf{Entry} & \texttt{A} = Byte to convert.\\
        \hline\multirow[t]{2}{4em}{\textbf{Exit}}
        & \texttt{H} = Most significant ASCII digit.\\
        & \texttt{L} = Less significant ASCII digit.\\
        \hline\textbf{Destroys} & \texttt{A}, \texttt{BC}, \texttt{HL}\\
        \hline\textbf{Calls} & None\\
        \hline
    \end{tabular}

    % ==========================================================================
    \subsubsection{F\_KRN\_BCD\_TO\_BIN}
    \label{func:fkrnbcdtobin}
    % ==========================================================================
    \begin{tabular}{l p{9cm}}
        \hline\textbf{Action}
        & Converts a byte of BCD to a byte of hexadecimal
        (e.g. 12 is converted into \texttt{0x0C}).\\
        \hline\textbf{Entry} & \texttt{A} = BCD.\\
        \hline\textbf{Exit} & \texttt{A} = Hexadecimal.\\
        \hline\textbf{Destroys} & \texttt{A}, \texttt{BC}\\
        \hline\textbf{Calls} & None\\
        \hline
    \end{tabular}

    % ==========================================================================
    \subsubsection{F\_KRN\_BIN\_TO\_BCD4}
    \label{func:fkrnbintobcd4}
    % ==========================================================================
    \begin{tabular}{l p{9cm}}
        \hline\textbf{Action}
        & Converts a byte of unsigned integer hexadecimal to 4-digit BCD
        (e.g. \texttt{0x80} is converted into 0128).\\
        \hline\textbf{Entry} & \texttt{A} = Unsigned integer to convert.\\
        \hline\multirow[t]{2}{4em}{\textbf{Exit}}
        & \texttt{H} = Hundreds digits.\\
        & \texttt{L} = Tens digits.\\
        \hline\textbf{Destroys} & \texttt{A}, \texttt{BC}, \texttt{HL}\\
        \hline\textbf{Calls} & None\\
        \hline
    \end{tabular}

    % ==========================================================================
    \subsubsection{F\_KRN\_BIN\_TO\_BCD6}
    \label{func:fkrnbintobcd6}
    % ==========================================================================
    \begin{tabular}{l p{9cm}}
        \hline\textbf{Action}
        & Converts two bytes of unsigned integer hexadecimal to 6-digit BCD
        (e.g. \texttt{0xFFFF} is converted into 065535).\\
        \hline\textbf{Entry} & \texttt{HL} = Unsigned integer to convert.\\
        \hline\multirow[t]{3}{4em}{\textbf{Exit}}
        & \texttt{C} = Thousands digits.\\
        & \texttt{D} = Hundreds digits.\\
        & \texttt{E} = Tens digits.\\
        \hline\textbf{Destroys} & \texttt{A}, \texttt{BC}, \texttt{DE}, \texttt{HL}\\
        \hline\textbf{Calls} & None\\
        \hline
    \end{tabular}

    % ==========================================================================
    \subsubsection{F\_KRN\_BCD\_TO\_ASCII}
    \label{func:fkrnbcdtoascii}
    % ==========================================================================
    \begin{tabular}{l p{9cm}}
        \hline\textbf{Action}
        & Converts 6-digit BCD to hexadecimal ASCII string
        (e.g. 512 is converted into 30 30 30 35 31 32).\\
        \hline\multirow[t]{4}{4em}{\textbf{Entry}}
        & \texttt{DE} = \textbf{MEMORY} address where the converted string
        will be stored.\\
        & \texttt{C} = first two digits of the 6-digit BCD to convert.\\
        & \texttt{H} = next two digits of the 6-digit BCD to convert.\\
        & \texttt{L} = last two digits of the 6-digit BCD to convert.\\
        \hline\textbf{Exit} & None\\
        \hline\textbf{Destroys} & \texttt{A}, \texttt{DE}\\
        \hline\textbf{Calls} & None\\
        \hline
    \end{tabular}

    % ==========================================================================
    \subsubsection{F\_KRN\_BITEXTRACT}
    \label{func:fkrnbitextract}
    % ==========================================================================
    \begin{tabular}{l p{9cm}}
        \hline\textbf{Action}
        & Extracts a group of bits from a byte and returns the group in the
        LSB position.\\
        \hline\multirow[t]{3}{4em}{\textbf{Entry}}
        & \texttt{E} = byte from where to extract bits.\\
        & \texttt{D} = number of bits to extract.\\
        & \texttt{A} = start extraction at bit number.\\
        \hline\textbf{Exit} & \texttt{A} = extracted group of bits\\
        \hline\textbf{Destroys} & \texttt{A}, \texttt{BC}, \texttt{DE}, 
        \texttt{HL}\\
        \hline\textbf{Calls} & None\\
        \hline
    \end{tabular}

    % ==========================================================================
    \subsubsection{F\_KRN\_BIN\_TO\_ASCII}
    \label{func:fkrnbintoascii}
    % ==========================================================================
    \begin{tabular}{l p{9cm}}
        \hline\textbf{Action}
        & Converts a 16-bit signed binary number (-32768 to 32767) to ASCII
        data (e.g. 32767 is converted into 33 32 37 36 37).\\
        \hline\multirow[t]{2}{4em}{\textbf{Entry}}
        & \texttt{D} = High byte of value to convert.\\
        & \texttt{E} = Low byte of value to convert.\\
        \hline\textbf{Exit} & \texttt{CLI\_buffer\_pgm} = converted ASCII data.
        First byte us the length.\\
        \hline\textbf{Destroys} & \texttt{A}, \texttt{BC}, \texttt{DE}, 
        \texttt{HL}, \texttt{CLI\_buffer\_pgm}\\
        \hline\textbf{Calls} & None\\
        \hline
    \end{tabular}

    % ==========================================================================
    \subsubsection{F\_KRN\_DEC\_TO\_BIN}
    \label{func:fkrndectobin}
    % ==========================================================================
    \begin{tabular}{l p{9cm}}
        \hline\textbf{Action}
        & Converts an ASCII string consisting of the length of the number
        (in bytes), a possible ASCII - or + sign, and a series of ASCII
        digits to two bytes of binary data. Note that the length is an
        ordinary binary number, not an ASCII number. (e.g. 05 33 32 37 36 37
        is converted into 7FFF).\\
        \hline\textbf{Entry} & \texttt{HL} = \textbf{MEMORY} address where
        the string to be converted is.\\
        \hline\textbf{Exit} & \texttt{HL} = converted bytes.\\
        \hline\textbf{Destroys} & \texttt{A}, \texttt{BC}, \texttt{DE}, 
        \texttt{HL}, \texttt{tmp\_byte}\\
        \hline\textbf{Calls} & None\\
        \hline
    \end{tabular}

    % ==========================================================================
    \subsubsection{F\_KRN\_PKEDDATE\_TO\_DMY}
    \label{func:fkrnpkeddatetodmy}
    % ==========================================================================
    \begin{tabular}{l p{9cm}}
        \hline\textbf{Action}
        & Extracts day, month and year from a packed date (used by DZFS to
        store dates).\\
        \hline\textbf{Entry} & \texttt{HL} = packed date.\\
        \hline\multirow[t]{3}{4em}{\textbf{Exit}}
        & \texttt{A} = day.\\
        & \texttt{B} = month.\\
        & \texttt{C} = year.\\
        \hline\textbf{Destroys} & \texttt{A}, \texttt{BC}, \texttt{HL},
        \texttt{tmp\_addr1}\\
        \hline\textbf{Calls} & None\\
        \hline
    \end{tabular}

    % ==========================================================================
    \subsubsection{F\_KRN\_PKEDTIME\_TO\_HMS}
    \label{func:fkrnpkedtimetohms}
    % ==========================================================================
    \begin{tabular}{l p{9cm}}
        \hline\textbf{Action}
        & Extracts hour, minutes and seconds from a packed time (used by
        DZFS to store times).\\
        \hline\textbf{Entry} & \texttt{HL} = packed time.\\
        \hline\multirow[t]{3}{4em}{\textbf{Exit}}
        & \texttt{A} = hour.\\
        & \texttt{B} = minutes.\\
        & \texttt{C} = seconds.\\
        \hline\textbf{Destroys} & \texttt{A}, \texttt{BC}, \texttt{HL},
        \texttt{tmp\_addr1}\\
        \hline\textbf{Calls} & None\\
        \hline
    \end{tabular}