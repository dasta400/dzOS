% ==========================================================================
\subsection{String manipulation Routines}
% ==========================================================================

    % ==========================================================================
    \subsubsection{F\_KRN\_IS\_PRINTABLE}
    \label{func:fkrnisprintable}
    % ==========================================================================
    \begin{tabular}{l p{15cm}}
        \hline\textbf{Action}
        & Checks if a character is a printable ASCII character.\\
        \hline\textbf{Entry} & \texttt{A} = character to check.\\
        \hline\textbf{Exit} & \texttt{C Flag} is set if character is printable.\\
        \hline\textbf{Destroys} & None\\
        \hline\textbf{Calls} & None\\
        \hline
    \end{tabular}

    % ==========================================================================
    \subsubsection{F\_KRN\_IS\_NUMERIC}
    \label{func:fkrnisnumeric}
    % ==========================================================================
    \begin{tabular}{l p{15cm}}
        \hline\textbf{Action}
        & Checks if a character is numeric (0, 1, 2, 3, 4, 5, 6, 7, 8 or 9).\\
        \hline\textbf{Entry} & \texttt{A} = character to check.\\
        \hline\textbf{Exit} & \texttt{C Flag} is set if character is numeric.\\
        \hline\textbf{Destroys} & None\\
        \hline\textbf{Calls} & None\\
        \hline
    \end{tabular}

    % ==========================================================================
    \subsubsection{F\_KRN\_TOUPPER}
    \label{func:fkrntoupper}
    % ==========================================================================
    \begin{tabular}{l p{15cm}}
        \hline\textbf{Action}
        & Converts a charcater to uppercase (e.g. \textit{a} is converted to
        \texttt{A}).\\
        \hline\textbf{Entry} & \texttt{A} = character to convert.\\
        \hline\textbf{Exit} & \texttt{A} = uppercased character.\\
        \hline\textbf{Destroys} & None\\
        \hline\textbf{Calls} & None\\
        \hline
    \end{tabular}

    % ==========================================================================
    \subsubsection{F\_KRN\_STRCMP}
    \label{func:fkrnstrcmp}
    % ==========================================================================
    \begin{tabular}{l p{15cm}}
        \hline\textbf{Action}
        & Compares two strings.\\
        \hline\multirow[t]{4}{4em}{\textbf{Entry}}
        & \texttt{A} = length of string 1.\\
        & \texttt{HL} = \textbf{MEMORY} address where the first byte of
        string 1 is located.\\
        & \texttt{B} = length of string 2.\\
        & \texttt{DE} = \textbf{MEMORY} address where the first byte of
        string 2 is located.\\
        \hline\multirow[t]{4}{4em}{\textbf{Exit}}
        & if str1 = str 2, \texttt{Z Flag} set and \texttt{C Flag} not set.\\
        & if str1 != str 2 and str1 longer than str2, \texttt{Z Flag} not 
        set and \texttt{C Flag} not set.\\
        & if str1 != str 2 and str1 shorter than str2, \texttt{Z Flag} not 
        set and \texttt{C Flag} set.\\
        \hline\textbf{Destroys} & \texttt{A}, \texttt{BC}, \texttt{DE},\texttt{HL} \\
        \hline\textbf{Calls} & None\\
        \hline
    \end{tabular}

    % ==========================================================================
    \subsubsection{F\_KRN\_STRCPY}
    \label{func:fkrnstrcpy}
    % ==========================================================================
    \begin{tabular}{l p{15cm}}
        \hline\textbf{Action}
        & Copies \textit{n} characters from string 1 to string 2.\\
        \hline\multirow[t]{3}{4em}{\textbf{Entry}}
        & \texttt{HL} = \textbf{MEMORY} address where the first byte of
        string 1 is located.\\
        & \texttt{DE} = \textbf{MEMORY} address where the first byte of
        string 2 is located.\\
        & \texttt{B} = number of characters to copy.\\
        \hline\textbf{Exit} & None\\
        \hline\textbf{Destroys} & \texttt{A}, \texttt{DE}, \texttt{HL}\\
        \hline\textbf{Calls} & None\\
        \hline
    \end{tabular}

    % ==========================================================================
    \subsubsection{F\_KRN\_STRLEN}
    \label{func:fkrnstrlen}
    % ==========================================================================
    \begin{tabular}{l p{15cm}}
        \hline\textbf{Action}
        & Gets the length of a string that is terminated with a specified
        character.\\
        \hline\multirow[t]{2}{4em}{\textbf{Entry}}
        & \texttt{HL} = \textbf{MEMORY} address where the first byte of the
        string is located.\\
        & \texttt{A} = terminating character.\\
        \hline\textbf{Exit} & \texttt{B} = lenght of the string.\\
        \hline\textbf{Destroys} & \texttt{BC}, \texttt{HL}\\
        \hline\textbf{Calls} & None\\
        \hline
    \end{tabular}

    % ==========================================================================
    \subsubsection{F\_KRN\_STRLENMAX}
    \label{func:fkrnstrlenmax}
    % ==========================================================================
    \begin{tabular}{l p{15cm}}
        \hline\textbf{Action}
        & Gets the length of a string that is terminated with a specified
        character, but only check up to a maximum of characters.\\
        \hline\multirow[t]{3}{4em}{\textbf{Entry}}
        & \texttt{HL} = \textbf{MEMORY} address where the first byte of the
        string is located.\\
        & \texttt{A} = terminating character.\\
        & \texttt{B} = maximum length to be checked.\\
        \hline\textbf{Exit} & \texttt{B} = lenght of the string.\\
        \hline\textbf{Destroys} & \texttt{BC}, \texttt{DE}, \texttt{HL}\\
        \hline\textbf{Calls} & None\\
        \hline
    \end{tabular}

    % ==========================================================================
    \subsubsection{F\_KRN\_INSTR}
    \label{func:fkrninstr}
    % ==========================================================================
    \begin{tabular}{l p{15cm}}
        \hline\textbf{Action}
        & Locates the first occurrence of a character within a string.\\
        \hline\multirow[t]{3}{4em}{\textbf{Entry}}
        & \texttt{HL} = \textbf{MEMORY} address where the first byte of the
        string is located.\\
        & \texttt{B} = character to search in string.\\
        & \texttt{D} = terminating character.\\
        \hline\multirow[t]{2}{4em}{\textbf{Exit}}
        & \texttt{E} = position of character in string.\\
        & \texttt{Carry Flag} = Set if character was found.\\
        \hline\textbf{Destroys} & \texttt{A}, \texttt{C}, \texttt{E}\\
        \hline\textbf{Calls} & None\\
        \hline
    \end{tabular}

    % ==========================================================================
    \subsubsection{F\_KRN\_STRCHR}
    \label{func:fkrnstrchr}
    % ==========================================================================
    \begin{tabular}{l p{15cm}}
        \hline\textbf{Action}
        & Finds the first occurrence of a character in a string terminated by a
            specified character.\\
        \hline\multirow[t]{3}{4em}{\textbf{Entry}}
        & \texttt{HL} = \textbf{MEMORY} address where the first byte of the
        string is located.\\
        & \texttt{D} = terminating character.\\
        & \texttt{E} = character to search in string.\\
        \hline\multirow[t]{2}{4em}{\textbf{Exit}}
        & \texttt{HL} = \textbf{MEMORY} address to the character found.\\
        & \texttt{Carry Flag} = Set if character was found.\\
        \hline\textbf{Destroys} & \texttt{A}, \texttt{HL}\\
        \hline\textbf{Calls} & None\\
        \hline
    \end{tabular}

    % ==========================================================================
    \subsubsection{F\_KRN\_STRCHRNTH}
    \label{func:fkrnstrchr}
    % ==========================================================================
    \begin{tabular}{l p{15cm}}
        \hline\textbf{Action}
        & Finds the \textit{nth} occurrence of a character in a string terminated by a
            specified character.\\
        \hline\multirow[t]{4}{4em}{\textbf{Entry}}
        & \texttt{HL} = \textbf{MEMORY} address where the first byte of the
        string is located.\\
        & \texttt{D} = terminating character.\\
        & \texttt{E} = character to search in string.\\
        & \texttt{B} = occurrence number (\textit{nth}).\\
        \hline\multirow[t]{2}{4em}{\textbf{Exit}}
        & \texttt{HL} = \textbf{MEMORY} address to the character found.\\
        & \texttt{Carry Flag} = Set if character was found.\\
        \hline\textbf{Destroys} & \texttt{A}, \texttt{B}, \texttt{HL}\\
        \hline\textbf{Calls} & None\\
        \hline
    \end{tabular}