% ==========================================================================
\subsection{Serial Routines}
% ==========================================================================

    % ==========================================================================
    \subsubsection{F\_KRN\_SERIAL\_SETFGCOLR}
    \label{func:fkrnserialsetfgcolr}
    % ==========================================================================
    \begin{tabular}{l p{9cm}}
        \hline\multirow[t]{4}{4em}{\textbf{Action}}
        & Set the colour that will be used for the foreground (text). \\
        & The colour will remain until a different one is set. \\
        \hline\textbf{Entry} & \texttt{A} = Colour number (as listed in 
        \hyperref[sec:appendixes]{Appendixes} section) \\
        \hline\textbf{Exit} & None \\
        \hline\textbf{Destroys} & \texttt{B}, \texttt{DE} \\
        \hline\multirow[t]{2}{4em}{\textbf{Calls}} 
        & \hyperref[func:fbiosserialconouta]{F\_BIOS\_SERIAL\_CONOUT\_A}\\
        & \textit{jp} \hyperref[func:fkrnserialsendansicode]{F\_KRN\_SERIAL\_SEND\_ANSI\_CODE}\\
        \hline
    \end{tabular}

    % ==========================================================================
    \subsubsection{F\_KRN\_SERIAL\_WRSTR}
    \label{func:fkrnserialwrstr}
    % ==========================================================================
    \begin{tabular}{l p{9cm}}
        \hline\multirow[t]{4}{4em}{\textbf{Action}}
        & Outputs a string, terminated with Carriage Return to the 
        \textbf{CONSOLE}.\\
        \hline\textbf{Entry} 
        & \texttt{HL} = address in \textbf{MEMORY} where the first character
        of the string to be output is.\\
        \hline\textbf{Exit} & None \\
        \hline\textbf{Destroys} & \texttt{A}, \texttt{HL} \\
        \hline\textbf{Calls}
        & \hyperref[func:fbiosserialconouta]{F\_BIOS\_SERIAL\_CONOUT\_A}\\
        \hline
    \end{tabular}

    % ==========================================================================
    \subsubsection{F\_KRN\_SERIAL\_WRSTRCLR}
    \label{func:fkrnserialwrstrclr}
    % ==========================================================================
    \begin{tabular}{l p{9cm}}
        \hline\textbf{Action}
        & Outputs a string, terminated with Carriage Return to the 
        \textbf{CONSOLE}, with a specific foreground colour. \\
        \hline\multirow[t]{2}{4em}{\textbf{Entry}}
        & \texttt{A} = Colour number (as listed in 
        \hyperref[sec:appendixes]{Appendixes} section) \\
        & \texttt{HL} = address in \textbf{MEMORY} where the first character
        of the string to be output is.\\
        \hline\textbf{Exit} & None \\
        \hline\textbf{Destroys} & \texttt{B}, \texttt{DE} \\
        \hline\multirow[t]{2}{4em}{\textbf{Calls}} 
        & \hyperref[func:fkrnserialsetfgcolr]{F\_KRN\_SERIAL\_SETFGCOLR}\\
        & \textit{jp} \hyperref[func:fkrnserialwrstr]{F\_KRN\_SERIAL\_WRSTR}\\
        \hline
    \end{tabular}

    % ==========================================================================
    \subsubsection{F\_KRN\_SERIAL\_WR6DIG\_NOLZEROS}
    \label{func:fkrnserialwr6dignolzeros}
    % ==========================================================================
    \begin{tabular}{l p{9cm}}
        \hline\multirow[t]{2}{4em}{\textbf{Action}}
        & Outputs to the \textbf{CONSOLE} a string of ASCII characters 
        representing a number, without outputing the leading zeros.\\
        & (.e.g. 30 30 31 32 30 34 is 001204, but the output wil be 1024)\\
        \hline\textbf{Entry} 
        & \texttt{IX} = address in \textbf{MEMORY} where the ASCII 
        characters are stored.\\
        \hline\textbf{Exit} & None \\
        \hline\textbf{Destroys} & \texttt{A}, \texttt{B}, \texttt{DE}, \texttt{IX} \\
        \hline\textbf{Calls}
        & \hyperref[func:fbiosserialconouta]{F\_BIOS\_SERIAL\_CONOUT\_A}\\
        \hline
    \end{tabular}

    % ==========================================================================
    \subsubsection{F\_KRN\_SERIAL\_RDCHARECHO}
    \label{func:fkrnserialrdcharecho}
    % ==========================================================================
    \begin{tabular}{l p{9cm}}
        \hline\textbf{Action}
        & Reads with echo. Reads a character from the \textbf{SIO/2} Channel
        A, and outputs it to the \textbf{CONSOLE}.\\
        \hline\textbf{Entry} & None \\
        \hline\textbf{Exit} & \texttt{A} = read character. \\
        \hline\textbf{Destroys} & None \\
        \hline\multirow[t]{2}{4em}{\textbf{Calls}}
        & \hyperref[func:fbiosserialconina]{F\_BIOS\_SERIAL\_CONIN\_A}\\
        & \hyperref[func:fbiosserialconouta]{F\_BIOS\_SERIAL\_CONOUT\_A}\\
        \hline
    \end{tabular}

    % ==========================================================================
    \subsubsection{F\_KRN\_SERIAL\_EMPTYLINES}
    \label{func:fkrnserialemptylines}
    % ==========================================================================
    \begin{tabular}{l p{9cm}}
        \hline\textbf{Action}
        & Outputs \textit{n} number of empty lines to the \textbf{CONSOLE}.\\
        \hline\textbf{Entry} & \texttt{B} = number (\textit{n}) of empty
        lines to output. \\
        \hline\textbf{Exit} & None \\
        \hline\textbf{Destroys} & \texttt{A} \\
        \hline\textbf{Calls}
        & \hyperref[func:fbiosserialconouta]{F\_BIOS\_SERIAL\_CONOUT\_A}\\
        \hline
    \end{tabular}

    % ==========================================================================
    \subsubsection{F\_KRN\_SERIAL\_PRN\_NIBBLE}
    \label{func:fkrnserialprnnibble}
    % ==========================================================================
    \begin{tabular}{l p{9cm}}
        \hline\textbf{Action}
        & Outputs a single hexadecimal nibble in hexadecimal notation.\\
        \hline\textbf{Entry}
        & \texttt{A} = nibble to output. Nibble will be the less significant 
        4 bits of the byte.\\
        \hline\textbf{Exit} & None \\
        \hline\textbf{Destroys} & \texttt{A} \\
        \hline\textbf{Calls}
        & \hyperref[func:fbiosserialconouta]{F\_BIOS\_SERIAL\_CONOUT\_A}\\
        \hline
    \end{tabular}

    % ==========================================================================
    \subsubsection{F\_KRN\_SERIAL\_PRN\_BYTE}
    \label{func:fkrnserialprnbyte}
    % ==========================================================================
    \begin{tabular}{l p{9cm}}
        \hline\textbf{Action}
        & Outputs a single hexadecimal byte in hexadecimal notation.\\
        \hline\textbf{Entry}
        & \texttt{A} = byte to output.\\
        \hline\textbf{Exit} & None \\
        \hline\textbf{Destroys} & \texttt{A} \\
        \hline\textbf{Calls}
        & \hyperref[func:fbiosserialconouta]{F\_BIOS\_SERIAL\_CONOUT\_A}\\
        \hline
    \end{tabular}

    % ==========================================================================
    \subsubsection{F\_KRN\_SERIAL\_PRN\_BYTES}
    \label{func:fkrnserialprnbytes}
    % ==========================================================================
    \begin{tabular}{l p{9cm}}
        \hline\textbf{Action}
        & Outputs \textit{n} number of bytes as ASCII characters.\\
        \hline\multirow[t]{2}{4em}{\textbf{Entry}}
        & \texttt{B} = number (\textit{n}) of bytes to output.\\
        & \texttt{HL} = address in \textbf{MEMORY} where the first byte to
        output is. \\
        \hline\textbf{Exit} & None \\
        \hline\textbf{Destroys} & \texttt{A}, \texttt{HL} \\
        \hline\textbf{Calls}
        & \hyperref[func:fbiosserialconouta]{F\_BIOS\_SERIAL\_CONOUT\_A}\\
        \hline
    \end{tabular}

    % ==========================================================================
    \subsubsection{F\_KRN\_SERIAL\_PRN\_WORD}
    \label{func:fkrnserialprnword}
    % ==========================================================================
    \begin{tabular}{l p{9cm}}
        \hline\textbf{Action}
        & Outputs the 4 hexadecimal digits of a word in hexadecimal notation.\\
        \hline\textbf{Entry}
        & \texttt{HL} = word to be output.\\
        \hline\textbf{Exit} & None \\
        \hline\textbf{Destroys} & \texttt{A} \\
        \hline\textbf{Calls}
        & \hyperref[func:fkrnserialprnbyte]{F\_KRN\_SERIAL\_PRN\_BYTE}\\
        \hline
    \end{tabular}

    % ==========================================================================
    \subsubsection{F\_KRN\_SERIAL\_SEND\_ANSI\_CODE}
    \label{func:fkrnserialsendansicode}
    % ==========================================================================
    \begin{tabular}{l p{9cm}}
        \hline\textbf{Action}
        & Writes an ANSI code to the \textbf{SIO/2} Channel A. \\
        \hline\textbf{Entry}
        & \texttt{DE} = address in \textbf{MEMORY} where the first byte of
        ANSI escape code is.\\
        & \texttt{B} = number of bytes in the ANSI escape code.\\
        \hline\textbf{Exit} & None \\
        \hline\textbf{Destroys} & \texttt{A}, \texttt{DE} \\
        \hline\textbf{Calls}
        & \hyperref[func:fbiosserialconouta]{F\_BIOS\_SERIAL\_CONOUT\_A}\\
        \hline
    \end{tabular}

    % ==========================================================================
    \subsubsection{F\_KRN\_SERIAL\_CLR\_SIOCHA\_BUFFER}
    \label{func:fkrnserialclrsiochabuffer}
    % ==========================================================================
    \begin{tabular}{l p{9cm}}
        \hline\textbf{Action}
        & Clear (sets to zeros) the SIO Channel A Buffer. \\
        \hline\textbf{Entry} & None\\
        \hline\textbf{Exit} & None \\
        \hline\textbf{Destroys} & \texttt{A}, \texttt{B}, \texttt{HL},
        \texttt{SIO\_CH\_A\_BUFFER\_USED}, \texttt{SIO\_CH\_A\_IN\_PTR},
        \texttt{SIO\_CH\_A\_RD\_PTR}\\
        \hline\textbf{Calls} & None\\
        \hline
    \end{tabular}