% ==========================================================================
\subsection{DZFS (file system) Routines}
% ==========================================================================

    % ==========================================================================
    \subsubsection{F\_KRN\_DZFS\_READ\_SUPERBLOCK}
    \label{func:fkrndzfsreadsuperblock}
    % ==========================================================================
    \begin{tabular}{l p{9cm}}
        \hline\multirow[t]{2}{4em}{\textbf{Action}}
        & Reads 512 bytes from Sector 0 (corresponding to the DZFS 
        \textit{Superblock}) into the disk buffer in \textbf{MEMORY}.\\
        & If the \textit{Superblock} does not containe the correct DZFS
        signature, \texttt{DISK\_is\_formatted} is set to \texttt{0x00}. Otherwise, is
        set to \texttt{0x01}.\\
        \hline\textbf{Entry} & None \\
        \hline\textbf{Exit} & None \\
        \hline\textbf{Destroys} & \texttt{A}, \texttt{DE},
        \texttt{DISK\_is\_formatted} \\
        \hline\textbf{Calls}
        & \hyperref[func:fbiosdiskreadsec]{F\_BIOS\_SD\_READ\_SEC}\\
        \hline
    \end{tabular}

    % ==========================================================================
    \subsubsection{F\_KRN\_DZFS\_READ\_BAT\_SECTOR}
    \label{func:fkrndzfsreadbatsector}
    % ==========================================================================
    \begin{tabular}{l p{9cm}}
        \hline\textbf{Action}
        & Reads a BAT Sector from \textbf{DISK} into \textbf{MEMORY}.\\
        \hline\textbf{Entry} & \texttt{DISK\_cur\_sector} holds the sector 
        number for the BAT.\\
        \hline\textbf{Exit} & \texttt{DISK Buffer} contains the BAT sector.\\
        \hline\textbf{Destroys} & \texttt{HL} \\
        \hline\textbf{Calls}
        & \hyperref[func:fkrndzfssectobuffer]{F\_KRN\_DZFS\_SEC\_TO\_BUFFER}\\
        \hline
    \end{tabular}

    % ==========================================================================
    \subsubsection{F\_KRN\_DZFS\_BATENTRY\_TO\_BUFFER}
    \label{func:fkrndzfsbatentrytobuffer}
    % ==========================================================================
    \begin{tabular}{l p{9cm}}
        \hline\textbf{Action}
        & Extracts the data of a BAT entry from the \texttt{DISK Buffer}
        in \textbf{MEMORY} and populates the values into System variables.\\
        \hline\textbf{Entry} & \texttt{A} = BAT entry number to extract data
        from.\\
        \hline\textbf{Exit} & DISK BAT System Variables are populated. See
        \hyperref[sec:ram_memmap]{RAM Memory Map} for for details.\\
        \hline\textbf{Destroys} & \texttt{A}, \texttt{BC}, \texttt{DE}, 
        \texttt{HL}, \texttt{IX}, \texttt{tmp\_addr1} \\
        \hline\textbf{Calls}
        & \hyperref[func:fkrnmultiply816slow]{F\_KRN\_MULTIPLY816\_SLOW}\\
        \hline
    \end{tabular}

    % ==========================================================================
    \subsubsection{F\_KRN\_DZFS\_SEC\_TO\_BUFFER}
    \label{func:fkrndzfssectobuffer}
    % ==========================================================================
    \begin{tabular}{l p{9cm}}
        \hline\textbf{Action}
        & Loads a Sector (512 bytes) from the \textbf{DISK} and copies the
        bytes into the \texttt{DISK Buffer} in \textbf{MEMORY}.\\
        \hline\textbf{Entry} & \texttt{HL} = Sector number to load.\\
        \hline\textbf{Exit} & \texttt{DISK Buffer} contains the bytes 
        of Sector loaded.\\
        \hline\textbf{Destroys} & \texttt{DE}, \texttt{HL} \\
        \hline\textbf{Calls}
        & \hyperref[func:fbiosdiskreadsec]{F\_BIOS\_SD\_READ\_SEC}\\
        \hline
    \end{tabular}

    % ==========================================================================
    \subsubsection{F\_KRN\_DZFS\_GET\_FILE\_BATENTRY}
    \label{func:fkrndzfsgetfilebatentry}
    % ==========================================================================
    \begin{tabular}{l p{9cm}}
        \hline\textbf{Action}
        & Gets the BAT's entry number of a specified filename.\\
        \hline\textbf{Entry} & \texttt{HL} = Address where the filename to 
        check is stored\\
        \hline\multirow[t]{2}{4em}{\textbf{Exit}}
        & BAT Entry values are stored in the \hyperref[sec:ram_memmap]{SYSVARS}.\\
        & \texttt{DE} = \$0000 if filename found. Otherwise, whatever value
        had at start.\\
        \hline\textbf{Destroys} & \texttt{A}, \texttt{B}, \texttt{DE}, \texttt{HL},
        \texttt{tmp\_byte}, \texttt{tmp\_addr2}, \texttt{tmp\_addr3} \\
        \hline\multirow[t]{4}{4em}{\textbf{Calls}}
        & \hyperref[func:fkrndzfssectobuffer]{F\_KRN\_DZFS\_SEC\_TO\_BUFFER}\\
        & \hyperref[func:fkrndzfsbatentrytobuffer]{F\_KRN\_DZFS\_BATENTRY\_TO\_BUFFER}\\
        & \hyperref[func:fkrnstrlenmax]{F\_KRN\_STRLENMAX}\\
        & \hyperref[func:fkrnstrcmp]{F\_KRN\_STRCMP}\\
        \hline
    \end{tabular}

    % ==========================================================================
    \subsubsection{F\_KRN\_DZFS\_LOAD\_FILE\_TO\_RAM}
    \label{func:fkrndzfsloadfiletoram}
    % ==========================================================================
    \begin{tabular}{l p{9cm}}
        \hline\textbf{Action}
        & Load a file from \textbf{DISK}. Copies the bytes stored in the
        \textbf{DISK} into \textbf{MEMORY}. If \hyperref[sec:ram_memmap]{SYSVARS}
        \textit{DISK\_loadsave\_addr} is not zero, then loads file to this
        address. If zero, then if \hyperref[sec:ram_memmap]{SYSVARS}
        \textit{DISK\_cur\_file\_load\_addr} is not zero, then loads file to
        this address. If also zero, then loads file to start of
        \hyperref[subsec:memmap:ram]{Free RAM}.\\
        \hline\multirow[t]{2}{4em}{\textbf{Entry}}
        & \texttt{DE} = 1st sector number in the \textbf{DISK}.\\
        & \texttt{IX} = file length in sectors.\\
        \hline\textbf{Exit} & None\\
        \hline\textbf{Destroys} & \texttt{BC}, \texttt{DE}, \texttt{HL},
        \texttt{IX}, \texttt{tmp\_addr1}\\
        \hline\textbf{Calls}
        & \hyperref[func:fbiosdiskreadsec]{F\_BIOS\_SD\_READ\_SEC}\\
        \hline
    \end{tabular}

    % ==========================================================================
    \subsubsection{F\_KRN\_DZFS\_DELETE\_FILE}
    \label{func:fkrndzfsdeletefile}
    % ==========================================================================
    \begin{tabular}{l p{9cm}}
        \hline\textbf{Action}
        & Marks a file as deleted. The mark is done by changing the first
        character of the filename to \texttt{0x7E} (\textasciitilde~)\\
        \hline\textbf{Entry}
        & \texttt{DE} = BAT Entry number.\\
        \hline\textbf{Exit} & None\\
        \hline\textbf{Destroys} & \texttt{A}, \texttt{DE}, \texttt{HL},\\
        \hline\multirow[t]{2}{4em}{\textbf{Calls}}
        & \hyperref[func:fkrnmultiply816slow]{F\_KRN\_MULTIPLY816\_SLOW}\\
        & \hyperref[func:fkrndzfssectortodisk]{F\_KRN\_DZFS\_SECTOR\_TO\_SD}\\
        \hline
    \end{tabular}

    % ==========================================================================
    \subsubsection{F\_KRN\_DZFS\_CHGATTR\_FILE}
    \label{func:fkrndzfschgattrfile}
    % ==========================================================================
    \begin{tabular}{l p{9cm}}
        \hline\textbf{Action}
        & Changes the attributes (RHSE) of a file.\\
        \hline\multirow[t]{2}{4em}{\textbf{Entry}}
        & \texttt{DE} = BAT Entry number.\\
        & \texttt{A} = attributes mask byte.\\
        \hline\textbf{Exit} & None\\
        \hline\textbf{Destroys} & \texttt{DE}, \texttt{HL},\\
        \hline\multirow[t]{2}{4em}{\textbf{Calls}}
        & \hyperref[func:fkrnmultiply816slow]{F\_KRN\_MULTIPLY816\_SLOW}\\
        & \hyperref[func:fkrndzfssectortodisk]{F\_KRN\_DZFS\_SECTOR\_TO\_SD}\\
        \hline
    \end{tabular}

    % ==========================================================================
    \subsubsection{F\_KRN\_DZFS\_RENAME\_FILE}
    \label{func:fkrndzfsrenamefile}
    % ==========================================================================
    \begin{tabular}{l p{9cm}}
        \hline\textbf{Action}
        & Changes the name of a file.\\
        \hline\multirow[t]{2}{4em}{\textbf{Entry}}
        & \texttt{IY} = \textbf{MEMORY} address where the new filename is
        stored.\\
        & \texttt{DE} = BAT Entry number.\\
        \hline\textbf{Exit} & None\\
        \hline\textbf{Destroys} & \texttt{A}, \texttt{BC}, \texttt{DE},
        \texttt{HL}, \texttt{IY}\\
        \hline\multirow[t]{2}{4em}{\textbf{Calls}}
        & \hyperref[func:fkrnmultiply816slow]{F\_KRN\_MULTIPLY816\_SLOW}\\
        & \hyperref[func:fkrndzfssectortodisk]{F\_KRN\_DZFS\_SECTOR\_TO\_SD}\\
        \hline
    \end{tabular}

    % ==========================================================================
    \subsubsection{F\_KRN\_DZFS\_FORMAT\_DISK}
    \label{func:fkrndzfsformatdisk}
    % ==========================================================================
    \begin{tabular}{l p{9cm}}
        \hline\textbf{Action}
        & Formats a \textbf{DISK} with DZFS.\\
        \hline\textbf{Entry}
        & \texttt{HL} = \textbf{MEMORY} address where the disk label is
        stored.\\
        \hline\textbf{Exit} & None\\
        \hline\textbf{Destroys} & \texttt{A}, \texttt{BC}, \texttt{DE},
        \texttt{HL}, \texttt{IX}, \texttt{IY}, \texttt{tmp\_addr1},
        \texttt{tmp\_byte}\\
        \hline\multirow[t]{12}{4em}{\textbf{Calls}}
        & \hyperref[func:fkrnserialwrstr]{F\_KRN\_SERIAL\_WRSTR}\\
        & \hyperref[func:fkrndzfscalcsn]{F\_KRN\_DZFS\_CALC\_SN}\\
        & \hyperref[func:fkrnrtcgetdate]{F\_KRN\_RTC\_GET\_DATE}\\
        & \hyperref[func:fbiosrtcgettime]{F\_BIOS\_RTC\_GET\_TIME}\\
        & \hyperref[func:fkrnbcdtoascii]{F\_KRN\_BCD\_TO\_ASCII}\\
        & \hyperref[func:fkrnbintobcd4]{F\_KRN\_BIN\_TO\_BCD4}\\
        & \hyperref[func:fkrnbintobcd6]{F\_KRN\_BIN\_TO\_BCD6}\\
        & \hyperref[func:fkrndzfssectortodisk]{F\_KRN\_DZFS\_SECTOR\_TO\_SD}\\
        & \hyperref[func:fkrnsetmemrng]{F\_KRN\_SETMEMRNG}\\
        & \hyperref[func:fbiosserialconouta]{F\_BIOS\_SERIAL\_CONOUT\_A}\\
        & \hyperref[func:fbiossdparkdisks]{F\_BIOS\_SD\_PARK\_DISKS}\\
        & \hyperref[func:fbiossdmountdisks]{F\_BIOS\_SD\_MOUNT\_DISKS}\\
        \hline
    \end{tabular}

    % ==========================================================================
    \subsubsection{F\_KRN\_DZFS\_CALC\_SN}
    \label{func:fkrndzfscalcsn}
    % ==========================================================================
    \begin{tabular}{l p{9cm}}
        \hline\textbf{Action}
        & Calculates the Serial Number (4 bytes) for a \textbf{DISK}.\\
        \hline\textbf{Entry}
        & \texttt{IX} = \textbf{MEMORY} address where the serial number will
        be stored.\\
        \hline\textbf{Exit} & None\\
        \hline\textbf{Destroys} & \texttt{A}, \texttt{BC}, \texttt{DE},
        \texttt{HL}, \texttt{IX} \\
        \hline\multirow[t]{3}{4em}{\textbf{Calls}}
        & \hyperref[func:fbiosrtcgetdate]{F\_BIOS\_RTC\_GET\_DATE}\\
        & \hyperref[func:fbiosrtcgettime]{F\_BIOS\_RTC\_GET\_TIME}\\
        & \hyperref[func:fkrnmultiply816slow]{F\_KRN\_MULTIPLY816\_SLOW}\\
        \hline
    \end{tabular}

    % ==========================================================================
    \subsubsection{F\_KRN\_DZFS\_SECTOR\_TO\_DISK}
    \label{func:fkrndzfssectortodisk}
    % ==========================================================================
    \begin{tabular}{l p{9cm}}
        \hline\textbf{Action}
        & Calls the \textbf{BIOS} subroutine that will store the data 
        (512 bytes) currently in \texttt{DISK Buffer} in \textbf{MEMORY},
        to the \textbf{DISK}.\\
        \hline\textbf{Entry}
        & \texttt{DISK\_cur\_sector} = the sector number in the \textbf{DISK}
        that will be written.\\
        \hline\textbf{Exit} & None\\
        \hline\textbf{Destroys} & \texttt{BC}, \texttt{DE}\\
        \hline\textbf{Calls}
        & \hyperref[func:fbiosdiskwritesec]{F\_BIOS\_SD\_WRITE\_SEC}\\
        \hline
    \end{tabular}

    % ==========================================================================
    \subsubsection{F\_KRN\_DZFS\_GET\_BAT\_FREE\_ENTRY}
    \label{func:fkrndzfsgetbatfreeentry}
    % ==========================================================================
    \begin{tabular}{l p{9cm}}
        \hline\textbf{Action}
        & Get number of available BAT entry.\\
        \hline\textbf{Entry} & None\\
        \hline\textbf{Exit} & \texttt{DISK\_cur\_file\_entry\_number} =
        entry number.\\
        \hline\textbf{Destroys} & \texttt{A}, \texttt{IY},
        \texttt{CF\_cur\_sector}, \texttt{CF\_cur\_file\_entry\_number}\\
        \hline\multirow[t]{2}{4em}{\textbf{Calls}}
        & \hyperref[func:fkrndzfsreadbatsector]{F\_KRN\_DZFS\_READ\_BAT\_SECTOR}\\
        & \hyperref[func:fkrndzfsbatentrytobuffer]{F\_KRN\_DZFS\_BATENTRY\_TO\_BUFFER}\\
        \hline
    \end{tabular}

    % ==========================================================================
    \subsubsection{F\_KRN\_DZFS\_ADD\_BAT\_ENTRY}
    \label{func:fkrndzfsaddbatentry}
    % ==========================================================================
    \begin{tabular}{l p{9cm}}
        \hline\textbf{Action}
        & Adds a BAT entry into the \textbf{DISK}.\\
        \hline\multirow[t]{4}{4em}{\textbf{Entry}}
        & \texttt{DE} = BAT entry number.\\
        & \texttt{DISK\_cur\_sector} = Sector number where the BAT Entry is
        in the \textbf{DISK}.\\
        & \texttt{DISK\_BUFFER\_START} = Sector (512 bytes) containing the
        BAT where the entry is.\\
        & \texttt{DISK BAT} = BAT Entry data that will be saved to 
        \textbf{DISK}.\\
        \hline\textbf{Exit} & None\\
        \hline\textbf{Destroys} & \texttt{A}, \texttt{BC}, \texttt{DE}, 
        \texttt{HL}\\
        \hline\textbf{Calls}
        & \hyperref[func:fkrnmultiply816slow]{F\_KRN\_MULTIPLY816\_SLOW}\\
        \hline
    \end{tabular}

    % ==========================================================================
    \subsubsection{F\_KRN\_DZFS\_CREATE\_NEW\_FILE}
    \label{func:fkrndzfscreatenewfile}
    % ==========================================================================
    \begin{tabular}{l p{9cm}}
        \hline\textbf{Action}
        & Creates a new file (and its corresponding BAT Entry) in the 
        \textbf{DISK}, from bytes stored in \textbf{MEMORY}.\\
        \hline\multirow[t]{3}{4em}{\textbf{Entry}}
        & \texttt{HL} = \textbf{MEMORY} address of the first byte to be
        stored.\\
        & \texttt{BC} = number of bytes to be stored in the \textbf{DISK}.\\
        & \texttt{IX} = \textbf{MEMORY} address where the filename is
        stored.\\
        \hline\textbf{Exit} & None\\
        \hline\textbf{Destroys} & \texttt{A}, \texttt{BC}, \texttt{DE}, 
        \texttt{HL}, \texttt{IX}, \texttt{tmp\_addr1}, \texttt{tmp\_addr2},
        \texttt{tmp\_addr3}, \texttt{tmp\_byte}\\
        \hline\multirow[t]{13}{4em}{\textbf{Calls}}
        & \hyperref[func:fkrndzfsgetbatfreeentry]{F\_KRN\_DZFS\_GET\_BAT\_FREE\_ENTRY}\\
        & \hyperref[func:fkrndiv1616]{F\_KRN\_DIV1616}\\
        & \hyperref[func:fkrnmultiply1616]{F\_KRN\_MULTIPLY1616}\\
        & \hyperref[func:fkrncopymem512]{F\_KRN\_COPYMEM512}\\
        & \hyperref[func:fkrnclearmemarea]{F\_KRN\_CLEAR\_MEMAREA}\\
        & \hyperref[func:fkrncleardiskbuffer]{F\_KRN\_CLEAR\_DISKBUFFER}\\
        & \hyperref[func:fkrndzfssectortodisk]{F\_KRN\_DZFS\_SECTOR\_TO\_SD}\\
        & \hyperref[func:fbiossdbusywait]{F\_BIOS\_SD\_BUSY\_WAIT}\\
        & \hyperref[func:fkrnserialwrstrclr]{F\_KRN\_SERIAL\_WRSTRCLR}\\
        & \hyperref[func:fkrndzfscalcfiletime]{F\_KRN\_DZFS\_CALC\_FILETIME}\\
        & \hyperref[func:fkrndzfscalcfiledate]{F\_KRN\_DZFS\_CALC\_FILEDATE}\\
        & \hyperref[func:fkrndzfssectobuffer]{F\_KRN\_DZFS\_SEC\_TO\_BUFFER}\\
        & \hyperref[func:fkrndzfssectobuffer]{F\_KRN\_DZFS\_ADD\_BAT\_ENTRY}\\
        \hline
    \end{tabular}

    % ==========================================================================
    \subsubsection{F\_KRN\_DZFS\_CALC\_FILETIME}
    \label{func:fkrndzfscalcfiletime}
    % ==========================================================================
    \begin{tabular}{l p{9cm}}
        \hline\multirow[t]{2}{4em}{\textbf{Action}}
        & Packs current Real-Time Clock time into two bytes, which is the
        format used to store times (created/modified) for files in the 
        \textbf{DISK}.\\
        & The formula used is: $2048 * hours + 32 * minutes + seconds / 2$\\
        \hline\textbf{Entry} & None\\
        \hline\textbf{Exit} & \texttt{HL} = RTC time\\
        \hline\textbf{Destroys} & \texttt{A}, \texttt{DE}, \texttt{HL}\\
        \hline\textbf{Calls}
        & \hyperref[func:fbiosrtcgettime]{F\_BIOS\_RTC\_GET\_TIME}\\
        \hline
    \end{tabular}

    % ==========================================================================
    \subsubsection{F\_KRN\_DZFS\_CALC\_FILEDATE}
    \label{func:fkrndzfscalcfiledate}
    % ==========================================================================
    \begin{tabular}{l p{9cm}}
        \hline\multirow[t]{2}{4em}{\textbf{Action}}
        & Packs current Real-Time Clock date into two bytes, which is the
        format used to store dates (created/modified) for files in the 
        \textbf{DISK}.\\
        & The formula used is: $512 * (year - 2000) + month * 32 + day$\\
        \hline\textbf{Entry} & None\\
        \hline\textbf{Exit} & \texttt{HL} = RTC date\\
        \hline\textbf{Destroys} & \texttt{A}, \texttt{DE}, \texttt{HL}\\
        \hline\textbf{Calls}
        & \hyperref[func:fbiosrtcgetdate]{F\_BIOS\_RTC\_GET\_DATE}\\
        \hline
    \end{tabular}

    % ==========================================================================
    \subsubsection{F\_KRN\_DZFS\_SHOW\_DISKINFO\_SHORT}
    \label{func:fkrndzfsshowdiskinfoshort}
    % ==========================================================================
    \begin{tabular}{l p{9cm}}
        \hline\textbf{Action}
        & Outputs to the \textbf{CONSOLE} some information of the 
        \textbf{DISK}: volume label, serial number, date/time creation.\\
        \hline\textbf{Entry} & None\\
        \hline\textbf{Exit} & None\\
        \hline\textbf{Destroys} & \texttt{A}, \texttt{BC}, \texttt{DE}, 
        \texttt{HL}\\
        \hline\multirow[t]{5}{4em}{\textbf{Calls}}
        & \hyperref[func:fkrnserialwrstrclr]{F\_KRN\_SERIAL\_WRSTRCLR}\\
        & \hyperref[func:fkrnserialprnbyte]{F\_KRN\_SERIAL\_PRN\_BYTE}\\
        & \hyperref[func:fkrnserialprnbytes]{F\_KRN\_SERIAL\_PRN\_BYTES}\\
        & \hyperref[func:fbiosserialconouta]{F\_BIOS\_SERIAL\_CONOUT\_A}\\
        & \hyperref[func:fkrnserialemptylines]{F\_KRN\_SERIAL\_EMPTYLINES}\\
        \hline
    \end{tabular}

    % ==========================================================================
    \subsubsection{F\_KRN\_DZFS\_SHOW\_DISKINFO}
    \label{func:fkrndzfsshowdiskinfo}
    % ==========================================================================
    \begin{tabular}{l p{9cm}}
        \hline\textbf{Action}
        & Outputs to the \textbf{CONSOLE} all information of the 
        \textbf{DISK}: volume label, serial number, date/time creation,
        file system ID, number of partitions, number of bytes per sector,
        number of sectors per block.\\
        \hline\textbf{Entry} & None\\
        \hline\textbf{Exit} & None\\
        \hline\textbf{Destroys} & \texttt{A}, \texttt{BC}, \texttt{DE}, 
        \texttt{HL}, \texttt{tmp\_addr1}\\
        \hline\multirow[t]{5}{4em}{\textbf{Calls}}
        & \hyperref[func:fkrndzfsshowdiskinfoshort]{F\_KRN\_DZFS\_SHOW\_DISKINFO\_SHORT}\\
        & \hyperref[func:fkrnserialwrstrclr]{F\_KRN\_SERIAL\_WRSTRCLR}\\
        & \hyperref[func:fkrnserialprnbyte]{F\_KRN\_SERIAL\_PRN\_BYTE}\\
        & \hyperref[func:fkrnserialprnbytes]{F\_KRN\_SERIAL\_PRN\_BYTES}\\
        & \hyperref[func:fbiosserialconouta]{F\_BIOS\_SERIAL\_CONOUT\_A}\\
        & \hyperref[func:fkrnserialemptylines]{F\_KRN\_SERIAL\_EMPTYLINES}\\
        \hline
    \end{tabular}

    % ==========================================================================
    \subsubsection{F\_KRN\_DZFS\_CHECK\_FILE\_EXISTS}
    \label{func:fkrndzfscheckfileexists}
    % ==========================================================================
    \begin{tabular}{l p{9cm}}
        \hline\textbf{Action}
        & Checks if a specified filename exists in the \textbf{DISK}. The
        filename MUST be terminated by a zero.\\
        \hline\textbf{Entry} & \texttt{HL} = \textbf{MEMORY} address where
        the filename to check is stored.\\
        \hline\textbf{Exit} & \texttt{Z Flag} set if filename is not found.\\
        \hline\textbf{Destroys} & \texttt{A}, \texttt{DE}, \texttt{tmp\_addr3}\\
        \hline\textbf{Calls}
        & \hyperref[func:fkrndzfsgetfilebatentry]{F\_KRN\_DZFS\_GET\_FILE\_BATENTRY}\\
        \hline
    \end{tabular}