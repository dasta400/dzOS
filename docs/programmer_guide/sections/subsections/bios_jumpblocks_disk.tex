    % ==========================================================================
    \subsection{DISK Routines}
    % ==========================================================================

        % ==========================================================================
        \subsubsection{F\_BIOS\_SD\_BUSY\_WAIT}
        \label{func:fbiossdbusywait}
        % ==========================================================================
        \begin{tabular}{l p{15cm}}
            \hline\textbf{Action}
            & Calls \textbf{ASMDC} to check if the \textbf{DISK} is busy, and
            loops until it is not busy. \\
            \hline\textbf{Entry} & None \\
            \hline\textbf{Exit} & None \\
            \hline\textbf{Destroys} & \texttt{A} \\
            \hline\multirow[t]{2}{4em}{\textbf{Calls}}
            & \hyperref[func:fbiosserialconoutb]{F\_BIOS\_SERIAL\_CONOUT\_B}\\
            & \hyperref[func:fbiosserialconinb]{F\_BIOS\_SERIAL\_CONIN\_B}\\
            \hline
        \end{tabular}

        % ==========================================================================
        \subsubsection{F\_BIOS\_SD\_GET\_STATUS}
        \label{func:fbiossdgetstatus}
        % ==========================================================================
        \begin{tabular}{l p{15cm}}
            \hline\textbf{Action}
            & Calls \textbf{ASMDC} to check the status of the SD Card module.\\
            \hline\textbf{Entry} & None \\
            \hline\multirow[t]{4}{4em}{\textbf{Exit}}
            & \textit{SD\_status} \\
            & bit 0 = set if SD card was not found \\
            & bit 1 = set if image file was not found \\
            & bit 2 = set if last command resulted in error \\
            \hline\textbf{Destroys} & \texttt{A} \\
            \hline\multirow[t]{3}{4em}{\textbf{Calls}}
            & \hyperref[func:fbiossdbusywait]{F\_BIOS\_SD\_BUSY}\\
            & \hyperref[func:fbiosserialconoutb]{F\_BIOS\_SERIAL\_CONOUT\_B}\\
            & \hyperref[func:fbiosserialconinb]{F\_BIOS\_SERIAL\_CONIN\_B}\\
            \hline
        \end{tabular}

        % ==========================================================================
        \subsubsection{F\_BIOS\_SD\_PARK\_DISKS}
        \label{func:fbiossdparkdisks}
        % ==========================================================================
        \begin{tabular}{l p{15cm}}
            \hline\textbf{Action}
            & Tells \textbf{ASMDC} to close the Image File\\
            \hline\textbf{Entry} & None \\
            \hline\textbf{Exit} & None \\
            \hline\textbf{Destroys} & \texttt{A} \\
            \hline\multirow[t]{2}{4em}{\textbf{Calls}} 
            & \hyperref[func:fbiossdbusywait]{F\_BIOS\_SD\_BUSY}\\
            & \hyperref[func:fbiosserialconoutb]{F\_BIOS\_SERIAL\_CONOUT\_B}\\
            \hline
        \end{tabular}

        % ==========================================================================
        \subsubsection{F\_BIOS\_SD\_MOUNT\_DISKS}
        \label{func:fbiossdmountdisks}
        % ==========================================================================
        \begin{tabular}{l p{15cm}}
            \hline\textbf{Action}
            & Tells \textbf{ASMDC} to open the Image File\\
            \hline\textbf{Entry} & None \\
            \hline\textbf{Exit} & None \\
            \hline\textbf{Destroys} & \texttt{A} \\
            \hline\multirow[t]{2}{4em}{\textbf{Calls}} 
            & \hyperref[func:fbiossdbusywait]{F\_BIOS\_SD\_BUSY}\\
            & \hyperref[func:fbiosserialconoutb]{F\_BIOS\_SERIAL\_CONOUT\_B}\\
            \hline
        \end{tabular}

        % ==========================================================================
        \subsubsection{F\_BIOS\_DISK\_READ\_SEC}
        \label{func:fbiosdiskreadsec}
        % ==========================================================================
        \begin{tabular}{l p{15cm}}
            \hline\textbf{Action}
            & Reads a Sector (512 bytes), from the \textbf{DISK} and places the bytes into the \texttt{CF\_BUFFER\_START} \\
            \hline\multirow[t]{5}{4em}{\textbf{Entry}}
            & \texttt{E} = sector address LBA 0 (bits 0-7) \\
            & \texttt{D} = sector address LBA 1 (bits 8-15) \\
            & \texttt{C} = sector address LBA 2 (bits 16-23) \\
            & \texttt{B} = sector address LBA 3 (bits 24-27) \\
            & BC are not used (set to zero), because max sector is 65,535\\
            \hline\textbf{Exit} & \texttt{CF\_BUFFER\_START} contains the 512 bytes read \\
            \hline\textbf{Destroys} & \texttt{A}, \texttt{B}, \texttt{HL}, \texttt{DISK\_BUFFER\_START} \\
            \hline\multirow[t]{3}{4em}{\textbf{Calls}}
            & \hyperref[func:fbiossdbusywait]{F\_BIOS\_SD\_BUSY}\\
            & \hyperref[func:fbiosserialconoutb]{F\_BIOS\_SERIAL\_CONOUT\_B}\\
            & \hyperref[func:fbiosserialconinb]{F\_BIOS\_SERIAL\_CONIN\_B}\\
            \hline
        \end{tabular}

        % ==========================================================================
        \subsubsection{F\_BIOS\_DISK\_WRITE\_SEC}
        \label{func:fbiosdiskwritesec}
        % ==========================================================================
        \begin{tabular}{l p{15cm}}
            \hline\textbf{Action}
            & Writes a Sector (512 bytes), from the \texttt{DISK\_BUFFER\_START}
            into the \textbf{DISK}\\
            \hline\multirow[t]{5}{4em}{\textbf{Entry}}
            & \texttt{E} = sector address LBA 0 (bits 0-7) \\
            & \texttt{D} = sector address LBA 1 (bits 8-15) \\
            & \texttt{C} = sector address LBA 2 (bits 16-23) \\
            & \texttt{B} = sector address LBA 3 (bits 24-27) \\
            & BC are not used (set to zero), because max sector is 65,535\\
            \hline\textbf{Exit} & \texttt{DISK\_BUFFER\_START} contains the 512 bytes written \\
            \hline\textbf{Destroys} & \texttt{A}, \texttt{HL}, \texttt{DISK\_BUFFER\_START} \\
            \hline\multirow[t]{3}{4em}{\textbf{Calls}}
            & \hyperref[func:fbiossdbusywait]{F\_BIOS\_SD\_BUSY}\\
            & \hyperref[func:fbiosserialconoutb]{F\_BIOS\_SERIAL\_CONOUT\_B}\\
            & \hyperref[func:fbiosserialconinb]{F\_BIOS\_SERIAL\_CONIN\_B}\\
            \hline
        \end{tabular}

        % ==========================================================================
        \subsubsection{F\_BIOS\_FDD\_GET\_STATUS}
        \label{func:fbiosfddgetstatus}
        % ==========================================================================
        \begin{tabular}{l p{15cm}}
            \hline\textbf{Action}
            & Calls \textbf{ASMDC} to check the status of the Floppy Disk Drive.\\
            \hline\textbf{Entry} & None \\
            \hline\multirow[t]{4}{4em}{\textbf{Exit}}
            & \textit{SD\_status} \\
            & bit 0 = set if FDD was not detected \\
            & bit 1 = Not used \\
            & bit 2 = set if last command resulted in error \\
            \hline\textbf{Destroys} & \texttt{A} \\
            \hline\multirow[t]{2}{4em}{\textbf{Calls}}
            & \hyperref[func:fbiosserialconoutb]{F\_BIOS\_SERIAL\_CONOUT\_B}\\
            & \hyperref[func:fbiosserialconinb]{F\_BIOS\_SERIAL\_CONIN\_B}\\
            \hline
        \end{tabular}

        % ==========================================================================
        \subsubsection{F\_BIOS\_FDD\_BUSY\_WAIT}
        \label{func:fbiosfddbusywait}
        % ==========================================================================
        \begin{tabular}{l p{15cm}}
            \hline\textbf{Action}
            & Calls \textbf{ASMDC} to check if the \textbf{FDD} is busy, and
            loops until it is not busy. \\
            \hline\textbf{Entry} & None \\
            \hline\textbf{Exit} & None \\
            \hline\textbf{Destroys} & \texttt{A} \\
            \hline\multirow[t]{2}{4em}{\textbf{Calls}}
            & \hyperref[func:fbiosserialconoutb]{F\_BIOS\_SERIAL\_CONOUT\_B}\\
            & \hyperref[func:fbiosserialconinb]{F\_BIOS\_SERIAL\_CONIN\_B}\\
            \hline
        \end{tabular}

        % ==========================================================================
        \subsubsection{F\_BIOS\_FDD\_CHANGE}
        \label{func:fbiosfddchange}
        % ==========================================================================
        \begin{tabular}{l p{15cm}}
            \hline\textbf{Action}
            & Tells the \textbf{ASMDC} that the current \textbf{DISK} for
            operations is now the \textbf{FDD}. \\
            \hline\textbf{Entry} & None \\
            \hline\textbf{Exit} & \texttt{DISK\_status} is updated\\
            \hline\textbf{Destroys} & \texttt{A} \\
            \hline\textbf{Calls}
            & \hyperref[func:fbiosserialconoutb]{F\_BIOS\_SERIAL\_CONOUT\_B}\\
            \hline
        \end{tabular}

        % ==========================================================================
        \subsubsection{F\_BIOS\_FDD\_LOWLVL\_FORMAT}
        \label{func:fbiosfddlowlvlformat}
        % ==========================================================================
        \begin{tabular}{l p{15cm}}
            \hline\textbf{Action}
            & Tells the \textbf{ASMDC} to low-level format a \textbf{DISK} in
            the \textbf{FDD}. This function does not set up any file system. It
            just fills with \texttt{0xF6} all bytes of all sectors.\\
            \hline\textbf{Entry} & None \\
            \hline\textbf{Exit} & \texttt{A} = \texttt{0x00} if everything OK. 
            Bit 2 set if command resulted in error.\\
            \hline\textbf{Destroys} & \texttt{A} \\
            \hline\multirow[t]{2}{4em}{\textbf{Calls}}
            & \hyperref[func:fbiosserialconoutb]{F\_BIOS\_SERIAL\_CONOUT\_B}\\
            & \hyperref[func:fbiosserialconinb]{F\_BIOS\_SERIAL\_CONIN\_B}\\
            \hline
        \end{tabular}

        % ==========================================================================
        \subsubsection{F\_BIOS\_FDD\_MOTOR\_ON}
        \label{func:fbiosfddmotoron}
        % ==========================================================================
        \begin{tabular}{l p{15cm}}
            \hline\textbf{Action}
            & Tells the \textbf{ASMDC} to switch the \textbf{FDD} motor on. It
            is a recommended practice to switch the motor on and off manually
            if multiple sectors are to read or written.\\
            \hline\textbf{Entry} & None \\
            \hline\textbf{Exit} & None \\
            \hline\textbf{Destroys} & \texttt{A} \\
            \hline\textbf{Calls}
            & \hyperref[func:fbiosserialconoutb]{F\_BIOS\_SERIAL\_CONOUT\_B}\\
            \hline
        \end{tabular}

        % ==========================================================================
        \subsubsection{F\_BIOS\_FDD\_MOTOR\_OFF}
        \label{func:fbiosfddmotoroff}
        % ==========================================================================
        \begin{tabular}{l p{15cm}}
            \hline\textbf{Action}
            & Tells the \textbf{ASMDC} to switch the \textbf{FDD} motor off. It
            is a recommended practice to switch the motor on and off manually
            if multiple sectors are to read or written.\\
            \hline\textbf{Entry} & None \\
            \hline\textbf{Exit} & None \\
            \hline\textbf{Destroys} & \texttt{A} \\
            \hline\textbf{Calls}
            & \hyperref[func:fbiosserialconoutb]{F\_BIOS\_SERIAL\_CONOUT\_B}\\
            \hline
        \end{tabular}

        % ==========================================================================
        \subsubsection{F\_BIOS\_FDD\_CHECK\_DISKIN}
        \label{func:fbiosfddcheckdiskin}
        % ==========================================================================
        \begin{tabular}{l p{15cm}}
            \hline\textbf{Action}
            & Asks the \textbf{ASMDC} to check if a Floppy Disk is inside the
            \textbf{FDD}.\\
            \hline\textbf{Entry} & None \\
            \hline\textbf{Exit} & \texttt{A} = \texttt{0x00} yes / \texttt{0xFF}
            no \\
            \hline\textbf{Destroys} & \texttt{A} \\
            \hline\multirow[t]{2}{4em}{\textbf{Calls}}
            & \hyperref[func:fbiosserialconoutb]{F\_BIOS\_SERIAL\_CONOUT\_B}\\
            & \hyperref[func:fbiosserialconinb]{F\_BIOS\_SERIAL\_CONIN\_B}\\
            \hline
        \end{tabular}

        % ==========================================================================
        \subsubsection{F\_BIOS\_FDD\_CHECK\_WPROTECT}
        \label{func:fbiosfddcheckwprotect}
        % ==========================================================================
        \begin{tabular}{l p{15cm}}
            \hline\textbf{Action}
            & Asks the \textbf{ASMDC} to check if the Floppy Disk is write
            protected.\\
            \hline\textbf{Entry} & None \\
            \hline\textbf{Exit} & \texttt{A} = \texttt{0x00} yes / \texttt{0xFF}
            no \\
            \hline\textbf{Destroys} & \texttt{A} \\
            \hline\multirow[t]{2}{4em}{\textbf{Calls}}
            & \hyperref[func:fbiosserialconoutb]{F\_BIOS\_SERIAL\_CONOUT\_B}\\
            & \hyperref[func:fbiosserialconinb]{F\_BIOS\_SERIAL\_CONIN\_B}\\
            \hline
        \end{tabular}
        