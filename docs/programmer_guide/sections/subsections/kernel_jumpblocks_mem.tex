% ==========================================================================
\subsection{MEMORY Routines}
% ==========================================================================

    % ==========================================================================
    \subsubsection{F\_KRN\_SETMEMRNG}
    \label{func:fkrnsetmemrng}
    % ==========================================================================
    \begin{tabular}{l p{15cm}}
        \hline\textbf{Action}
        & Sets (changes) a value in a \textbf{MEMORY} position range.\\
        \hline\multirow[t]{3}{4em}{\textbf{Entry}}
        & \texttt{HL} = \textbf{MEMORY} start position (first byte).\\
        & \texttt{BC} = number of bytes to set.\\
        & \texttt{A} = value to set.\\
        \hline\textbf{Exit} & None\\
        \hline\textbf{Destroys} & \texttt{BC}, \texttt{HL}\\
        \hline\textbf{Calls} & None\\
        \hline
    \end{tabular}

    % ==========================================================================
    \subsubsection{F\_KRN\_COPYMEM512}
    \label{func:fkrncopymem512}
    % ==========================================================================
    \begin{tabular}{l p{15cm}}
        \hline\textbf{Action}
        & Copies bytes from one area of \textbf{MEMORY} to another, in group
        of 512 bytes (i.e. max. 512 bytes). If less than 512 bytes are to be
        copied, the rest will be filled with zeros.\\
        \hline\multirow[t]{3}{4em}{\textbf{Entry}}
        & \texttt{HL} = \textbf{MEMORY} origin position (from where to copy
        the bytes).\\
        & \texttt{DE} = \textbf{MEMORY} destination position (to where to
        copy the bytes).\\
        & \texttt{BC} = number of bytes to copy (MUST be less or equal to
        512).\\
        \hline\textbf{Exit} & None\\
        \hline\textbf{Destroys} & \texttt{A}, \texttt{BC}, \texttt{DE},
        \texttt{HL}\\
        \hline\textbf{Calls} & None\\
        \hline
    \end{tabular}

    % ==========================================================================
    \subsubsection{F\_KRN\_SHIFT\_BYTES\_BY1}
    \label{func:fkrnshiftbytesby1}
    % ==========================================================================
    \begin{tabular}{l p{15cm}}
        \hline\textbf{Action}
        & Moves bytes (by one) to the right and replaces first byte with
        bytes counter.\\
        \hline\multirow[t]{2}{4em}{\textbf{Entry}}
        & \texttt{HL} = \textbf{MEMORY} address of last byte to move.\\
        & \texttt{BC} = number of bytes to move.\\
        \hline\textbf{Exit} & None\\
        \hline\textbf{Destroys} & \texttt{A}, \texttt{DE}, \texttt{HL}\\
        \hline\textbf{Calls} & None\\
        \hline
    \end{tabular}

    % ==========================================================================
    \subsubsection{F\_KRN\_CLEAR\_MEMAREA}
    \label{func:fkrnclearmemarea}
    % ==========================================================================
    \begin{tabular}{l p{15cm}}
        \hline\textbf{Action}
        & Clears (with zeros) a number of bytes, starting at a specified
        \textbf{MEMORY} address. Maximum 256 bytes can be cleared.\\
        \hline\multirow[t]{2}{4em}{\textbf{Entry}}
        & \texttt{IX} = \textbf{MEMORY} address of first byte to clear.\\
        & \texttt{B} = number of bytes to clear.\\
        \hline\textbf{Exit} & None\\
        \hline\textbf{Destroys} & \texttt{A}, \texttt{BC}, \texttt{IX}\\
        \hline\textbf{Calls} & None\\
        \hline
    \end{tabular}

    % ==========================================================================
    \subsubsection{F\_KRN\_CLEAR\_DISKBUFFER}
    \label{func:fkrncleardiskbuffer}
    % ==========================================================================
    \begin{tabular}{l p{15cm}}
        \hline\textbf{Action}
        & Clears (with zeros) the \textbf{MEMORY} area of the \textbf{DISK}
        buffer.\\
        \hline\textbf{Entry} & None\\
        \hline\textbf{Exit} & None\\
        \hline\textbf{Destroys} & \texttt{BC}, \texttt{IX}\\
        \hline\textbf{Calls} & \hyperref[func:fkrnclearmemarea]{F\_KRN\_CLEAR\_MEMAREA}\\
        \hline
    \end{tabular}