% ==========================================================================
\subsection{Math Routines}
% ==========================================================================

    % ==========================================================================
    \subsubsection{F\_KRN\_MULTIPLY816\_SLOW}
    \label{func:fkrnmultiply816slow}
    % ==========================================================================
    \begin{tabular}{l p{15cm}}
        \hline\multirow[t]{2}{4em}{\textbf{Action}}
        & Multiplies an 8-bit number by a 16-bit number (HL = A * DE).\\
        & It does a slow multiplication by adding the multiplier to itself
        as many times as multiplicand (e.g. 8 * 4 = 8+8+8+8).\\
        \hline\multirow[t]{2}{4em}{\textbf{Entry}}
        & \texttt{A} = Multiplicand\\
        & \texttt{DE} = Multiplier\\
        \hline\textbf{Exit} & \texttt{HL} = Product \\
        \hline\textbf{Destroys} & \texttt{B}, \texttt{HL} \\
        \hline\textbf{Calls} & None\\
        \hline
    \end{tabular}

    % ==========================================================================
    \subsubsection{F\_KRN\_MULTIPLY1616}
    \label{func:fkrnmultiply1616}
    % ==========================================================================
    \begin{tabular}{l p{15cm}}
        \hline\textbf{Action}
        & Multiplies two 16-bit numbers (HL = HL * DE)\\
        \hline\multirow[t]{2}{4em}{\textbf{Entry}}
        & \texttt{HL} = Multiplicand\\
        & \texttt{DE} = Multiplier\\
        \hline\textbf{Exit} & \texttt{HL} = Product \\
        \hline\textbf{Destroys} & \texttt{A}, \texttt{BC}, \texttt{DE}, \texttt{HL} \\
        \hline\textbf{Calls} & None\\
        \hline
    \end{tabular}

    % ==========================================================================
    \subsubsection{F\_KRN\_DIV1616}
    \label{func:fkrndiv1616}
    % ==========================================================================
    \begin{tabular}{l p{15cm}}
        \hline\textbf{Action}
        & Divides two 16-bit numbers (BC = BC / DE, HL = remainder)\\
        \hline\multirow[t]{2}{4em}{\textbf{Entry}}
        & \texttt{BC} = Dividend\\
        & \texttt{DE} = Divisor\\
        \hline\multirow[t]{2}{4em}{\textbf{Exit}}
        & \texttt{BC} = Quotient\\
        & \texttt{HL} = Remainder\\
        \hline\textbf{Destroys} & \texttt{A}, \texttt{BC}, \texttt{HL} \\
        \hline\textbf{Calls} & None\\
        \hline
    \end{tabular}

    % ==========================================================================
    \subsubsection{F\_KRN\_CRC16\_INI}
    \label{func:fkrncrc16ini}
    % ==========================================================================
    \begin{tabular}{l p{15cm}}
        \hline\textbf{Action}
        & Initialises the CRC to 0 and the polynomial to the appropriate bit
        pattern, to generate a CRC-16/BUYPASS1\\
        \hline\textbf{Entry} & None \\
        \hline\multirow[t]{2}{4em}{\textbf{Exit}}
        & \texttt{MATH\_CRC} = 0 (initial CRC value)\\
        & \texttt{MATH\_polynomial} = CRC polynomial\\
        \hline\textbf{Destroys} & \texttt{HL} \\
        \hline\textbf{Calls} & None\\
        \hline
    \end{tabular}

    CRC-16/BUYPASS1: A 16-bit cyclic redundancy check (CRC) based on the IBM
    Binary Synchronous Communications protocol (BSC or Bisync). It uses the
    polynomial $X^{16} + X^{15} +X^2 + 1$.

    % ==========================================================================
    \subsubsection{F\_KRN\_CRC16\_GEN}
    \label{func:fkrncrc16gen}
    % ==========================================================================
    \begin{tabular}{l p{15cm}}
        \hline\textbf{Action}
        & Combines the previous CRC with the CRC generated from the current
        data byte, to generate a CRC-16/BUYPASS1.\\
        \hline\multirow[t]{3}{4em}{\textbf{Entry}}
        & \texttt{A} = current data byte.\\
        & \texttt{MATH\_CRC} = previous CRC\\
        & \texttt{MATH\_polynomial} = CRC polynomial\\
        \hline\textbf{Exit}
        & \texttt{MATH\_CRC} = CRC with current data byte included\\
        \hline\textbf{Destroys} & \texttt{A}, \texttt{BC}, \texttt{DE}, \texttt{HL} \\
        \hline\textbf{Calls} & None\\
        \hline
    \end{tabular}