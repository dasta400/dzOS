% ==========================================================================
\section{dastaZ80 File System (DZFS)}
% ==========================================================================
In summary, a file system is a layer of abstraction to store, retrieve and
update a set of files.

A file system manages access to the data and the metadata of the files, and
manages the available space of the device, dividing the storage area into
units of storage and keeping a map of every storage unit of the device.

DZFS main goal is to be very simple to implement. As the free
\textbf{MEMORY}\footnote{Free \textbf{MEMORY} is the \textbf{RAM} that is not
used by the OS, the System variables and the buffers, and hence available to use
for the user and programs.} of the dastaZ80 is about 48,096 bytes, it makes no
sense to have files bigger than that, as will not fit. Therefore, DZFS defines
that \textit{a Block can store only a single file}.

dastaZ80 access the \textbf{DISK} via Logical Block Addressing (LBA), which
is a particularly simple linear addressing schema, in which each sector is
assigned a unique number rather than referring to a cylinder, head, and
sector (CHS) to access the disk.

A typical LBA scheme uses a 28-bit value that allows up to 8.4 GB of data
storage capacity. DZFS schema is as follows:

\begin{center}
    \begin{tabular}{ c c c c }
        \hline
        LBA 3 & LBA 2 & LBA 1 & LBA 0\\
        \hline
        XXXX & XXXX XXXX & BBBB BBBB & BBSS SSSS\\
        \hline
    \end{tabular}   
\end{center}

Where:

\begin{itemize}
    \item S is Sector (6 bits)
    \item B is Block (10 bits)
    \item X not used (12 bits)
\end{itemize}

    % ==========================================================================
    \subsection{DZFS characteristics}
    % ==========================================================================

    \begin{itemize}
        \item \textbf{Bytes per Sector}: 512
        \item \textbf{Sectors per Block}: 64
        \item \textbf{Bytes per Block}: 32,768 (64 * 512). This also defines the
        maximum size of a file and the BAT maximum size.
        \item \textbf{Bytes per BAT entry}: 32
        \item \textbf{BAT entries}: 1024 (32,768 / 32). This also defines the 
        maximum number of files per \textbf{DISK}.
        \item \textbf{Maximum bytes per File}: 1 Block (32,768 bytes)
        \item \textbf{Maximum bytes per DISK}: 1024 Blocks (1 Block = 1 File)
        * 32,768 bytes per Block = 33,554,432 bytes (33.5 MB)
    \end{itemize}

    % ==========================================================================
    \subsection{DISK anatomy}
    % ==========================================================================

    A \textbf{DISK} is divided into areas:

    \begin{itemize}
        \item \textbf{Superblock} = 512 bytes (1 Sector)
        \item \textbf{Block Allocation Table (BAT)} = 1 Block (64 Sectors = 32,768 bytes)
        \item \textbf{Data Area} = 1023 Blocks (65,472 Sectors = 33,521,664 bytes)
    \end{itemize}

        % ==========================================================================
        \subsubsection{Superblock}
        % ==========================================================================
        The first 512 bytes on the \textbf{DISK} contain fundamental information
        about the geometry, and is used by the OS to know how to access every
        other information on the \textbf{DISK}. On IBM PC-compatibles, this is
        known as the \textit{Master Boot Record} or \textit{MBR} for short. In
        DZFS, it is called \textit{Superblock}, as it is an orphan sector that
        doesn't belong to any block.

        \begin{longtable}{ |m{2.2cm}|m{1.3cm}|m{9cm}|m{2.7cm}| }
            \hline
            \rowcolor{lightgray}
            Offset & Length (bytes) & Description & Example\\
            \hline
            \endfirsthead

            \hline
            \rowcolor{lightgray}
            Offset & Length (bytes) & Description & Example\\
            \hline
            \endhead

            \texttt{0x00} & 2 & \textbf{Signature}. Used to check that this is a
            Superblock. Set to \texttt{0xABBA} & \texttt{AB BA}\\
            \hline
            \texttt{0x02} & 1 & \textbf{Not used} & \texttt{00}\\
            \hline
            \texttt{0x03} & 8 & \textbf{File System Identifier}. ASCII values
            for human-readable. Padded with spaces. & \texttt{DZFSV1}\\
            \hline
            \texttt{0x0B} & 4 & \textbf{Volume Serial Number} &
            \texttt{35 2A 15 F2}\\
            \hline
            \texttt{0x0F} & 1 & \textbf{Not used}. & \texttt{00}\\
            \hline
            \texttt{0x10} & 16 & \textbf{Volume Label}. ASCII values. Padded
            with spaces. & \texttt{dastaZ80 Main}\\
            \hline
            \texttt{0x20} & 8 & \textbf{Volume Date Creation}. ASCII values
            (ddmmyyyy). & \texttt{03102022}\\
            \hline
            \texttt{0x28} & 6 & \textbf{Volume Time Creation}. ASCII values
            (hhmmss). & \texttt{142232}\\
            \hline
            \texttt{0x2E} & 2 & \textbf{Bytes per Sector} (in Hexadecimal
            little-endian) & \texttt{00 02}\\
            \hline
            \texttt{0x30} & 1 & \textbf{Sectors per Block} (in Hexadecimal) &
            \texttt{40}\\
            \hline
            \texttt{0x31} & 1 & \textbf{Not used}. & \texttt{00}\\
            \hline
            \texttt{0x32} - \texttt{0x64} & 51 & \textbf{Copyright notice}
            (ASCII value) & \texttt{Copyright 2022David Asta The MIT License
            (MIT)}\\
            \hline
            \texttt{0x65} - 0x1FF & 411 & \textbf{Not used} (filled with
            \texttt{0x00}) & \texttt{00 00 00 00 00 00 00 .........}\\
            \hline
        \end{longtable}

        % ==========================================================================
        \subsubsection{Block Allocation Table (BAT)}
        % ==========================================================================
        The BAT is an area of 32,768 bytes (i.e. 1 Block) on the \textbf{DISK}
        used to store the  details about the files saved in the \textit{Data
        Area}, and is comprised of file descriptors called \textit{entry}. Each
        entry holds information about a single file, and is 32 bytes in length.

        For simplicity, each entry works also as index. The first entry 
        describes the first file on the \textbf{DISK}, the second entry 
        describes the second file, and so on. 

        \begin{longtable}{ |m{1cm}|m{1.3cm}|m{8cm}|m{5cm}| }
            \hline
            \rowcolor{lightgray}
            Offset & Length (bytes) & Description & Example\\
            \hline
            \endfirsthead

            \hline
            \rowcolor{lightgray}
            Offset & Length (bytes) & Description & Example\\
            \hline
            \endhead

            \multirow{4}{4em}{0x00} & \multirow{4}{4em}{14} & \textbf{Filename}
            & \texttt{46 49 4C 45 30 30 30 30 31 20 20 20 20 20}\\
            & & Padded with spaces at the end. &\\
            & & (only allowed A to Z and 0 to 9. No spaces allowed. Cannot start
            with a number.) &\\
            & & First character also indicates 00=available, 7E=deleted (will
            appear as \textasciitilde) &\\
            \hline
            \multirow{6}{4em}{0x0E} & \multirow{6}{4em}{1} & \textbf{Attributes}
            (0=Inactive / 1=Active)
            & Read Only, System file, Executable = 1101 = \texttt{0D}\\
            & & Bit 0 = Read Only &\\
            & & Bit 1 = Hidden &\\
            & & Bit 2 = System &\\
            & & Bit 3 = Executable &\\
            & & Bit 4-7 = File Type (see below) &\\
            \hline
            \multirow{4}{4em}{0x0F} & \multirow{4}{4em}{2} & \textbf{Time created}
            & \texttt{F5 9A}\\
            & & 5 bits for hour (binary number 0-23) &\\
            & & 6 bits for minutes (binary number 0-59) &\\
            & & 5 bits for seconds (binary number seconds / 2) &\\
            \hline
            \multirow{4}{4em}{0x11} & \multirow{4}{4em}{2} & \textbf{Date created}
            & \texttt{69 1B}\\
            & & 7 bits for year since 2000 (max. is year 2127) &\\
            & & 4 bits for month (binary number 0-12) &\\
            & & 5 bits for day (binary number 0-31) &\\
            \hline
            \texttt{0x13} & 2 & \textbf{Time last modified} (same formula as
            Time created) & \texttt{F5 9A}\\
            \hline
            \texttt{0x15} & 2 & \textbf{Date last modified} (same formula as
            Date created) & \texttt{69 1B}\\
            \hline
            \texttt{0x17} & 2 & \textbf{File size in bytes} (little-endian) &
            \texttt{26 00}\\
            \hline
            \texttt{0x19} & 1 & \textbf{File size in sectors} (little-endian) &
            \texttt{01}\\
            \hline
            \texttt{0x1A} & 2 & \textbf{Entry number} (little-endian) &
            \texttt{00 00}\\
            \hline
            \multirow{2}{4em}{0x1C} & \multirow{2}{4em}{2} & \textbf{1st Sector}
            (where the file data starts) & \texttt{41 00}\\
            & & It is calculated when the file is created. The formula is: 65 + 
            64 * entry\_number &\\
            \hline
            \texttt{0x1E} & 2 & \textbf{Load address} (The start address
            little-endian where it will be loaded in RAM) & \texttt{68 25}\\
            \hline
        \end{longtable}

        The value of the bits 4 to 7 of the \textit{Attributes} field define the
        \textit{File Type}:

        \begin{longtable}{ |m{1.4cm}|m{1.7cm}|m{6.8cm}| }
            \hline
            \rowcolor{lightgray}
            Bits 4-7 & File Type & Description\\
            \hline
            \endfirsthead

            \hline
            \rowcolor{lightgray}
            Bits 4-7 & File Type ID & Description\\
            \hline
            \endhead

            \texttt{0x00} & \textbf{USR} & User defined\\
            \texttt{0x01} & \textbf{EXE} & Executable binary\\
            \texttt{0x02} & \textbf{BIN} & Binary (non-executable) data\\
            \texttt{0x03} & \textbf{BAS} & BASIC code\\
            \texttt{0x04} & \textbf{TXT} & Plain ASCII Text file\\
            \texttt{0x05} & \textbf{SC1} & Screen 1 (Graphics I Mode) Picture\\
            \texttt{0x06} & \textbf{FN6} & Font (8x6) for Text Mode\\
            \texttt{0x07} & \textbf{SC2} & Screen 2 (Graphics II Mode) Picture\\
            \texttt{0x08} & \textbf{FN8} & Font (8×8) for Graphics Modes\\
            \texttt{0x09} & \textbf{SC3} & Screen 3 (Multicolour Mode) Picture\\
            \texttt{0x0A} & & Not used\\
            \texttt{0x0B} & & Not used\\
            \texttt{0x0C} & & Not used\\
            \texttt{0x0D} & & Not used\\
            \texttt{0x0E} & & Not used\\
            \texttt{0x0F} & & Not used\\
            \hline
        \end{longtable}

        % ==========================================================================
        \subsubsection{Data Area}
        % ==========================================================================
        The Data Area is the area of the \textbf{DISK} used to store file data 
        (e.g. programs, documents).
        
        It is divided into Blocks of 64 Sectors each.

    % ==========================================================================
    \subsection{How Volume Serial Number is calculated}
    % ==========================================================================
    Calculated by combining the date and time at the point of format:

    \begin{itemize}
        \item first byte is calculated as follows:
        \begin{itemize}
            \item day + miliseconds (converted to hexadecimal)
            \item e.g. 3 + 50 = 53 (0x35)
        \end{itemize}
        \item second byte is calculated as follows:
        \begin{itemize}
            \item month + seconds (converted to hexadecimal)
            \item e.g. 10 + 32 = 42 (0x2A)
        \end{itemize}
        \item last two bytes are calculated as follows:
        \begin{itemize}
            \item (hours [if pm + 12] * 256) + minutes + year (converted to
            hexadecimal)
            \item e.g. (2 + 12 = 14 * 256 = 3584) + 22 + 2012 = 5618
            (\texttt{0x15} \texttt{0xF2})
        \end{itemize}
    \end{itemize}

    % ==========================================================================
    \subsection{How Dates (creation/last modified) are calculated}
    % ==========================================================================

    Dates (day, month, 4-digit year) are converted into two bytes as follows:

    \begin{itemize}
        \item Remove century from year (e.g. $2013 - 2000 = 13$)
        \item Convert resulting number to hexadecimal (e.g. 13 = \texttt{0x0D})
        \item Bitwise Shift Left 9 positions (e.g. \texttt{0x0D} \(\ll\) 9 = 
        \texttt{0x1A00})
        \item Convert month to hexadecimal (e.g. 11 = \texttt{0x0B})
        \item Bitwise Shift Left 5 positions (e.g. \texttt{0x0B} \(\ll\) 5 = 
        \texttt{0x0160})
        \item Add converted month to converted year (e.g. \texttt{0x1A00} + 
        \texttt{0x0160} = \texttt{0x1B60}
        \item Convert day to hexadecimal (e.g. 9 = \texttt{0x09})
        \item Add converted day to the sum of converted month and converted year
        (e.g. \texttt{0x1B60} + \texttt{0x09} = \texttt{0x1B69}
    \end{itemize}

    % ==========================================================================
    \subsection{How Times (creation/last modified) are calculated}
    % ==========================================================================

    Times (hours, minutes, seconds) are converted into two bytes as follows:

    \begin{itemize}
        \item Convert hours to hexadecimal (e.g. 19 = \texttt{0x13})
        \item \item Bitwise Shift Left 3 positions (e.g. \texttt{0x13} \(\ll\) 3 = 
        \texttt{0x98})
        \item Convert minutes to hexadecimal (e.g. 23 = \texttt{0x17})
        \item Bitwise Shift Left 5 positions (e.g. \texttt{0x17} \(\ll\) 5 = 
        \texttt{0x02E0})
        \item Logical OR most significant byte (MSB) of converted minutes with
        less significant byte (LSB) of converted hours (e.g. \texttt{0x02}
        \(\lor\) \texttt{0x98} = \texttt{0x9A})
        \item Logical OR LSB of converted minutes with MSB of converted hours
        (e.g. \texttt{0xE0} \(\lor\) \texttt{0x00} = \texttt{0xE0})
        \item Convert seconds to hexadecimal (e.g. 42 = \texttt{0x2A})
        \item Divide the converted seconds by 2 (e.g. \texttt{0x2A} / 2 = 
        \texttt{0x15})
        \item Add converted seconds to ORed converted hours and minutes
        (e.g. \texttt{0x9AE0} + \texttt{0x15} = \texttt{0x9AF5})
    \end{itemize}

    % ==========================================================================
    \subsection{Block Number, Sector Number and Addresses}
    % ==========================================================================

    To locate files in a Disk Image File it is useful to know how Blocks and
    Sector Numbers relate to the Address in the disk.

    Given a Sector Number (\textit{SecNum}), multiply it by the number of Bytes
    per Sector (512) to obtain the address where the data will start.

    Below is provided a table for quick reference:

    \hfill\break

    % \begin{tabular}{ |m{1.3cm}m{1.3cm}m{2.2cm}|l|m{1.3cm}m{1.3cm}m{2.2cm}|>{\raggedleft\arraybackslash}m{2cm}| }
    %     \hline
    %     \rowcolor{lightgray}
    %     Block & SecNum & Address & & Block & SecNum & Address\\
    %     \hline
    %     \hline
    %     0 (\texttt{0x0000}) & 1     & \texttt{0x00000200} & & 1     & 65  (\texttt{0x0041}) & \texttt{0x00008200} \\
    %     2     & 129   & \texttt{0x00010200} & & 3     & 193 (\texttt{0x00C1}) & \texttt{0x00018200} \\
    %     4     & 257   & \texttt{0x00020200} & & 5     & 321 (\texttt{0x0141}) & \texttt{0x00028200} \\
    %     6     & 385   & \texttt{0x00030200} & & 7     & 449 (\texttt{0x01C1}) & \texttt{0x00038200} \\
    %     8     & 513   & \texttt{0x00040200} & & 9     & 577   & \texttt{0x00048200} \\
    %     10    & 641   & \texttt{0x00050200} & & 11    & 705   & \texttt{0x00058200} \\
    %     12    & 705   & \texttt{0x00060200} & & 13    & 833   & \texttt{0x00068200} \\
    %     14    & 897   & \texttt{0x00070200} & & 15    & 961   & \texttt{0x00078200} \\
    %     16    & 1025  & \texttt{0x00080200} & & 17    & 1089  & \texttt{0x00088200} \\
    %     18    & 1153  & \texttt{0x00090200} & & 19    & 1217  & \texttt{0x00098200} \\
    %     20    & 1281  & \texttt{0x000A0200} & & 21    & 1345  & \texttt{0x000A8200} \\
    %     22    & 1409  & \texttt{0x000B0200} & & 23    & 1473  & \texttt{0x000B8200} \\
    %     ...   & ...   & ...                 & & ...   & ...   & ... \\
    %     1022  & 65409 & \texttt{0x01FF0200} & & 1023  & 65473 & \texttt{0x01FF8200} \\
    %     \hline
    % \end{tabular}

    \begin{center}
        \begin{tabular}{ |m{1.3cm}m{3.3cm}m{2.2cm}|>{\raggedleft\arraybackslash}m{2cm}| }
            \hline
            \rowcolor{lightgray}
            Block & SecNum & Address \\
            \hline
            \hline
            0    & 1     (\texttt{0x0000}) & \texttt{0x00000200} \\
            1    & 65    (\texttt{0x0041}) & \texttt{0x00008200} \\
            2    & 129   (\texttt{0x0081}) & \texttt{0x00010200} \\
            3    & 193   (\texttt{0x00C1}) & \texttt{0x00018200} \\
            4    & 257   (\texttt{0x0101}) & \texttt{0x00020200} \\
            5    & 321   (\texttt{0x0141}) & \texttt{0x00028200} \\
            6    & 385   (\texttt{0x0181}) & \texttt{0x00030200} \\
            7    & 449   (\texttt{0x01C1}) & \texttt{0x00038200} \\
            8    & 513   (\texttt{0x0201}) & \texttt{0x00040200} \\
            9    & 577   (\texttt{0x0241}) & \texttt{0x00048200} \\
            10   & 641   (\texttt{0x0281}) & \texttt{0x00050200} \\
            11   & 705   (\texttt{0x02C1}) & \texttt{0x00058200} \\
            12   & 705   (\texttt{0x0301}) & \texttt{0x00060200} \\
            13   & 833   (\texttt{0x0341}) & \texttt{0x00068200} \\
            14   & 897   (\texttt{0x0381}) & \texttt{0x00070200} \\
            15   & 961   (\texttt{0x03C1}) & \texttt{0x00078200} \\
            16   & 1025  (\texttt{0x0401}) & \texttt{0x00080200} \\
            17   & 1089  (\texttt{0x0441}) & \texttt{0x00088200} \\
            18   & 1153  (\texttt{0x0481}) & \texttt{0x00090200} \\
            19   & 1217  (\texttt{0x04C1}) & \texttt{0x00098200} \\
            20   & 1281  (\texttt{0x0501}) & \texttt{0x000A0200} \\
            21   & 1345  (\texttt{0x0541}) & \texttt{0x000A8200} \\
            22   & 1409  (\texttt{0x0581}) & \texttt{0x000B0200} \\
            23   & 1473  (\texttt{0x05C1}) & \texttt{0x000B8200} \\
            ...  & ...                     & ...                 \\
            1023 & 65473 (\texttt{0xFFC1}) & \texttt{0x01FF8200} \\
            \hline
        \end{tabular}
    \end{center}