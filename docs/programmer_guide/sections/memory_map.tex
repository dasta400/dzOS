% ==========================================================================
\section{Memory Map}
% ==========================================================================

    % ==========================================================================
    \subsection{ROM}
    % ==========================================================================

    The \textbf{ROM} is a 16KB EEPROM, and is divided as follows:

    \begin{center}
        \begin{tabular}{ |m{1.3cm}|m{1.3cm}|m{3.3cm}|m{2.7cm}|>{\raggedleft\arraybackslash}m{2cm}| }
            \hline
            \rowcolor{lightgray}
            \multicolumn{2}{|c|}{Address} &
            \multicolumn{2}{|c|}{Description} &
            Size (bytes) \\
            \hline
            \hline
            \texttt{0x0000} & \texttt{0x0007} & Cold Boot & \textbf{\multirow{3}{4em}{BIOS}} & 8 \\
            \texttt{0x0008} & \texttt{0x0215} & init SIO/2 & & 526 \\
            \texttt{0x0216} & \texttt{0x0FFF} & BIOS code & & 3,562 \\
            \hline
            \texttt{0x1000} & \texttt{0x26C7} & Kernel code & \textbf{\multirow{2}{4em}{Kernel}} & 5,832 \\
            \texttt{0x26B7} & \texttt{0x26C7} & dzOS version build & & 17 \\
            \hline
            \texttt{0x26C8} & \texttt{0x3A88} & CLI code & \textbf{CLI} & 5,057 \\
            \hline
            \texttt{0x3A89} & \texttt{0x3AAB} & Bootstrap & \textbf{BOOTSTRAP} & 35 \\
            \hline
            \texttt{0x3AAC} & \texttt{0x3D9B} & 8x6 Font Pattern set (alphanumeric only 0-Z) & & 752 \\
            \hline
            \texttt{0x3E20} & \texttt{0x3F0F} & BIOS Jumpblock & \textbf{\multirow{2}{4em}{Jumpblocks}} & 240 \\
            \texttt{0x3F10} & \texttt{0x3FFF} & Kernel Jumpblock & & 240 \\
            \hline
        \end{tabular}
    \end{center}

        % ==========================================================================
        \subsubsection{BIOS}
        % ==========================================================================

        The Basic Input/Output System (BIOS) is the part of the DZOS operating
        system that contains the subroutines that communicate directly with the
        hardware. It is, therefore, hardware dependent.

        % ==========================================================================
        \subsubsection{Kernel}
        % ==========================================================================

        The Kernel's main task is to facilitate interactions between any
        software and the BIOS. It is, therefore, hardware independent.

        Though it is entirely possible for any software to either communicate
        directly with the BIOS via it's public functions (aka
        \hyperref[sec:biosjumpblocks]{Jumpblocks}) or even directly with the
        hardware via it's ports, it is highly recommended to use the Kernel
        public functions (aka \hyperref[sec:kerneljumpblocks]{Kernel
        Jumpblocks}). The reason being that the Kernel subroutines are well
        tested and often already contain error handling, but more importantly
        because some functions set/unset values in the so called
        \hyperref[sec:ram_memmap]{System Variables (SYSVARS)}. Failing to
        set/unset those values, could result on undesired behaviour of the OS
        or of some software dependant on it.

        % ==========================================================================
        \subsubsection{CLI}
        % ==========================================================================

        The Command-Line Interface is the the part of the OS that interacts with
        the user, by providing commands that can be entered via the keyboard and
        information provided via the screen.

        A full list of available CLI commands is provided in the
        \textit{dastaZ80 User’s Manual}.

        % ==========================================================================
        \subsubsection{BOOTSTRAP}
        % ==========================================================================

        The bootstrap is a part of the BIOS that is executed first when the
        computer is powered ON or reset (cold boot). The one and only purpose of
        this code is to copy the contents of the \textbf{ROM} into \textbf{RAM}
        and then disable the \textbf{ROM} chip, so that the whole OS resides in
        \textbf{RAM}.

        It does this by first copying all bytes (16 KB) from the \textbf{ROM}
        into the so called \textbf{High RAM} (starting at \texttt{0x8000}). Then
        electronically disables the \textbf{ROM} chip, and finally copies all 16
        KB bytes from \textbf{High RAM} into \textbf{Low RAM} (starting at
        \texttt{0x0000}). Then it gives control to back to the BIOS' cold boot
        subroutine, which starts executing code from \textbf{Low RAM}.

        % ==========================================================================
        \subsubsection{Jumpblocks}
        % ==========================================================================

    % ==========================================================================
    \subsection{RAM}
    % ==========================================================================
    \label{subsec:memmap:ram}

    The \textbf{RAM} is a 64KB SRAM, and is divided as follows:

    \begin{longtable}{ |l|l|l|l|l| }\hline
        \hline
        \rowcolor{lightgray}
        \multicolumn{2}{|c|}{Address} &
        \multicolumn{2}{|c|}{Description} &
        Size (bytes) \\
        \hline
        \hline
        \endfirsthead

        \hline
        \rowcolor{lightgray}
        \multicolumn{2}{|c|}{Address} &
        \multicolumn{2}{|c|}{Description} &
        Size (bytes) \\
        \hline
        \hline
        \endhead

        \texttt{0x4000} & \texttt{0x401F} 
        & \multicolumn{2}{|l|}{\textbf{Stack}} & 32\\
        \hline
        \texttt{0x4020} & \texttt{0x4174} 
        & \multicolumn{2}{|l|}{\textbf{System Variables}} & 389\\
        \hline
        \texttt{0x41A5} & \texttt{0x421F} 
        & \multicolumn{2}{|l|}{\textbf{Reserved for future use}} & 123\\
        \hline
        \texttt{0x4220} & \texttt{0x441F} 
        & \multicolumn{2}{|l|}{\textbf{DISK Buffer}} & 512\\
        \hline
        \texttt{0x4420} & \texttt{0xFFFF} 
        & \multicolumn{2}{|l|}{\textbf{Free RAM}} & 48,096\\
        \hline
    \end{longtable}

        % ==========================================================================
        \subsubsection{Stack}
        % ==========================================================================

        A \textit{Stack} is a list of words (2 bytes) that uses Last In First Out 
        (LIFO) access method. It is used by the \textbf{CPU} to keep track of
        \textbf{MEMORY} addresses when executing a \textit{call} instruction.

        The programmer can also store (\textit{PUSH}) or retrieve (\textit{POP})
        values on/from the top of the stack.

        Usage of the Stack requires very careful attention. doing (\textit{PUSH})
        without the corresponding (\textit{POP}) or vice versa, will set the CPU on
        the wrong path of execution. Most of the time just hanging the computer, but
        also potentially destroying information if an access to disk is triggered by
        the wrong call.

        % ==========================================================================
        \subsubsection{System Variables (SYSVARS)}
        % ==========================================================================
        \label{sec:ram_memmap}

        The area of \textbf{RAM} called \textit{System Variables (SYSVARS)} is an 
        area heavily used by the OS, but it can also be used by a program to
        communicate with the OS.

        The area has been \textit{split} as follows:

        \begin{itemize}
            \item \textbf{SIO}
            \begin{itemize}
                \item \texttt{0x4020} - \textbf{SIO\_CH\_A\_BUFFER} (64 bytes):
                Buffer for SIO Channel A.
                \item \texttt{0x4060} - \textbf{SIO\_CH\_A\_IN\_PTR} (2 bytes)
                \item \texttt{0x4062} - \textbf{SIO\_CH\_A\_RD\_PTR} (2 bytes)
                \item \texttt{0x4064} - \textbf{SIO\_CH\_A\_BUFFER\_USED} (1 byte)
                \item \texttt{0x4065} - \textbf{SIO\_CH\_A\_LASTCHAR} (1 bytes)
                \item \texttt{0x4066} - \textbf{SIO\_CH\_B\_BUFFER} (64 bytes):
                Buffer for SIO Channel B.
                \item \texttt{0x40A6} - \textbf{SIO\_CH\_B\_IN\_PTR} (2 bytes)
                \item \texttt{0x40A6} - \textbf{SIO\_CH\_B\_RD\_PTR} (2 bytes)
                \item \texttt{0x40AA} - \textbf{SIO\_CH\_B\_BUFFER\_USED} (1 byte)
            \end{itemize}
            \item \textbf{DISK Superblock}
            \begin{itemize}
                \item \texttt{0x40AB} - \textbf{DISK\_is\_formatted} (1 byte): tells
                to the OS if the \textbf{DISK} can be used.
                \begin{itemize}
                    \item \texttt{0xFF} = formatted with \textit{DZFS}.
                    \item \texttt{0x00} = not formatted.
                \end{itemize}
                \item \texttt{0x40AC} - \textbf{DISK\_show\_deleted} (1 byte)
                \begin{itemize}
                    \item \texttt{0x00} = do not show deleted files in \textit{cat}
                    command results.
                    \item \texttt{0x01} = show also deleted files in \textit{cat}
                    command results.
                \end{itemize}
                \item \texttt{0x40AD} - \textbf{DISK\_cur\_sector} (2 bytes): current
                Sector being used by the OS.
            \end{itemize}
            \item \textbf{DISK BAT}
            \begin{itemize}
                \item \texttt{0x40AF} - \textbf{DISK\_cur\_file\_name} (14 bytes): 
                Filename of file currently being load or saved.
                \item \texttt{0x40BD} - \textbf{DISK\_cur\_file\_attribs} (1 byte):
                Attributes of file currently being load or saved.
                \begin{itemize}
                    \item Bit 0: if set, file is Read Only.
                    \item Bit 1: if set, file is Hidden (it does not display in
                    \textit{cat} command results).
                    \item Bit 2: if set, file is System (it does not display in
                    \textit{cat} command results).
                    \item Bit 3: if set, file is Executable.
                    \item Bits 4-7: not used.
                \end{itemize}
                \item \texttt{0x40BE} - \textbf{DISK\_cur\_file\_time\_created} (2
                bytes): time when currently being load or saved file was created.
                \item \texttt{0x40C0} - \textbf{DISK\_cur\_file\_date\_created} (2
                bytes): date when currently being load or saved file was created.
                \item \texttt{0x40C2} - \textbf{DISK\_cur\_file\_time\_modified} (2
                bytes): time when currently being load or saved file was last modified.
                \item \texttt{0x40C4} - \textbf{DISK\_cur\_file\_date\_modified} (2
                bytes): date when currently being load or saved file was last modified.
                \item \texttt{0x40C6} - \textbf{DISK\_cur\_file\_size\_bytes} (2
                bytes): size in bytes of file currently being load or saved.
                \item \texttt{0x40C8} - \textbf{DISK\_cur\_file\_size\_sectors} (1
                byte): size in sectors of file currently being load or saved.
                \item \texttt{0x40C9} - \textbf{DISK\_cur\_file\_entry\_number} (2
                bytes): entry number in the BAT, of file currently being load or
                saved.
                \item \texttt{0x40CB} - \textbf{DISK\_cur\_file\_1st\_sector} (2
                bytes): sector number, of the first sector, where the bytes of file
                currently being load or saved are stored in the \textbf{DISK}.
                \item \texttt{0x40CD} - \textbf{DISK\_cur\_file\_load\_addr} (2
                bytes): address where the bytes of file currently being load will be
                stored in \textbf{RAM}.
            \end{itemize}
            \item \textbf{CLI}: buffers used by CLI to store temporary data.
            \begin{itemize}
                \item \texttt{0x40CF} - \textbf{CLI\_prompt\_addr} (2 bytes): The
                address of the CLI Prompt subroutine. Programs that need to return
                control to CLI on exit, MUST jump to the address stored here.
                \item \texttt{0x40D1} - \textbf{CLI\_buffer} (6 bytes): generic
                buffer.
                \item \texttt{0x40D7} - \textbf{CLI\_buffer\_cmd} (16 bytes): when a
                user enters a command and its parameters, the command alone is
                stored here.
                \item \texttt{0x40E7} - \textbf{CLI\_buffer\_parm1\_val} (16 bytes):
                when a user enters a command and its parameters, the first parameter
                is stored here.
                \item \texttt{0x40F7} - \textbf{CLI\_buffer\_parm2\_val} (16 bytes):
                when a user enters a command and its parameters, the second parameter
                is stored here.
                \item \texttt{0x4107} - \textbf{CLI\_buffer\_pgm} (32 bytes): generic
                buffer.
                \item \texttt{0x4127} - \textbf{CLI\_buffer\_full\_cmd} (64 bytes):
                when a user enters a command and its parameters, the entire line
                entered by the user is stored here. This is useful for passing
                parameters to programs called with \textit{run} command.
            \end{itemize}
            \item \textbf{RTC}
            \begin{itemize}
                \item \texttt{0x4167} - \textbf{RTC\_hour} (1 byte): 24h format,
                in hexadecimal (\texttt{0x00}-\texttt{0x17}).
                \item \texttt{0x4168} - \textbf{RTC\_minutes} (1 byte): in
                hexadecimal (\texttt{0x00}-\texttt{0x3B}).
                \item \texttt{0x4169} - \textbf{RTC\_seconds} (1 byte): in
                hexadecimal (\texttt{0x00}-\texttt{0x3B}).
                \item \texttt{0x416A} - \textbf{RTC\_century} (1 byte): 20 part of
                year 20xx, in hexadecimal (\texttt{0x14} = 20).
                \item \texttt{0x416B} - \textbf{RTC\_year} (1 byte): xx part of
                year 20xx, in hexadecimal (e.g. \texttt{0x16} = 22). The \textbf{RTC}
                supports until 2079, therefore maximum value is \texttt{0x4F}.
                \item \texttt{0x416C} - \textbf{RTC\_year4} (2 bytes): four digit
                year, in hexadecimal (e.g. \texttt{0x07E6} = 2022). The \textbf{RTC}
                supports until 2079, therefore maximum value is \texttt{0x081F}.
                \item \texttt{0x416E} - \textbf{RTC\_month} (1 byte): in
                hexadecimal (\texttt{0x00}-\texttt{0x0C}).
                \item \texttt{0x416F} - \textbf{RTC\_day} (1 byte): in hexadecimal
                (\texttt{0x00}-\texttt{0x1F}).
                \item \texttt{0x4170} - \textbf{RTC\_day\_of\_the\_week} (1 byte):
                \texttt{0x00}=Sunday, \texttt{0x01}=Monday, \texttt{0x02}=Tuesday,
                \texttt{0x03}=Wednesday, \texttt{0x04}=Thursday, \texttt{0x05}=
                Friday, \texttt{0x06}=Saturday
            \end{itemize}
            \item \textbf{Math}
            \begin{itemize}
                \item \texttt{0x4171} - \texttt{MATH\_CRC} (2 bytes): CRC-16 CRC.
                \item \texttt{0x4173} - \texttt{MATH\_polynomial} (2 bytes): CRC-16
                Polynomial.
            \end{itemize}
            \item \textbf{Generic}
            \begin{itemize}
                \item \texttt{0x4175} - \textbf{FDD\_detected} (1 byte): 
                1=FDD detected, 0=Not detected
                \item \texttt{0x4176} - \textbf{SD\_images\_num} (1 byte): number
                of Disk Image Files found by \textbf{ASMDC}.
                \item \texttt{0x4177} - \textbf{DISK\_current} (1 byte): current 
                \textbf{DISK} unit active. All disk operations will be on this
                \textbf{DISK}.
                \item \texttt{0x4178} - \textbf{DISK\_status} (1 byte): status of
                the \textbf{FDD}.
                \begin{itemize}
                    \item Low Nibble (\texttt{0x00} if all OK)
                    \begin{itemize}
                        \item bit 0 = not used.
                        \item bit 1 = not used.
                        \item bit 2 = set if last command resulted in error.
                        \item bit 3 = not used.
                    \end{itemize}
                    \item High Nibble: error code of last operation.
                \end{itemize}
                \item \texttt{0x4179} - \textbf{DISK\_file\_type} (1 byte): File
                Type when creating (\textit{save}) next file.
                \item \texttt{0x417A} - \textbf{DISK\_loadsave\_addr} (2 bytes): see
                \hyperref[sec:howto_readdata]{Read data from DISK} and 
                \hyperref[sec:howto_writedata]{Write data to DISK}.
                \item \texttt{0x417C} - \textbf{tmp\_addr1} (2 bytes): temporary
                storage for an address.
                \item \texttt{0x417E} - \textbf{tmp\_addr2} (2 bytes): temporary
                storage for an address.
                \item \texttt{0x4180} - \textbf{tmp\_addr3} (2 bytes): temporary
                storage for an address.
                \item \texttt{0x4182} - \textbf{tmp\_byte} (1 byte): temporary
                storage for a byte.
                \item \texttt{0x4183} - \textbf{tmp\_byte2} (1 byte): temporary
                storage for a byte.
            \end{itemize}
            \item \textbf{VDP}
            \begin{itemize}
                \item \texttt{0x4184} - \textbf{NMI\_enable}: Enable (1) / Disable
                (0) the execution of the \hyperref[sec:nmi]{NMI subroutine}.
                \item \texttt{0x4185} - \textbf{NMI\_usr\_jump}: Enable (1) / Disable
                (0) the user configurable \textit{BIOS\_NMI\_JP} jump of the 
                \hyperref[sec:nmi]{NMI subroutine}.
                \item \texttt{0x4186} - \textbf{VDP\_cur\_mode}:
                \begin{itemize}
                    \item 0 = Text Mode
                    \item 1 = Graphics I Mode
                    \item 2 = Graphics II Mode
                    \item 3 = Multicolour Mode
                    \item 4 = Graphics II Mode Bitmapped
                \end{itemize}
                \item \texttt{0x4187} - \textbf{VDP\_cursor\_x} (1 byte): Current
                horizontal position of the cursor on the \textbf{VDP} screen.
                \item \texttt{0x4188} - \textbf{VDP\_cursor\_y} (1 byte): Current
                vertical position of the cursor on the \textbf{VDP} screen.
                \item \texttt{0x4189} - \textbf{VDP\_PTRNTAB\_addr} (2 bytes):
                Address of current Mode’s Pattern Table.
                \item \texttt{0x418B} - \textbf{VDP\_NAMETAB\_addr} (2 bytes):
                Address of current Mode’s Name Table.
                \item \texttt{0x418D} - \textbf{VDP\_COLRTAB\_addr} (2 bytes):
                Address of current Mode’s Colour Table.
                \item \texttt{0x418F} - \textbf{VDP\_SPRPTAB\_addr} (2 bytes):
                Address of current Mode’s Sprite Pattern Table.
                \item \texttt{0x4191} - \textbf{VDP\_SPRATAB\_addr} (2 bytes):
                Address of current Mode’s Sprite Attribute Table.
                \item \texttt{0x4193} - \textbf{VDP\_jiffy\_byte1} (1 byte):
                \hyperref[subsec:jiffy_counter]{Jiffy Counter}'s byte 1.
                \item \texttt{0x4194} - \textbf{VDP\_jiffy\_byte2} (1 byte):
                \hyperref[subsec:jiffy_counter]{Jiffy Counter}'s byte 2.
                \item \texttt{0x4195} - \textbf{VDP\_jiffy\_byte3} (1 byte):
                \hyperref[subsec:jiffy_counter]{Jiffy Counter}'s byte 3.
            \end{itemize}
            \item \textbf{System Colour Scheme}
            \\These are the default colours used by messages displayed on the
            \textbf{High Resolution Screen} (VGA), and can be re-defined by the
            user by changing each byte value.
            \begin{itemize}
                \item \texttt{0x4196} - \textbf{col\_kernel\_debug} (1 byte):
                    default is Cyan.
                \item \texttt{0x4197} - \textbf{col\_kernel\_disk} (1 byte):
                    default is Magenta.
                \item \texttt{0x4198} - \textbf{col\_kernel\_error} (1 byte):
                    default is Red.
                \item \texttt{0x4199} - \textbf{col\_kernel\_info} (1 byte):
                    default is Green.
                \item \texttt{0x419A} - \textbf{col\_kernel\_notice} (1 byte):
                    default is Yellow.
                \item \texttt{0x419B} - \textbf{col\_kernel\_warning} (1 byte):
                    default is Magenta.
                \item \texttt{0x419C} - \textbf{col\_kernel\_welcome} (1 byte):
                    default is Blue.
                \item \texttt{0x419D} - \textbf{col\_CLI\_debug} (1 byte):
                    default is Cyan.
                \item \texttt{0x419E} - \textbf{col\_CLI\_disk} (1 byte):
                    default is Magenta.
                \item \texttt{0x419F} - \textbf{col\_CLI\_error} (1 byte):
                    default is Red.
                \item \texttt{0x41A0} - \textbf{col\_CLI\_info} (1 byte):
                    default is Green.
                \item \texttt{0x41A1} - \textbf{col\_CLI\_input} (1 byte):
                    default is White.
                \item \texttt{0x41A2} - \textbf{col\_CLI\_notice} (1 byte):
                    default is Yellow.
                \item \texttt{0x41A3} - \textbf{col\_CLI\_prompt} (1 byte):
                    default is Blue.
                \item \texttt{0x41A4} - \textbf{col\_CLI\_warning} (1 byte):
                    default is Magenta.
            \end{itemize}
        \end{itemize}

        % ==========================================================================
        \subsubsection{DISK Buffer}
        % ==========================================================================

        Read and Write operations on \textbf{DISK} are done Sector by Sector (i.e 
        512 Bytes).

        When loading a file, dzOS asks \textbf{ASMDC} for the first 512 bytes of the
        file, and stores it in this buffer. After the bytes are moved to
        \textbf{RAM}, dzOS asks \textbf{ASMDC} for the next 512 bytes, and so on
        until the file is read entirely.

        When saving a file, dzOS copies the first 512 bytes of the file from
        \textbf{RAM} to this buffer. After sending the bytes to \textbf{ASMDC}, dzOS
        copies the next 512 bytes of the file, and so on until the file is saved
        entirely.

        When doing a \textit{cat} of a \textbf{DISK}, dzOS asks \textbf{ASMDC} for
        the first 512 bytes of the BAT, and stores it in this buffer. After the list
        of files is shown on the screen, dzOS asks \textbf{ASMDC} for the next 512
        bytes, and so on until the entire catalogue has been shown.

    % ==========================================================================
    \subsection{VDP}
    % ==========================================================================
    \label{subsec:vdp_memmap}

        % ==========================================================================
        \subsubsection{Text Mode}
        % ==========================================================================
        \begin{center}
            \begin{tabular}{c r}
                \hline
                Pattern Table           & \texttt{0x0000}\\
                \hline
                Name Table              & \texttt{0x0800}\\
                \hline
                \multirow{2}{*}{UNUSED} & \texttt{0x0BC0}\\
                                        & \texttt{0x3FFF}\\
                \hline
            \end{tabular}
        \end{center}

        % ==========================================================================
        \subsubsection{Graphics I Mode}
        % ==========================================================================
        \begin{center}
            \begin{tabular}{c r}
                \hline
                Sprites Patterns         & \texttt{0x0000}\\
                \hline
                Pattern Table           & \texttt{0x0800}\\
                \hline
                SPRITE ATTRIBUTTES      & \texttt{0x1000}\\
                \hline
                UNUSED                  & \texttt{0x1080}\\
                \hline
                Name Table              & \texttt{0x1400}\\
                \hline
                UNUSED                  & \texttt{0x1800}\\
                \hline
                COLOUR TABLE            & \texttt{0x2000}\\
                \hline
                \multirow{2}{*}{UNUSED} & \texttt{0x2020}\\
                                        & \texttt{0x3FFF}\\
                \hline
            \end{tabular}
        \end{center}

        % ==========================================================================
        \subsubsection{Graphics II Mode}
        % ==========================================================================
        \begin{center}
            \begin{tabular}{c r}
                \hline
                Pattern Table           & \texttt{0x0000}\\
                \hline
                Sprites Patterns         & \texttt{0x1800}\\
                \hline
                COLOUR TABLE            & \texttt{0x2000}\\
                \hline
                Name Table              & \texttt{0x3800}\\
                \hline
                SPRITE ATTRIBUTTES      & \texttt{0x3B00}\\
                \hline
                \multirow{2}{*}{UNUSED} & \texttt{0x3C00}\\
                                        & \texttt{0x3FFF}\\
                \hline
            \end{tabular}
        \end{center}

        % ==========================================================================
        \subsubsection{Multicolour Mode}
        % ==========================================================================
        \begin{center}
            \begin{tabular}{c r}
                \hline
                Sprites Patterns         & \texttt{0x0000}\\
                \hline
                Pattern Table           & \texttt{0x0800}\\
                \hline
                UNUSED                  & \texttt{0x0E00}\\
                \hline
                SPRITE ATTRIBUTTES      & \texttt{0x1000}\\
                \hline
                UNUSED                  & \texttt{0x1080}\\
                \hline
                Name Table              & \texttt{0x1400}\\
                \hline
                \multirow{2}{*}{UNUSED} & \texttt{0x1700}\\
                                        & \texttt{0x3FFF}\\
                \hline
            \end{tabular}
        \end{center}