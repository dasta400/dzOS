% ==========================================================================
\section{How To}
% ==========================================================================

    % ==========================================================================
    \subsection{Read data from DISK}
    % ==========================================================================
    \label{sec:howto_readdata}
    Given \texttt{DISK\_is\_formatted} is equal to \texttt{0xFF} (i.e.
    \textbf{DISK} is formatted with DZFS file system), call
    \hyperref[func:fkrndzfsloadfiletoram]{F\_KRN\_DZFS\_LOAD\_FILE\_TO\_RAM}
    with \texttt{DE} equal to first sector (512 bytes) to read and \texttt{IX}
    equal to how many sectors to read.

    Read bytes will be copied into \textbf{MEMORY}, following these rules:

    \begin{itemize}
        \item if \textit{\hyperref[sec:ram_memmap]{DISK\_loadsave\_addr}} $<>$
        0, load bytes to this address.
        \item if \textit{\hyperref[sec:ram_memmap]{DISK\_loadsave\_addr}} = 0,
        \begin{itemize}
            \item if \textit{\hyperref[sec:ram_memmap]{DISK\_cur\_file\_load\_addr}}
            $<>$ 0, load bytes to this address.
            \item if \textit{\hyperref[sec:ram_memmap]{DISK\_cur\_file\_load\_addr}}
            = 0, load bytes to start of \hyperref[subsec:memmap:ram]{Free RAM}
            (\texttt{0x4420}).
        \end{itemize}
    \end{itemize}

    % ==========================================================================
    \subsection{Write data to DISK}
    % ==========================================================================
    \label{sec:howto_writedata}
    Given \texttt{DISK\_is\_formatted} is equal to \texttt{0xFF} (i.e.
    \textbf{DISK} is formatted with DZFS file system):

    \begin{itemize}
        \item Store the filename (in ASCII) somewhere in \textbf{MEMORY}.
        \item call \hyperref[func:fkrndzfsgetfilebatentry]{F\_KRN\_DZFS\_GET\_FILE\_BATENTRY},
            with \texttt{HL} equal to the \texttt{MEMORY} address where the
            filename is stored. If a file with the specified filename does not
            exist, flag \texttt{z} will be set to indicate that it is OK to save
            the file.
        \item call \hyperref[func:fkrndzfscreatenewfile]{F\_KRN\_DZFS\_CREATE\_NEW\_FILE},
            with:
            \begin{itemize}
                \item \texttt{HL} equal to the address in \textbf{MEMORY} of
                first byte to be stored.
                \item \texttt{BC} equal to the total number of bytes to
                be stored.
                \item \texttt{IX} equal to the address in \textbf{MEMORY}
                where the filename is stored.
                \item \textit{\hyperref[sec:ram_memmap]{DISK\_loadsave\_addr}}
                equal to:
                \begin{itemize}
                    \item zero, will use the address in \textbf{MEMORY} of first
                    byte as the load address when loading the file (i.e. 
                    \textit{\hyperref[sec:ram_memmap]{DISK\_loadsave\_addr}}).
                    \item non zero, will use this number as the load address
                    when loading the file (i.e. 
                    \textit{\hyperref[sec:ram_memmap]{DISK\_loadsave\_addr}}).
                \end{itemize}
            \end{itemize}
    \end{itemize}

    % ==========================================================================
    \subsection{Convert between HEX and DEC and ASCII}
    % ==========================================================================
    In many situations your programs will need to convert between different
    number notations (hexadecimal, decimal, ASCII). For example, all characters
    typed by the user are read by the function
    \hyperref[func:fbiosserialconina]{F\_BIOS\_SERIAL\_CONIN\_A}, which stores
    the ASCII value of the pressed key in the \texttt{A} register. In order to
    do manipulations of data, our program will need to convert this ASCII data
    into either hexadecimal or decimal notation.

    Take as an example the CLI command for saving files to disk (\textit{save}).
    As shown in the \textit{dastaZ80 User's Manual} section \textit{5.3 Disk
    Commands}, this command takes two parameters: \textit{$<$start\_address$>$},
    which is expressed in hexadecimal, and \textit{$<$number\_of\_bytes$>$}, which
    is expressed in decimal. But in both cases,
    \hyperref[func:fbiosserialconina]{F\_BIOS\_SERIAL\_CONIN\_A } will give
    us (in the \texttt{A} register) the ASCII representation of the numbers typed
    by the user.

    Before we can set a pointer to the memory address specified by 
    \textit{$<$start\_address$>$}, and set our counter to 
    \textit{$<$number\_of\_bytes$>$}, we need to convert those ASCII numbers
    into hexadecimal and decimal respectively.

    The Kernel, offers a series of functions to help the programmer with the
    conversions:

    \begin{itemize}
        \item \hyperref[func:fkrnasciiadrtohex]{F\_KRN\_ASCIIADR\_TO\_HEX}:
        Converts ASCII 4 chars to HEX 2 bytes. (e.g. 32 35 37 30 to \texttt{0x2570})
        \item \hyperref[func:fkrnasciitohex]{F\_KRN\_ASCII\_TO\_HEX}: Converts
        ASCII 2 chars to HEX 1 byte. (e.g. 33 45 to \texttt{0x3E})
        \item \hyperref[func:fkrnhextoascii]{KRN\_HEX\_TO\_ASCII}: Converts
        HEX 1 byte to ASCII 2 chars. (e.g. \texttt{0x3E} to 33 45)
        \item \hyperref[func:fkrnbcdtobin]{F\_KRN\_BCD\_TO\_BIN}: Converts a
        byte of BCD to a byte of hexadecimal. (e.g. 12 is converted into \texttt{0x0C}).
        \item \hyperref[func:fkrnbintobcd4]{F\_KRN\_BIN\_TO\_BCD4}: Converts HEX
        1 byte to DEC 4 digits. (e.g. \texttt{0x80} to 0128)
        \item \hyperref[func:fkrnbintobcd6]{F\_KRN\_BIN\_TO\_BCD6}: Converts HEX
        2 bytes to DEC 6 digits. (e.g. \texttt{0xFFF} to 065535)
        \item \hyperref[func:fkrnbcdtoascii]{F\_KRN\_BCD\_TO\_ASCII}: Converts
        DEC 6 digits to ASCII 6 chars. (e.g. 512 to 30 30 30 35 31 32)
        \item \hyperref[func:fkrnbintoascii]{F\_KRN\_BIN\_TO\_ASCII}: Converts
        HEX 2 bytes to ASCII string. (e.g. \texttt{0x7FFF} to 33 32 37 36 37)
        \item \hyperref[func:fkrndectobin]{F\_KRN\_DEC\_TO\_BIN}:
        Converts HEX n bytes to ASCII string. First byte tells the number of
        bytes to convert (e.g. 05 33 32 37 36 37 to \texttt{0x7FFF})
    \end{itemize}

    % ==========================================================================
    \subsection{How to display Sprites}
    % ==========================================================================
    
    A \textit{sprite} is a two-dimensional bitmap that can be made to move
    and change shape in the screen with very little programming effort, thanks
    to the \textbf{VDP} support for hardware sprites.

    The \textbf{VDP} has 32 sprite planes each of which can contain a single
    sprite.

    Sprites can be of two sizes; 8x8 pixels (Size 0) or 16x16 (Size 1) pixels.
    There is also the possibility to magnify a sprite, thus a 8x8 sprite becomes
    16x16, and a 16x16 sprite becomes 32x32. Unfortunatelly, the sprite
    resolution is cut in half.

    Two tables are required in \textbf{VRAM} in order to display a sprite; the 
    \textit{Sprite Attribute Table} and the \textit{Sprite Pattern Table}.

    The address of these tables will change depending on which mode we are using.
    Refer to the \hyperref[subsec:vdp_memmap]{VDP Memory Map} to know the addresses.

    The \textit{Sprite Pattern Table} defines the shape of each sprite. It takes
    8 bytes to define the pattern of Size 0 (8x8) sprite and 32 bytes for a Size
    1 (16x16) sprite. The table has a maximum length of 2048 bytes, therefore a
    maximum of 256 patterns can be defined for Size 0 and 64 for Size 1 sprites.

    The \textit{Sprite Attribute Table} contains four bytes of information for
    every sprite. The table is ordered sequentally, where the first four bytes
    contain the information for sprite 0, the next 4 bytes contain the
    information for sprite 1, and so on.

    The information of the four bytes in the \textit{Sprite Attribute Table} is
    as follows:

    \begin{itemize}
        \item Vertical coordinate: determines the distance (in pixels) the
                sprite will be offset from the top of the screen. The position
                is measured relative to the upper left hand corner of the sprite.
                A value of \texttt{0xFF} will put the sprite at the top of the
                screen. A value of \texttt{0xBF} will put the sprite at the
                bottom of the screen. But because the position is meassured
                relative to the upper left hand corner of the sprite, it will
                not appear.
        \item Horizontal coordinate: determines the distance (in pixels) the
                sprite will be offset from the left hand side of the screen. A
                value of \texttt{0x00} will put the sprite at the left hand side
                of the screen. A value of \texttt{0xFF} will put the sprite at
                the right hand side of the screen. But because the position is
                meassured relative to the upper left hand corner of the sprite,
                it will not appear.
        \item Sprite Name Pointer: the value in this byte determines which
                pattern from the \textit{Sprite Pattern Table} will be used as
                the sprite's shape. This highly simplify the production of
                sprite animations, as just the pointer needs to be changed.
        \item Colour and Early Clock Bit:
            \begin{itemize}
                \item The lower nibble define the sprite
                        colour, which can be any of the \hyperref[sec:vdp_colours]{
                        VDP Composite colours}.
                \item The MSB of the higher nibble is called the \textit{Early
                        Clock Bit}, and when set as 1 it shifts the position of
                        the sprite to the left by 32 pixels.
            \end{itemize}
    \end{itemize}

        % ==========================================================================
        \subsubsection{Example}
        % ==========================================================================

        Lets assume we are working in mode 2 (Graphics II Mode). Following the
        \hyperref[subsec:vdp_memmap]{VDP Memory Map}, we can see that the Sprite
        Pattern Table is located at \texttt{0x1800} and the Sprite Attribute
        Table at \texttt{0x3B00}.

        First we will fill the patterns of Sprite 0 with the 8x8 sprite from the
        \textit{Video Display Processors Programmer's Guide}\cite{ti1}. Hence we
        need to assign the following values:

        \begin{itemize}
            \item \texttt{0x1800} = \texttt{0x10}
            \item \texttt{0x1801} = \texttt{0x10}
            \item \texttt{0x1802} = \texttt{0xFE}
            \item \texttt{0x1803} = \texttt{0x7C}
            \item \texttt{0x1804} = \texttt{0x38}
            \item \texttt{0x1805} = \texttt{0x6C}
            \item \texttt{0x1806} = \texttt{0x44}
            \item \texttt{0x1807} = \texttt{0x00}
        \end{itemize}

        At this point, most probably (if just started the computer) we won't see
        anything yet, beacuse the colour is set to \texttt{0x00} (Transparent).

        Lets change the colour byte for sprite 0 in the Sprite Attribute Table
        (\texttt{0x3B03}) to \texttt{0x03} (Light Green).

        You should be seeing a litlle green star at the top left of the screen.

        By changing the bytes corresponding to the Y position \texttt{0x3B00}
        and X position \texttt{0x3B01}, we can move the sprite around the
        screen. Lets try for example \texttt{0x3B00} = \texttt{0x5F} and
        \texttt{0x3B01} = \texttt{0x7F} to display the sprite at the center of
        the screen.

        This can be easily tested from the command line, by using the command 
        \textit{vpoke} to change the bytes. For example, \textit{vpoke 1800,10}.


    % ==========================================================================
    \subsection{Develop software for dzOS}
    % ==========================================================================

        % ==========================================================================
        \subsubsection{Available RAM}
        % ==========================================================================
        Programs can be loaded from disk to any area of \textbf{RAM}. Nevertheless,
        addresses below \texttt{0x4420} SHOULD not be used, at these contain the
        Operating System's variables. Modifying these without the proper care will
        result in undesired behaviour, system crash or even lost of data on the disk.
        Therefore, taking in consideration that the free RAM area starts at
        \texttt{0x4420} and ends at \texttt{0xFFFF}, the programmer can load
        programs of maximum 48,095 bytes (48 KB).

        % ==========================================================================
        \subsubsection{Storing your variables}
        % ==========================================================================
        Variables for programs can be store anywhere in the free \textbf{RAM} space.

        The OS is having its own internal variables that can be accessed by the user.
        Also, some variables are used only by CLI and therefore could be re-used
        during the execution of a program.\\

        Refer to the section \hyperref[sec:ram_memmap]{System Variables (SYSVARS)}
        on this guide to know the exact locations.

        \begin{itemize}
            \item The \textbf{DISK Superblock} and \textbf{DISK BAT} areas can be
            re-used if you are not using \textbf{DISK} routines.
            \item The \textbf{CLI} area can safely be re-used in your program, as
            the CLI is not running meanwhile your program is.
            \item The \textbf{RTC} area can be re-used if you are not calling any
            \textbf{RTC} routines.
            \item The \textbf{Math} area can be re-used if you are not calling any
            \textbf{Math} routines.
            \item The \textbf{SIO}, \textbf{Generic} and \textbf{VDP} areas MUST not
            be touched.
        \end{itemize}

        All in all, you may end up having some extra 700 bytes here.

        % ==========================================================================
        \subsubsection{Receiving parameters from CLI}
        % ==========================================================================
        When a user types a command in CLI, the entered command is stored in an area
        of 64 bytes in the \hyperref[sec:ram_memmap]{System Variables (SYSVARS)} 
        called \textit{CLI\_buffer\_full\_cmd}. From there, you can read the full
        command, which will be the name of your binary program, and the parameter or
        parameters.

        % ==========================================================================
        \subsubsection{Returning to CLI}
        % ==========================================================================

        If your program allows the user to return to CLI, it must then jump to the
        loop subroutine known as \textit(CLI Prompt). The address of this subroutine
        is stored in the \hyperref[sec:ram_memmap]{System Variables (SYSVARS)}
        \textit{CLI\_prompt\_addr}.

        Simply make your program to load the value stored at that location and jump
        (\textit{jp}) to it.

        % ==========================================================================
        \subsubsection{Developing with Z80 Assembler}
        % ==========================================================================
        In order for dzOS to know where to load the program in \textbf{RAM}, the
        executable code must provide the load address. For compatibility with SDCC
        \footnote{Small Device C Compiler (SDCC) is a retargettable, optimizing
        Standard C (ANSI C89, ISO C99, ISO C11) compiler suite that targets (amongst
        others) the Zilog Z80 based MCUs. (http://sdcc.sourceforge.net/)}, we will
        store it in the bytes 3 and 4 of the executable.

        For programs developed in Z80 Assembler, add the following at the top of the
        source code:

        % \lstset{language={[z80]Assembler}}
        \begin{lstlisting}
        .ORG    $4420           ; start of code at
                                ;   start of free RAM
        jp      $4425           ; first instruction
                                ;   must jump to the
                                ;   executable code
        .BYTE   $20, $44        ; load address
                                ;   (values must be
                                ;   same as .org above)

        .ORG    $4425           ; start of program
                                ;   (must be same as jp above)
        ; your program here
        \end{lstlisting}

        The first .ORG (\textit{.ORG \$4420}) indicates the start address used for
        creating the binary file after compilation.

        \texttt{0x4420} is where the \hyperref[subsec:memmap:ram]{Free RAM}
        starts, giving you 48 KB for your program. Programs SHOULD not be loaded
        at a lower address, for the reason explained before.

        The first instruction MUST be a jump (\textit{jp}) instruction to the actual
        executable code (i.e. your program code) The \textit{.BYTE} instruction just
        inserts the two bytes after the jump instruction. The values MUST be in
        hexadecimal little-endian format.

        Because the jp instruction in Z80 is translated as \textit{C3 nn nn} (where
        \textit{nn} are the bytes where to jump), this will use the first three
        bytes (\texttt{0x00}, \texttt{0x01}, \texttt{0x02}) in the binary, therefore
        we store the load address at bytes 3 and 4 and your program can start just
        after, at byte \texttt{0x05}.

        Once assembled, the binary will be loaded by dzOS at the load address, and
        when executed, the first thing that will happen is a \textit{jp} instruction
        and then the execution will continue from the executable code of your
        program.

        If your program allows the user to return to CLI, add the following on your
        source code:

        % \lstset{language={[z80]Assembler}}
        \begin{lstlisting}
        ld      HL, (CLI_prompt_addr)   ; return control
        jp      (HL)                    ;   to CLI
        \end{lstlisting}

        For convenience, two files are provided in the Github repository
        \footnote{https://github.com/dasta400/dzSoftware}:
        \textit{\_header.inc} and \textit{\_template.asm}

        % ==========================================================================
        \subsubsection{Developing with SDCC}
        % ==========================================================================

        In the Github repository, there is a file (\textit{crt0.s} that sets:
        \begin{itemize}
            \item the start address for the binary at \texttt{0x4420}
            \item the values \texttt{0x20} and \texttt{0x44} in the binary at bytes
            5 and 6.
            \item first instruction of your program to be started located at
            \texttt{0x4425}
        \end{itemize}

        Therefore, by using this file all programs will be loaded at the correct
        address.